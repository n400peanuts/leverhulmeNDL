% Options for packages loaded elsewhere
\PassOptionsToPackage{unicode}{hyperref}
\PassOptionsToPackage{hyphens}{url}
%
\documentclass[
]{article}
\usepackage{lmodern}
\usepackage{amssymb,amsmath}
\usepackage{ifxetex,ifluatex}
\ifnum 0\ifxetex 1\fi\ifluatex 1\fi=0 % if pdftex
  \usepackage[T1]{fontenc}
  \usepackage[utf8]{inputenc}
  \usepackage{textcomp} % provide euro and other symbols
\else % if luatex or xetex
  \usepackage{unicode-math}
  \defaultfontfeatures{Scale=MatchLowercase}
  \defaultfontfeatures[\rmfamily]{Ligatures=TeX,Scale=1}
\fi
% Use upquote if available, for straight quotes in verbatim environments
\IfFileExists{upquote.sty}{\usepackage{upquote}}{}
\IfFileExists{microtype.sty}{% use microtype if available
  \usepackage[]{microtype}
  \UseMicrotypeSet[protrusion]{basicmath} % disable protrusion for tt fonts
}{}
\makeatletter
\@ifundefined{KOMAClassName}{% if non-KOMA class
  \IfFileExists{parskip.sty}{%
    \usepackage{parskip}
  }{% else
    \setlength{\parindent}{0pt}
    \setlength{\parskip}{6pt plus 2pt minus 1pt}}
}{% if KOMA class
  \KOMAoptions{parskip=half}}
\makeatother
\usepackage{xcolor}
\IfFileExists{xurl.sty}{\usepackage{xurl}}{} % add URL line breaks if available
\IfFileExists{bookmark.sty}{\usepackage{bookmark}}{\usepackage{hyperref}}
\hypersetup{
  pdftitle={FLO replication - Preprocessing + analysis + results summary},
  pdfauthor={Eva},
  hidelinks,
  pdfcreator={LaTeX via pandoc}}
\urlstyle{same} % disable monospaced font for URLs
\usepackage[margin=1in]{geometry}
\usepackage{color}
\usepackage{fancyvrb}
\newcommand{\VerbBar}{|}
\newcommand{\VERB}{\Verb[commandchars=\\\{\}]}
\DefineVerbatimEnvironment{Highlighting}{Verbatim}{commandchars=\\\{\}}
% Add ',fontsize=\small' for more characters per line
\usepackage{framed}
\definecolor{shadecolor}{RGB}{248,248,248}
\newenvironment{Shaded}{\begin{snugshade}}{\end{snugshade}}
\newcommand{\AlertTok}[1]{\textcolor[rgb]{0.94,0.16,0.16}{#1}}
\newcommand{\AnnotationTok}[1]{\textcolor[rgb]{0.56,0.35,0.01}{\textbf{\textit{#1}}}}
\newcommand{\AttributeTok}[1]{\textcolor[rgb]{0.77,0.63,0.00}{#1}}
\newcommand{\BaseNTok}[1]{\textcolor[rgb]{0.00,0.00,0.81}{#1}}
\newcommand{\BuiltInTok}[1]{#1}
\newcommand{\CharTok}[1]{\textcolor[rgb]{0.31,0.60,0.02}{#1}}
\newcommand{\CommentTok}[1]{\textcolor[rgb]{0.56,0.35,0.01}{\textit{#1}}}
\newcommand{\CommentVarTok}[1]{\textcolor[rgb]{0.56,0.35,0.01}{\textbf{\textit{#1}}}}
\newcommand{\ConstantTok}[1]{\textcolor[rgb]{0.00,0.00,0.00}{#1}}
\newcommand{\ControlFlowTok}[1]{\textcolor[rgb]{0.13,0.29,0.53}{\textbf{#1}}}
\newcommand{\DataTypeTok}[1]{\textcolor[rgb]{0.13,0.29,0.53}{#1}}
\newcommand{\DecValTok}[1]{\textcolor[rgb]{0.00,0.00,0.81}{#1}}
\newcommand{\DocumentationTok}[1]{\textcolor[rgb]{0.56,0.35,0.01}{\textbf{\textit{#1}}}}
\newcommand{\ErrorTok}[1]{\textcolor[rgb]{0.64,0.00,0.00}{\textbf{#1}}}
\newcommand{\ExtensionTok}[1]{#1}
\newcommand{\FloatTok}[1]{\textcolor[rgb]{0.00,0.00,0.81}{#1}}
\newcommand{\FunctionTok}[1]{\textcolor[rgb]{0.00,0.00,0.00}{#1}}
\newcommand{\ImportTok}[1]{#1}
\newcommand{\InformationTok}[1]{\textcolor[rgb]{0.56,0.35,0.01}{\textbf{\textit{#1}}}}
\newcommand{\KeywordTok}[1]{\textcolor[rgb]{0.13,0.29,0.53}{\textbf{#1}}}
\newcommand{\NormalTok}[1]{#1}
\newcommand{\OperatorTok}[1]{\textcolor[rgb]{0.81,0.36,0.00}{\textbf{#1}}}
\newcommand{\OtherTok}[1]{\textcolor[rgb]{0.56,0.35,0.01}{#1}}
\newcommand{\PreprocessorTok}[1]{\textcolor[rgb]{0.56,0.35,0.01}{\textit{#1}}}
\newcommand{\RegionMarkerTok}[1]{#1}
\newcommand{\SpecialCharTok}[1]{\textcolor[rgb]{0.00,0.00,0.00}{#1}}
\newcommand{\SpecialStringTok}[1]{\textcolor[rgb]{0.31,0.60,0.02}{#1}}
\newcommand{\StringTok}[1]{\textcolor[rgb]{0.31,0.60,0.02}{#1}}
\newcommand{\VariableTok}[1]{\textcolor[rgb]{0.00,0.00,0.00}{#1}}
\newcommand{\VerbatimStringTok}[1]{\textcolor[rgb]{0.31,0.60,0.02}{#1}}
\newcommand{\WarningTok}[1]{\textcolor[rgb]{0.56,0.35,0.01}{\textbf{\textit{#1}}}}
\usepackage{graphicx,grffile}
\makeatletter
\def\maxwidth{\ifdim\Gin@nat@width>\linewidth\linewidth\else\Gin@nat@width\fi}
\def\maxheight{\ifdim\Gin@nat@height>\textheight\textheight\else\Gin@nat@height\fi}
\makeatother
% Scale images if necessary, so that they will not overflow the page
% margins by default, and it is still possible to overwrite the defaults
% using explicit options in \includegraphics[width, height, ...]{}
\setkeys{Gin}{width=\maxwidth,height=\maxheight,keepaspectratio}
% Set default figure placement to htbp
\makeatletter
\def\fps@figure{htbp}
\makeatother
\setlength{\emergencystretch}{3em} % prevent overfull lines
\providecommand{\tightlist}{%
  \setlength{\itemsep}{0pt}\setlength{\parskip}{0pt}}
\setcounter{secnumdepth}{-\maxdimen} % remove section numbering

\title{FLO replication - Preprocessing + analysis + results summary}
\author{Eva}
\date{4/3/2020}

\begin{document}
\maketitle

{
\setcounter{tocdepth}{3}
\tableofcontents
}
\hypertarget{clean-ws-set-wd}{%
\section{clean WS, set WD}\label{clean-ws-set-wd}}

\begin{Shaded}
\begin{Highlighting}[]
\KeywordTok{rm}\NormalTok{(}\DataTypeTok{list =} \KeywordTok{ls}\NormalTok{());}
\end{Highlighting}
\end{Shaded}

Set your local working directory. This should be (and is assumed to be
in the rest of the code) the highest point in your local folder:

\begin{Shaded}
\begin{Highlighting}[]
\NormalTok{localGitDir <-}\StringTok{ 'C:/Users/eva_v/Documents/GitHub/leverhulmeNDL'}
\CommentTok{#setwd(localGitDir);}
\end{Highlighting}
\end{Shaded}

\begin{Shaded}
\begin{Highlighting}[]
\NormalTok{fribbleSet <-}\StringTok{ }\KeywordTok{read.csv}\NormalTok{(}\KeywordTok{paste}\NormalTok{(localGitDir, }\StringTok{"/exp1/stimuli/stimuli.csv"}\NormalTok{, }\DataTypeTok{sep =} \StringTok{""}\NormalTok{), }
                       \DataTypeTok{header =}\NormalTok{ T,}
                       \DataTypeTok{colClasses=}\KeywordTok{c}\NormalTok{(}\StringTok{"cueID"}\NormalTok{=}\StringTok{"factor"}\NormalTok{,}
                        \StringTok{"bodyShape"}\NormalTok{=}\StringTok{"factor"}\NormalTok{,}
                        \StringTok{"label"}\NormalTok{=}\StringTok{"factor"}\NormalTok{,}
                        \StringTok{"fribbleID"}\NormalTok{=}\StringTok{"factor"}
\NormalTok{                        ));}
\end{Highlighting}
\end{Shaded}

\hypertarget{load-functions-from-the-lab-repo}{%
\section{Load functions from the lab
repo}\label{load-functions-from-the-lab-repo}}

\begin{Shaded}
\begin{Highlighting}[]
\NormalTok{urlFolder <-}\StringTok{ 'https://api.github.com/repos/n400peanuts/languagelearninglab/git/trees/master?recursive=1'}
\NormalTok{urlRaw <-}\StringTok{ 'https://raw.githubusercontent.com/n400peanuts/languagelearninglab/master/tools/'}

\NormalTok{loadFunctionsGithub <-}\ControlFlowTok{function}\NormalTok{(urlFolder, urlRaw)\{}
  \ControlFlowTok{if}\NormalTok{ (}\OperatorTok{!}\KeywordTok{require}\NormalTok{(httr)) \{}
    \KeywordTok{stop}\NormalTok{(}\StringTok{"httr not installed"}\NormalTok{)}
\NormalTok{  \} }
  \ControlFlowTok{else} \ControlFlowTok{if}\NormalTok{ (}\OperatorTok{!}\KeywordTok{require}\NormalTok{(RCurl))\{}
    \KeywordTok{stop}\NormalTok{(}\StringTok{"RCurl not installed"}\NormalTok{) }
\NormalTok{  \}}
  \ControlFlowTok{else}\NormalTok{ \{}
    \KeywordTok{print}\NormalTok{(}\StringTok{'----loading. Please wait----'}\NormalTok{)}
\NormalTok{  \};}
\NormalTok{  httr}\OperatorTok{::}\KeywordTok{GET}\NormalTok{(urlFolder)->}\StringTok{ }\NormalTok{req}
  \KeywordTok{stop_for_status}\NormalTok{(req)}
\NormalTok{  filelist <-}\StringTok{ }\KeywordTok{unlist}\NormalTok{(}\KeywordTok{lapply}\NormalTok{(}\KeywordTok{content}\NormalTok{(req)}\OperatorTok{$}\NormalTok{tree, }\StringTok{"["}\NormalTok{, }\StringTok{"path"}\NormalTok{), }\DataTypeTok{use.names =}\NormalTok{ F)}
\NormalTok{  urlFunctions <-}\StringTok{ }\KeywordTok{grep}\NormalTok{(}\StringTok{"docs/tools/"}\NormalTok{, filelist, }\DataTypeTok{value =} \OtherTok{TRUE}\NormalTok{, }\DataTypeTok{fixed =} \OtherTok{TRUE}\NormalTok{)}
  \KeywordTok{gsub}\NormalTok{(}\StringTok{"docs/tools/"}\NormalTok{, }\StringTok{""}\NormalTok{, urlFunctions) ->}\StringTok{ }\NormalTok{functions}
  \ControlFlowTok{for}\NormalTok{ (i }\ControlFlowTok{in} \DecValTok{1}\OperatorTok{:}\KeywordTok{length}\NormalTok{(functions))\{}
\NormalTok{    RCurl}\OperatorTok{::}\KeywordTok{getURL}\NormalTok{(}\KeywordTok{paste0}\NormalTok{(urlRaw, functions[i]), }\DataTypeTok{ssl.verifypeer =} \OtherTok{FALSE}\NormalTok{)->}\StringTok{ }\NormalTok{temp}
    \KeywordTok{eval}\NormalTok{(}\KeywordTok{parse}\NormalTok{(}\DataTypeTok{text =}\NormalTok{ temp), }\DataTypeTok{envir =}\NormalTok{ .GlobalEnv)}
\NormalTok{  \};}
\NormalTok{\}}

\KeywordTok{loadFunctionsGithub}\NormalTok{(}\DataTypeTok{urlFolder =}\NormalTok{ urlFolder, }\DataTypeTok{urlRaw =}\NormalTok{ urlRaw);}
\end{Highlighting}
\end{Shaded}

\begin{verbatim}
## [1] "----loading. Please wait----"
\end{verbatim}

\begin{Shaded}
\begin{Highlighting}[]
\KeywordTok{rm}\NormalTok{(urlFolder, urlRaw)}
\end{Highlighting}
\end{Shaded}

\hypertarget{check-stimuli-set}{%
\section{Check stimuli set}\label{check-stimuli-set}}

It's important to check that every fribble is unique in the way its
features are assembled within each category. Feature position and
identity are coded into cueID.

I'm going to check whether the combination of cues used to build the
fribble is unique by filtering for n \textgreater{} 1:

\begin{Shaded}
\begin{Highlighting}[]
\NormalTok{fribbleSet }\OperatorTok
\StringTok{  }\KeywordTok{group_by}\NormalTok{(category, cueID) }\OperatorTok
\StringTok{  }\KeywordTok{count}\NormalTok{() }\OperatorTok
\StringTok{  }\KeywordTok{filter}\NormalTok{(n }\OperatorTok{>}\StringTok{ }\DecValTok{1}\NormalTok{);}
\end{Highlighting}
\end{Shaded}

\begin{verbatim}
## Warning: Factor `cueID` contains implicit NA, consider using
## `forcats::fct_explicit_na`
\end{verbatim}

\begin{verbatim}
## # A tibble: 0 x 3
## # Groups:   category, cueID [1]
## # ... with 3 variables: category <int>, cueID <fct>, n <int>
\end{verbatim}

Great, each Fribble is unique!

\hypertarget{load-data}{%
\section{Load data}\label{load-data}}

List the files present in the folder, and load them.

\begin{Shaded}
\begin{Highlighting}[]
\NormalTok{df <-}\StringTok{ }\KeywordTok{list.files}\NormalTok{(}\KeywordTok{paste}\NormalTok{(localGitDir, }\StringTok{"/exp1/data/"}\NormalTok{, }\DataTypeTok{sep =} \StringTok{""}\NormalTok{)); }
\end{Highlighting}
\end{Shaded}

We have 6 files.

\begin{Shaded}
\begin{Highlighting}[]
\ControlFlowTok{for}\NormalTok{ (i }\ControlFlowTok{in} \DecValTok{1}\OperatorTok{:}\KeywordTok{length}\NormalTok{(df))\{}
  \KeywordTok{gsub}\NormalTok{(}\StringTok{".csv$"}\NormalTok{, }\StringTok{""}\NormalTok{, df[i]) ->}\StringTok{ }\NormalTok{id}
  \KeywordTok{assign}\NormalTok{(id, }\KeywordTok{data.frame}\NormalTok{())}
  \KeywordTok{read.csv}\NormalTok{(}\KeywordTok{paste}\NormalTok{(localGitDir, }\StringTok{"/exp1/data/"}\NormalTok{, df[i], }\DataTypeTok{sep =} \StringTok{""}\NormalTok{),}
           \DataTypeTok{na.strings=}\KeywordTok{c}\NormalTok{(}\StringTok{""}\NormalTok{,}\StringTok{"NA"}\NormalTok{),}
           \DataTypeTok{colClasses=}\KeywordTok{c}\NormalTok{(}\StringTok{"presentedLabel"}\NormalTok{=}\StringTok{"factor"}\NormalTok{,}
                        \StringTok{"presentedImage"}\NormalTok{=}\StringTok{"factor"}\NormalTok{,}
                        \StringTok{"learningType"}\NormalTok{=}\StringTok{"factor"}\NormalTok{,}
                        \StringTok{"Trial.Type"}\NormalTok{=}\StringTok{"factor"}\NormalTok{,}
                        \StringTok{"Test.Part"}\NormalTok{=}\StringTok{"factor"}\NormalTok{,}
                        \StringTok{"Key.Press"}\NormalTok{=}\StringTok{"factor"}
\NormalTok{                        ))->}\StringTok{ }\NormalTok{temp}
  \KeywordTok{assign}\NormalTok{(}\KeywordTok{paste0}\NormalTok{(id), temp)}
\NormalTok{\};}

\KeywordTok{rm}\NormalTok{(temp, df, i, id);}
\end{Highlighting}
\end{Shaded}

The dataset name is decided autonomously by Gorilla. Importantly,
Gorilla produces a different file per condition, and codes the
conditions by the last 4 letters.

\begin{itemize}
\item
  2yjh is the FL learning
\item
  q8hp is the LF learning
\end{itemize}

I'm going to rename them for clarity.

\begin{Shaded}
\begin{Highlighting}[]
\NormalTok{dataFL<-}\StringTok{`}\DataTypeTok{data_exp_15519-v13_task-2yjh}\StringTok{`}
\NormalTok{dataFL}\OperatorTok{$}\NormalTok{Experiment.Version <-}\StringTok{ }\KeywordTok{c}\NormalTok{(}\DecValTok{14}\NormalTok{)}
\NormalTok{dataFL2<-}\StringTok{`}\DataTypeTok{data_exp_15519-v14_task-2yjh}\StringTok{`}
\NormalTok{dataFL3<-}\StringTok{`}\DataTypeTok{data_exp_15519-v15_task-2yjh}\StringTok{`}

\KeywordTok{rm}\NormalTok{(}\StringTok{`}\DataTypeTok{data_exp_15519-v13_task-2yjh}\StringTok{`}\NormalTok{)}
\KeywordTok{rm}\NormalTok{(}\StringTok{`}\DataTypeTok{data_exp_15519-v14_task-2yjh}\StringTok{`}\NormalTok{)}
\KeywordTok{rm}\NormalTok{(}\StringTok{`}\DataTypeTok{data_exp_15519-v15_task-2yjh}\StringTok{`}\NormalTok{)}

\NormalTok{dataLF <-}\StringTok{ `}\DataTypeTok{data_exp_15519-v13_task-q8hp}\StringTok{`}
\NormalTok{dataLF}\OperatorTok{$}\NormalTok{Experiment.Version <-}\StringTok{ }\KeywordTok{c}\NormalTok{(}\DecValTok{14}\NormalTok{)}
\NormalTok{dataLF2 <-}\StringTok{ `}\DataTypeTok{data_exp_15519-v14_task-q8hp}\StringTok{`}
\NormalTok{dataLF3 <-}\StringTok{ `}\DataTypeTok{data_exp_15519-v15_task-q8hp}\StringTok{`}

\KeywordTok{rm}\NormalTok{(}\StringTok{`}\DataTypeTok{data_exp_15519-v13_task-q8hp}\StringTok{`}\NormalTok{)}
\KeywordTok{rm}\NormalTok{(}\StringTok{`}\DataTypeTok{data_exp_15519-v14_task-q8hp}\StringTok{`}\NormalTok{)}
\KeywordTok{rm}\NormalTok{(}\StringTok{`}\DataTypeTok{data_exp_15519-v15_task-q8hp}\StringTok{`}\NormalTok{)}
\end{Highlighting}
\end{Shaded}

\begin{Shaded}
\begin{Highlighting}[]
\KeywordTok{rbind}\NormalTok{(dataFL, dataFL2, dataFL3)->}\StringTok{ }\NormalTok{dataFL}
\KeywordTok{rbind}\NormalTok{(dataLF, dataLF2, dataLF3)->}\StringTok{ }\NormalTok{dataLF}

\KeywordTok{rm}\NormalTok{(dataFL2, dataFL3, dataLF2, dataLF3)}
\end{Highlighting}
\end{Shaded}

Gorilla's output is extremely messy. Each row is a screen event.
However, we want only the events related to 1. the presentations of the
fribbles and the labels 2. participants' response and 3. what type of
tasks.

I have coded these info in some columns and rows that I'm going to
select:

\begin{Shaded}
\begin{Highlighting}[]
\NormalTok{raw_dataFL<-}\StringTok{ }\NormalTok{dataFL[}\KeywordTok{c}\NormalTok{(}\StringTok{'Participant.Private.ID'}\NormalTok{, }\StringTok{'learningType'}\NormalTok{, }\StringTok{'Test.Part'}\NormalTok{ , }
         \StringTok{'presentedImage'}\NormalTok{, }\StringTok{'presentedLabel'}\NormalTok{, }\StringTok{'Reaction.Time'}\NormalTok{, }\StringTok{"Key.Press"}\NormalTok{,}
          \StringTok{'Trial.Type'}\NormalTok{, }\StringTok{'Trial.Index'}\NormalTok{, }\StringTok{'Correct'}\NormalTok{, }\StringTok{'Experiment.Version'}\NormalTok{)]}

\NormalTok{raw_dataLF<-}\StringTok{ }\NormalTok{dataLF[}\KeywordTok{c}\NormalTok{(}\StringTok{'Participant.Private.ID'}\NormalTok{, }\StringTok{'learningType'}\NormalTok{, }\StringTok{'Test.Part'}\NormalTok{ , }
         \StringTok{'presentedImage'}\NormalTok{, }\StringTok{'presentedLabel'}\NormalTok{, }\StringTok{'Reaction.Time'}\NormalTok{, }\StringTok{"Key.Press"}\NormalTok{,}
          \StringTok{'Trial.Type'}\NormalTok{, }\StringTok{'Trial.Index'}\NormalTok{, }\StringTok{'Correct'}\NormalTok{, }\StringTok{'Experiment.Version'}\NormalTok{)]}
\end{Highlighting}
\end{Shaded}

Select rows:

\begin{Shaded}
\begin{Highlighting}[]
\NormalTok{rowsIwantTokeep <-}\StringTok{ }\KeywordTok{c}\NormalTok{(}\StringTok{"learningBlock1"}\NormalTok{, }\StringTok{"learningBlock2"}\NormalTok{, }\StringTok{"learningBlock3"}\NormalTok{,}
                        \StringTok{"learningBlock4"}\NormalTok{, }\StringTok{"generalizationPL"}\NormalTok{, }\StringTok{"generalizationLP"}\NormalTok{,}
                        \StringTok{"randomDot"}\NormalTok{, }\StringTok{"contingencyJudgement"}\NormalTok{)}

\NormalTok{raw_dataFL <-}\StringTok{ }\NormalTok{raw_dataFL }\OperatorTok\StringTok{ }
\StringTok{  }\KeywordTok{filter}\NormalTok{(Test.Part }\OperatorTok\StringTok{ }\NormalTok{rowsIwantTokeep ) }\OperatorTok
\StringTok{  }\KeywordTok{rename}\NormalTok{(}\DataTypeTok{subjID =}\NormalTok{ Participant.Private.ID, }
         \DataTypeTok{learning =}\NormalTok{ learningType,}
         \DataTypeTok{task =}\NormalTok{ Test.Part, }
         \DataTypeTok{fribbleID =}\NormalTok{ presentedImage,}
         \DataTypeTok{label =}\NormalTok{ presentedLabel, }
         \DataTypeTok{rt =}\NormalTok{ Reaction.Time, }
         \DataTypeTok{resp =}\NormalTok{ Key.Press, }
         \DataTypeTok{trialType =}\NormalTok{ Trial.Type,}
         \DataTypeTok{trialIndex =}\NormalTok{ Trial.Index,}
         \DataTypeTok{acc =}\NormalTok{ Correct)}

\NormalTok{raw_dataLF <-}\StringTok{ }\NormalTok{raw_dataLF }\OperatorTok\StringTok{ }
\StringTok{  }\KeywordTok{filter}\NormalTok{(Test.Part }\OperatorTok\StringTok{ }\NormalTok{rowsIwantTokeep ) }\OperatorTok
\StringTok{  }\KeywordTok{rename}\NormalTok{(}\DataTypeTok{subjID =}\NormalTok{ Participant.Private.ID, }
         \DataTypeTok{learning =}\NormalTok{ learningType,}
         \DataTypeTok{task =}\NormalTok{ Test.Part, }
         \DataTypeTok{fribbleID =}\NormalTok{ presentedImage,}
         \DataTypeTok{label =}\NormalTok{ presentedLabel, }
         \DataTypeTok{rt =}\NormalTok{ Reaction.Time, }
         \DataTypeTok{resp =}\NormalTok{ Key.Press, }
         \DataTypeTok{trialType =}\NormalTok{ Trial.Type,}
         \DataTypeTok{trialIndex =}\NormalTok{ Trial.Index,}
         \DataTypeTok{acc =}\NormalTok{ Correct)}

\KeywordTok{rm}\NormalTok{(rowsIwantTokeep, dataFL, dataLF);}
\end{Highlighting}
\end{Shaded}

I'm going to merge both datasets, FL and LF, because we have anyway a
column ``learning'' that can tell us which one is which.

\begin{Shaded}
\begin{Highlighting}[]
\KeywordTok{rbind}\NormalTok{(raw_dataFL, raw_dataLF)->}\StringTok{ }\NormalTok{raw_data; }
\KeywordTok{rm}\NormalTok{(raw_dataFL, raw_dataLF);}
\end{Highlighting}
\end{Shaded}

\hypertarget{check-learning}{%
\section{Check learning}\label{check-learning}}

Let's filter and check learning trials:

\begin{Shaded}
\begin{Highlighting}[]
\NormalTok{learningBlocks <-}\StringTok{ }\KeywordTok{c}\NormalTok{(}\StringTok{"learningBlock1"}\NormalTok{, }\StringTok{"learningBlock2"}\NormalTok{, }\StringTok{"learningBlock3"}\NormalTok{, }\StringTok{"learningBlock4"}\NormalTok{);}

\NormalTok{learning <-}\StringTok{ }\NormalTok{raw_data }\OperatorTok\StringTok{ }
\StringTok{  }\KeywordTok{filter}\NormalTok{(task }\OperatorTok\StringTok{ }\NormalTok{learningBlocks) }

\NormalTok{learning <-}\StringTok{ }\KeywordTok{droplevels}\NormalTok{(learning);}
\KeywordTok{rm}\NormalTok{(learningBlocks)}
\end{Highlighting}
\end{Shaded}

\hypertarget{how-many-trials-per-participant}{%
\subsection{How many trials per
participant?}\label{how-many-trials-per-participant}}

\begin{Shaded}
\begin{Highlighting}[]
\NormalTok{learning }\OperatorTok\StringTok{                             }
\StringTok{  }\KeywordTok{group_by}\NormalTok{(subjID, learning) }\OperatorTok\StringTok{    }
\StringTok{  }\KeywordTok{count}\NormalTok{() }
\end{Highlighting}
\end{Shaded}

\begin{verbatim}
## # A tibble: 120 x 3
## # Groups:   subjID, learning [120]
##     subjID learning     n
##      <int> <fct>    <int>
##  1 1414932 LF         120
##  2 1414933 LF         120
##  3 1414937 FL         120
##  4 1414945 FL         120
##  5 1414957 FL         120
##  6 1415040 FL         120
##  7 1420163 FL         120
##  8 1420165 FL         120
##  9 1420169 LF         120
## 10 1420171 LF         120
## # ... with 110 more rows
\end{verbatim}

Great, 120 trials per participant, per learning.

Let's check whether the blocks' length varied across participants:

\begin{Shaded}
\begin{Highlighting}[]
\NormalTok{learning }\OperatorTok\StringTok{                             }
\StringTok{  }\KeywordTok{group_by}\NormalTok{(subjID, task) }\OperatorTok
\StringTok{  }\KeywordTok{count}\NormalTok{()}
\end{Highlighting}
\end{Shaded}

\begin{verbatim}
## # A tibble: 480 x 3
## # Groups:   subjID, task [480]
##     subjID task               n
##      <int> <fct>          <int>
##  1 1414932 learningBlock1    21
##  2 1414932 learningBlock2    28
##  3 1414932 learningBlock3    47
##  4 1414932 learningBlock4    24
##  5 1414933 learningBlock1    26
##  6 1414933 learningBlock2    22
##  7 1414933 learningBlock3    44
##  8 1414933 learningBlock4    28
##  9 1414937 learningBlock1    27
## 10 1414937 learningBlock2    47
## # ... with 470 more rows
\end{verbatim}

Great! Each participant had a different amount of trials distributed
across blocks. That's important because our random dot task was
presented at the end of each block, and we wanted its presentation to be
unpredictable. Anyway, the sum of all the learning trials was always
120.

\hypertarget{did-we-assign-our-learning-randomly-every-couple-of-people}{%
\subsection{Did we assign our learning randomly every couple of
people?}\label{did-we-assign-our-learning-randomly-every-couple-of-people}}

\begin{Shaded}
\begin{Highlighting}[]
\KeywordTok{table}\NormalTok{(learning}\OperatorTok{$}\NormalTok{subjID, learning}\OperatorTok{$}\NormalTok{learning)}
\end{Highlighting}
\end{Shaded}

\begin{verbatim}
##          
##            FL  LF
##   1414932   0 120
##   1414933   0 120
##   1414937 120   0
##   1414945 120   0
##   1414957 120   0
##   1415040 120   0
##   1420163 120   0
##   1420165 120   0
##   1420169   0 120
##   1420171   0 120
##   1420177 120   0
##   1420180 120   0
##   1420185   0 120
##   1420199 120   0
##   1420204   0 120
##   1420552   0 120
##   1420573   0 120
##   1420577   0 120
##   1420580 120   0
##   1420622 120   0
##   1422463 120   0
##   1422465 120   0
##   1422466 120   0
##   1422467   0 120
##   1422470   0 120
##   1422472 120   0
##   1422473   0 120
##   1422475   0 120
##   1422476   0 120
##   1422477 120   0
##   1422675 120   0
##   1422676   0 120
##   1422677 120   0
##   1422678   0 120
##   1422679 120   0
##   1422680   0 120
##   1422681   0 120
##   1422689 120   0
##   1422715   0 120
##   1422716 120   0
##   1431942   0 120
##   1431944 120   0
##   1431946 120   0
##   1431948   0 120
##   1431949 120   0
##   1431952   0 120
##   1431953 120   0
##   1431954   0 120
##   1431956   0 120
##   1431957 120   0
##   1431958 120   0
##   1431959   0 120
##   1431960   0 120
##   1431961 120   0
##   1431963   0 120
##   1431965 120   0
##   1431966   0 120
##   1431968   0 120
##   1431969 120   0
##   1431970   0 120
##   1431972 120   0
##   1431974 120   0
##   1431978 120   0
##   1431979 120   0
##   1431981   0 120
##   1431984 120   0
##   1431989   0 120
##   1431992 120   0
##   1431997 120   0
##   1431998   0 120
##   1431999   0 120
##   1432003   0 120
##   1432007   0 120
##   1432009 120   0
##   1432011 120   0
##   1432030   0 120
##   1432052 120   0
##   1432075 120   0
##   1432301   0 120
##   1432323   0 120
##   1457883   0 120
##   1458992 120   0
##   1458996   0 120
##   1458997   0 120
##   1458998   0 120
##   1459001 120   0
##   1459002 120   0
##   1459003 120   0
##   1459007 120   0
##   1459009 120   0
##   1459013 120   0
##   1459015   0 120
##   1459018 120   0
##   1459020   0 120
##   1459024 120   0
##   1459025   0 120
##   1459029 120   0
##   1459036   0 120
##   1459039   0 120
##   1459043   0 120
##   1459046   0 120
##   1459047 120   0
##   1459048 120   0
##   1459052 120   0
##   1459057 120   0
##   1459064 120   0
##   1459067   0 120
##   1459078   0 120
##   1459109   0 120
##   1459696 116   0
##   1459697 120   0
##   1459699   0 120
##   1459700   0 120
##   1459701   0 120
##   1459702 120   0
##   1459703 120   0
##   1459706 120   0
##   1459708 120   0
##   1459709   0 120
##   1459767 120   0
\end{verbatim}

Kind of. After chicking with Gorilla's suppoert: apparently, if a
participant access Gorilla, but it's not allowed to start the experiment
(e.g., the browser is not suitable), or leaves the session, this counts
anyway for the randomisation.

The rows related to the presentation of fribbles and labels, inherit
Gorilla's http address of where they are stored. Nothing I can do to
change this in Gorilla, but we can clean the rows by those info like
this:

\begin{Shaded}
\begin{Highlighting}[]
\KeywordTok{as.factor}\NormalTok{(}\KeywordTok{gsub}\NormalTok{(}\StringTok{"/task/70033/56/asset/|/task/70033/57/asset/|/task/70033/58/asset/"}\NormalTok{, }\StringTok{""}\NormalTok{, learning}\OperatorTok{$}\NormalTok{fribbleID))->}\StringTok{ }\NormalTok{learning}\OperatorTok{$}\NormalTok{fribbleID}
\KeywordTok{as.factor}\NormalTok{(}\KeywordTok{gsub}\NormalTok{(}\StringTok{".jpg$"}\NormalTok{, }\StringTok{""}\NormalTok{, learning}\OperatorTok{$}\NormalTok{fribbleID))->}\StringTok{ }\NormalTok{learning}\OperatorTok{$}\NormalTok{fribbleID}

\KeywordTok{as.factor}\NormalTok{(}\KeywordTok{gsub}\NormalTok{(}\StringTok{"/task/70033/56/asset/|/task/70033/57/asset/|/task/70033/58/asset/"}\NormalTok{, }\StringTok{""}\NormalTok{, learning}\OperatorTok{$}\NormalTok{label))->}\StringTok{ }\NormalTok{learning}\OperatorTok{$}\NormalTok{label}
\KeywordTok{as.factor}\NormalTok{(}\KeywordTok{gsub}\NormalTok{(}\StringTok{".mp3$"}\NormalTok{, }\StringTok{""}\NormalTok{, learning}\OperatorTok{$}\NormalTok{label))->}\StringTok{ }\NormalTok{learning}\OperatorTok{$}\NormalTok{label}
\NormalTok{learning}\OperatorTok{$}\NormalTok{resp <-}\StringTok{ }\KeywordTok{as.factor}\NormalTok{(}\StringTok{'NA'}\NormalTok{)}
\end{Highlighting}
\end{Shaded}

This is how the learning dataframe looks like now:

\begin{Shaded}
\begin{Highlighting}[]
\KeywordTok{head}\NormalTok{(learning);}
\end{Highlighting}
\end{Shaded}

\begin{verbatim}
##    subjID learning           task fribbleID label rt resp
## 1 1414937       FL learningBlock1     20375 FLbim NA   NA
## 2 1414937       FL learningBlock1     31075 FLtob NA   NA
## 3 1414937       FL learningBlock1     32775 FLtob NA   NA
## 4 1414937       FL learningBlock1     32875 FLtob NA   NA
## 5 1414937       FL learningBlock1     22025 FLbim NA   NA
## 6 1414937       FL learningBlock1     10425 FLdep NA   NA
##                 trialType trialIndex acc Experiment.Version
## 1 audio-keyboard-response         22  NA                 14
## 2 audio-keyboard-response         25  NA                 14
## 3 audio-keyboard-response         28  NA                 14
## 4 audio-keyboard-response         31  NA                 14
## 5 audio-keyboard-response         34  NA                 14
## 6 audio-keyboard-response         37  NA                 14
\end{verbatim}

\begin{Shaded}
\begin{Highlighting}[]
\KeywordTok{summary}\NormalTok{(learning);}
\end{Highlighting}
\end{Shaded}

\begin{verbatim}
##      subjID        learning              task        fribbleID       label     
##  Min.   :1414932   FL:7556   learningBlock1:3529   10475  :  124   FLbim:2519  
##  1st Qu.:1422477   LF:6840   learningBlock2:3773   31675  :  124   FLdep:2517  
##  Median :1431970             learningBlock3:3595   13375  :  121   FLtob:2520  
##  Mean   :1437270             learningBlock4:3499   22775  :  120   LFbim:2280  
##  3rd Qu.:1459009                                   30375  :  120   LFdep:2280  
##  Max.   :1459767                                   32475  :  120   LFtob:2280  
##                                                    (Other):13667               
##        rt         resp                         trialType      trialIndex   
##  Min.   : 12.36   NA:14396   audio-keyboard-response:7556   Min.   : 22.0  
##  1st Qu.: 52.50              image-keyboard-response:6840   1st Qu.:115.0  
##  Median : 88.00                                             Median :211.0  
##  Mean   :126.25                                             Mean   :210.8  
##  3rd Qu.:214.71                                             3rd Qu.:307.0  
##  Max.   :249.00                                             Max.   :400.0  
##  NA's   :14389                                                             
##       acc        Experiment.Version
##  Min.   : NA     Min.   :14.00     
##  1st Qu.: NA     1st Qu.:14.00     
##  Median : NA     Median :14.00     
##  Mean   :NaN     Mean   :14.33     
##  3rd Qu.: NA     3rd Qu.:15.00     
##  Max.   : NA     Max.   :15.00     
##  NA's   :14396
\end{verbatim}

Our fribbles were presented two times during learning.

\hypertarget{check-if-fribbles-are-presented-2-times}{%
\subsection{Check if fribbles are presented \textgreater{} 2
times:}\label{check-if-fribbles-are-presented-2-times}}

\begin{Shaded}
\begin{Highlighting}[]
\NormalTok{learning }\OperatorTok\StringTok{                             }
\StringTok{  }\KeywordTok{group_by}\NormalTok{(subjID, fribbleID) }\OperatorTok\StringTok{    }
\StringTok{  }\KeywordTok{count}\NormalTok{() }\OperatorTok
\StringTok{  }\KeywordTok{filter}\NormalTok{(n }\OperatorTok{>}\DecValTok{2}\NormalTok{)}
\end{Highlighting}
\end{Shaded}

\begin{verbatim}
## Warning: Factor `fribbleID` contains implicit NA, consider using
## `forcats::fct_explicit_na`
\end{verbatim}

\begin{verbatim}
## # A tibble: 0 x 3
## # Groups:   subjID, fribbleID [1]
## # ... with 3 variables: subjID <int>, fribbleID <fct>, n <int>
\end{verbatim}

None, perfect.

\hypertarget{check-whether-there-are-fribbles-presented-only-once}{%
\subsection{Check whether there are fribbles presented only
once:}\label{check-whether-there-are-fribbles-presented-only-once}}

\begin{Shaded}
\begin{Highlighting}[]
\NormalTok{learning }\OperatorTok\StringTok{                             }
\StringTok{  }\KeywordTok{group_by}\NormalTok{(subjID, fribbleID) }\OperatorTok\StringTok{    }
\StringTok{  }\KeywordTok{count}\NormalTok{() }\OperatorTok
\StringTok{  }\KeywordTok{filter}\NormalTok{(n }\OperatorTok{<}\StringTok{ }\DecValTok{2}\NormalTok{)}
\end{Highlighting}
\end{Shaded}

\begin{verbatim}
## # A tibble: 4 x 3
## # Groups:   subjID, fribbleID [4]
##    subjID fribbleID     n
##     <int> <fct>     <int>
## 1 1459696 12075         1
## 2 1459696 12675         1
## 3 1459696 13375         1
## 4 1459696 22125         1
\end{verbatim}

Perfect.

\hypertarget{check-the-association-between-the-fribbles-and-the-labels-high-and-low-frequency-with-the-correct-labels}{%
\subsection{Check the association between the fribbles and the labels
(high and low frequency with the correct
labels)}\label{check-the-association-between-the-fribbles-and-the-labels-high-and-low-frequency-with-the-correct-labels}}

Fribbles ID are coded in this way: e.g., 10175-\textgreater{} {[}1{]} is
the category {[}01{]} is the number of the fribble {[}75{]} is the
frequency.

In the column fribbleID we have the fribble presented, in the column
label we have the sound played.

Association between fribbles and labels are fixed:

\begin{itemize}
\item
  category 1, regardless of the frequency, has the label: dep
\item
  category 2, regardless of the frequency, has the label: bim
\item
  category 3, regardless of the frequency, has the label: tob
\end{itemize}

I'm going to add a column for category, fribble number, and frequency,
in order to check whether everything is okay:

We should have only 3 categories, presented twice per participant. Each
category is made of 20 exemplars.

\begin{Shaded}
\begin{Highlighting}[]
\NormalTok{learning}\OperatorTok{$}\NormalTok{category <-}\StringTok{ }\DecValTok{0}
\NormalTok{learning[}\KeywordTok{substr}\NormalTok{(}\KeywordTok{as.character}\NormalTok{(learning}\OperatorTok{$}\NormalTok{fribbleID), }\DecValTok{1}\NormalTok{, }\DecValTok{1}\NormalTok{)}\OperatorTok{==}\DecValTok{1}\NormalTok{,]}\OperatorTok{$}\NormalTok{category <-}\StringTok{ }\DecValTok{1}
\NormalTok{learning[}\KeywordTok{substr}\NormalTok{(}\KeywordTok{as.character}\NormalTok{(learning}\OperatorTok{$}\NormalTok{fribbleID), }\DecValTok{1}\NormalTok{, }\DecValTok{1}\NormalTok{)}\OperatorTok{==}\DecValTok{2}\NormalTok{,]}\OperatorTok{$}\NormalTok{category <-}\StringTok{ }\DecValTok{2}
\NormalTok{learning[}\KeywordTok{substr}\NormalTok{(}\KeywordTok{as.character}\NormalTok{(learning}\OperatorTok{$}\NormalTok{fribbleID), }\DecValTok{1}\NormalTok{, }\DecValTok{1}\NormalTok{)}\OperatorTok{==}\DecValTok{3}\NormalTok{,]}\OperatorTok{$}\NormalTok{category <-}\StringTok{ }\DecValTok{3}

\NormalTok{(}\KeywordTok{nrow}\NormalTok{(learning[learning}\OperatorTok{$}\NormalTok{category}\OperatorTok{==}\DecValTok{1}\NormalTok{,]) }\OperatorTok{/}\StringTok{ }\KeywordTok{length}\NormalTok{(}\KeywordTok{unique}\NormalTok{(learning}\OperatorTok{$}\NormalTok{subjID))) }\OperatorTok{/}\StringTok{ }\DecValTok{2}
\end{Highlighting}
\end{Shaded}

\begin{verbatim}
## [1] 19.9875
\end{verbatim}

\begin{Shaded}
\begin{Highlighting}[]
\NormalTok{(}\KeywordTok{nrow}\NormalTok{(learning[learning}\OperatorTok{$}\NormalTok{category}\OperatorTok{==}\DecValTok{2}\NormalTok{,]) }\OperatorTok{/}\StringTok{ }\KeywordTok{length}\NormalTok{(}\KeywordTok{unique}\NormalTok{(learning}\OperatorTok{$}\NormalTok{subjID))) }\OperatorTok{/}\StringTok{ }\DecValTok{2}
\end{Highlighting}
\end{Shaded}

\begin{verbatim}
## [1] 19.99583
\end{verbatim}

\begin{Shaded}
\begin{Highlighting}[]
\NormalTok{(}\KeywordTok{nrow}\NormalTok{(learning[learning}\OperatorTok{$}\NormalTok{category}\OperatorTok{==}\DecValTok{3}\NormalTok{,]) }\OperatorTok{/}\StringTok{ }\KeywordTok{length}\NormalTok{(}\KeywordTok{unique}\NormalTok{(learning}\OperatorTok{$}\NormalTok{subjID))) }\OperatorTok{/}\StringTok{ }\DecValTok{2}
\end{Highlighting}
\end{Shaded}

\begin{verbatim}
## [1] 20
\end{verbatim}

We have 15 high frequency and 5 low frequency exemplars x category:

\begin{Shaded}
\begin{Highlighting}[]
\NormalTok{learning}\OperatorTok{$}\NormalTok{frequency <-}\StringTok{ }\DecValTok{25}
\NormalTok{learning[}\KeywordTok{substr}\NormalTok{(}\KeywordTok{as.character}\NormalTok{(learning}\OperatorTok{$}\NormalTok{fribbleID), }\DecValTok{4}\NormalTok{, }\DecValTok{5}\NormalTok{)}\OperatorTok{==}\DecValTok{75}\NormalTok{,]}\OperatorTok{$}\NormalTok{frequency <-}\StringTok{ }\DecValTok{75}

\NormalTok{(}\KeywordTok{nrow}\NormalTok{(learning[learning}\OperatorTok{$}\NormalTok{frequency}\OperatorTok{==}\DecValTok{25}\NormalTok{,]) }\OperatorTok{/}\StringTok{ }\KeywordTok{length}\NormalTok{(}\KeywordTok{unique}\NormalTok{(learning}\OperatorTok{$}\NormalTok{subjID))) }\OperatorTok{/}\StringTok{ }\DecValTok{2}
\end{Highlighting}
\end{Shaded}

\begin{verbatim}
## [1] 14.99583
\end{verbatim}

\begin{Shaded}
\begin{Highlighting}[]
\NormalTok{(}\KeywordTok{nrow}\NormalTok{(learning[learning}\OperatorTok{$}\NormalTok{frequency}\OperatorTok{==}\DecValTok{75}\NormalTok{,]) }\OperatorTok{/}\StringTok{ }\KeywordTok{length}\NormalTok{(}\KeywordTok{unique}\NormalTok{(learning}\OperatorTok{$}\NormalTok{subjID))) }\OperatorTok{/}\StringTok{ }\DecValTok{2}
\end{Highlighting}
\end{Shaded}

\begin{verbatim}
## [1] 44.9875
\end{verbatim}

Now let's check the fribble-label association:

\begin{Shaded}
\begin{Highlighting}[]
\KeywordTok{table}\NormalTok{(learning}\OperatorTok{$}\NormalTok{category, learning}\OperatorTok{$}\NormalTok{label, learning}\OperatorTok{$}\NormalTok{frequency)}
\end{Highlighting}
\end{Shaded}

\begin{verbatim}
## , ,  = 25
## 
##    
##     FLbim FLdep FLtob LFbim LFdep LFtob
##   1     0   630     0     0   570     0
##   2   629     0     0   570     0     0
##   3     0     0   630     0     0   570
## 
## , ,  = 75
## 
##    
##     FLbim FLdep FLtob LFbim LFdep LFtob
##   1     0  1887     0     0  1710     0
##   2  1890     0     0  1710     0     0
##   3     0     0  1890     0     0  1710
\end{verbatim}

Okay, each label was associated to its correct fribble (coded here as
category).

\hypertarget{check-testing}{%
\section{Check Testing}\label{check-testing}}

I'm going to select the tests and clean the rows from Gorilla's http
address:

\begin{Shaded}
\begin{Highlighting}[]
\NormalTok{tests <-}\StringTok{ }\KeywordTok{c}\NormalTok{(}\StringTok{"generalizationPL"}\NormalTok{, }\StringTok{"generalizationLP"}\NormalTok{, }\StringTok{"contingencyJudgement"}\NormalTok{, }\StringTok{"randomDot"}\NormalTok{);}

\NormalTok{testing <-}\StringTok{ }\NormalTok{raw_data }\OperatorTok\StringTok{ }
\StringTok{  }\KeywordTok{filter}\NormalTok{(task }\OperatorTok\StringTok{ }\NormalTok{tests)  }


\NormalTok{testing <-}\StringTok{ }\KeywordTok{droplevels}\NormalTok{(testing);}
\KeywordTok{rm}\NormalTok{(tests);}

\KeywordTok{as.factor}\NormalTok{(}\KeywordTok{gsub}\NormalTok{(}\StringTok{"/task/70033/56/asset/|/task/70033/57/asset/|/task/70033/58/asset/"}\NormalTok{, }\StringTok{""}\NormalTok{, testing}\OperatorTok{$}\NormalTok{fribbleID))->}\StringTok{ }\NormalTok{testing}\OperatorTok{$}\NormalTok{fribbleID}
\KeywordTok{as.factor}\NormalTok{(}\KeywordTok{gsub}\NormalTok{(}\StringTok{".jpg$"}\NormalTok{, }\StringTok{""}\NormalTok{, testing}\OperatorTok{$}\NormalTok{fribbleID))->}\StringTok{ }\NormalTok{testing}\OperatorTok{$}\NormalTok{fribbleID}

\KeywordTok{as.factor}\NormalTok{(}\KeywordTok{gsub}\NormalTok{(}\StringTok{"/task/70033/56/asset/|/task/70033/57/asset/|/task/70033/58/asset/"}\NormalTok{, }\StringTok{""}\NormalTok{, testing}\OperatorTok{$}\NormalTok{label))->}\StringTok{ }\NormalTok{testing}\OperatorTok{$}\NormalTok{label}
\KeywordTok{as.factor}\NormalTok{(}\KeywordTok{gsub}\NormalTok{(}\StringTok{".mp3$"}\NormalTok{, }\StringTok{""}\NormalTok{, testing}\OperatorTok{$}\NormalTok{label))->}\StringTok{ }\NormalTok{testing}\OperatorTok{$}\NormalTok{label}
\end{Highlighting}
\end{Shaded}

\hypertarget{check-test-1-generalization-from-picture-to-labels}{%
\subsection{Check test 1: Generalization from picture to
labels}\label{check-test-1-generalization-from-picture-to-labels}}

We filter the rows for this task, and clean both the resp and fribble
columns.

\begin{Shaded}
\begin{Highlighting}[]
\NormalTok{generalizationPL <-}\StringTok{ }\NormalTok{testing }\OperatorTok
\StringTok{  }\KeywordTok{filter}\NormalTok{(task }\OperatorTok{==}\StringTok{ 'generalizationPL'}\NormalTok{) }
\NormalTok{generalizationPL <-}\StringTok{ }\KeywordTok{droplevels}\NormalTok{(generalizationPL);}

\KeywordTok{as.factor}\NormalTok{(}\KeywordTok{gsub}\NormalTok{(}\StringTok{"/task/70033/56/asset/|/task/70033/57/asset/|/task/70033/58/asset/"}\NormalTok{, }\StringTok{""}\NormalTok{, generalizationPL}\OperatorTok{$}\NormalTok{resp))->}\StringTok{ }\NormalTok{generalizationPL}\OperatorTok{$}\NormalTok{resp}
\KeywordTok{as.factor}\NormalTok{(}\KeywordTok{gsub}\NormalTok{(}\StringTok{".mp3$"}\NormalTok{, }\StringTok{""}\NormalTok{, generalizationPL}\OperatorTok{$}\NormalTok{resp))->}\StringTok{ }\NormalTok{generalizationPL}\OperatorTok{$}\NormalTok{resp}
\KeywordTok{as.factor}\NormalTok{(}\KeywordTok{gsub}\NormalTok{(}\StringTok{".jpg"}\NormalTok{, }\StringTok{""}\NormalTok{, generalizationPL}\OperatorTok{$}\NormalTok{resp))->}\StringTok{ }\NormalTok{generalizationPL}\OperatorTok{$}\NormalTok{resp}

\KeywordTok{gsub}\NormalTok{(}\StringTok{'[[:punct:]]|"'}\NormalTok{, }\StringTok{""}\NormalTok{, generalizationPL}\OperatorTok{$}\NormalTok{label)->}\StringTok{ }\NormalTok{generalizationPL}\OperatorTok{$}\NormalTok{label }

\KeywordTok{as.factor}\NormalTok{(}\KeywordTok{gsub}\NormalTok{(}\StringTok{'mp3'}\NormalTok{, }\StringTok{"_"}\NormalTok{, generalizationPL}\OperatorTok{$}\NormalTok{label))->}\StringTok{ }\NormalTok{generalizationPL}\OperatorTok{$}\NormalTok{label}
\end{Highlighting}
\end{Shaded}

\hypertarget{check-how-many-trials-participants}{%
\subsubsection{Check how many trials
participants}\label{check-how-many-trials-participants}}

\begin{Shaded}
\begin{Highlighting}[]
\NormalTok{generalizationPL }\OperatorTok\StringTok{                             }
\StringTok{  }\KeywordTok{group_by}\NormalTok{(subjID) }\OperatorTok\StringTok{  }
\StringTok{  }\KeywordTok{count}\NormalTok{() }
\end{Highlighting}
\end{Shaded}

\begin{verbatim}
## # A tibble: 120 x 2
## # Groups:   subjID [120]
##     subjID     n
##      <int> <int>
##  1 1414932    24
##  2 1414933    24
##  3 1414937    24
##  4 1414945    24
##  5 1414957    24
##  6 1415040    24
##  7 1420163    24
##  8 1420165    24
##  9 1420169    24
## 10 1420171    24
## # ... with 110 more rows
\end{verbatim}

Great, 24 trials per participant.

\hypertarget{check-whether-participants-saw-a-unique-fribble}{%
\subsubsection{Check whether participants saw a unique
fribble:}\label{check-whether-participants-saw-a-unique-fribble}}

\begin{Shaded}
\begin{Highlighting}[]
\NormalTok{generalizationPL }\OperatorTok\StringTok{                             }
\StringTok{  }\KeywordTok{group_by}\NormalTok{(subjID, fribbleID) }\OperatorTok\StringTok{  }
\StringTok{  }\KeywordTok{count}\NormalTok{() }\OperatorTok
\StringTok{  }\KeywordTok{filter}\NormalTok{(n }\OperatorTok{>}\StringTok{ }\DecValTok{1}\NormalTok{)}
\end{Highlighting}
\end{Shaded}

\begin{verbatim}
## Warning: Factor `fribbleID` contains implicit NA, consider using
## `forcats::fct_explicit_na`
\end{verbatim}

\begin{verbatim}
## # A tibble: 0 x 3
## # Groups:   subjID, fribbleID [1]
## # ... with 3 variables: subjID <int>, fribbleID <fct>, n <int>
\end{verbatim}

Great!

Integrate stimuli info. In the file ``fribbleSet'' I have listed all the
fribbles ID and their category, along with their cueIDs and body shape.
I'm going to add those columns by merging the test file with the
fribbleSet by fribbleID. The rest of the file is left untouched.

\begin{Shaded}
\begin{Highlighting}[]
\KeywordTok{merge}\NormalTok{(generalizationPL, fribbleSet, }\DataTypeTok{by =} \StringTok{'fribbleID'}\NormalTok{)->}\StringTok{ }\NormalTok{generalizationPL;}
\NormalTok{generalizationPL}\OperatorTok{$}\NormalTok{label.y <-}\StringTok{ }\OtherTok{NULL}\NormalTok{;}

\NormalTok{generalizationPL <-}\StringTok{ }\KeywordTok{rename}\NormalTok{(generalizationPL, }\DataTypeTok{label =}\NormalTok{ label.x);}
\end{Highlighting}
\end{Shaded}

Let's check the responses they made, just to see if they make sense.

For example, we want the resp column to be one of the labels.

\begin{Shaded}
\begin{Highlighting}[]
\NormalTok{generalizationPL }\OperatorTok\StringTok{                             }
\StringTok{  }\KeywordTok{group_by}\NormalTok{(subjID, resp) }\OperatorTok\StringTok{  }
\StringTok{  }\KeywordTok{count}\NormalTok{() }
\end{Highlighting}
\end{Shaded}

\begin{verbatim}
## Warning: Factor `resp` contains implicit NA, consider using
## `forcats::fct_explicit_na`

## Warning: Factor `resp` contains implicit NA, consider using
## `forcats::fct_explicit_na`

## Warning: Factor `resp` contains implicit NA, consider using
## `forcats::fct_explicit_na`
\end{verbatim}

\begin{verbatim}
## # A tibble: 434 x 3
## # Groups:   subjID, resp [434]
##     subjID resp      n
##      <int> <fct> <int>
##  1 1414932 bim       6
##  2 1414932 dep       5
##  3 1414932 tob       9
##  4 1414932 <NA>      4
##  5 1414933 bim       8
##  6 1414933 dep       8
##  7 1414933 tob       8
##  8 1414937 bim       8
##  9 1414937 dep       7
## 10 1414937 tob       8
## # ... with 424 more rows
\end{verbatim}

Great, some participant missed some trials (coded as NA), but that's
okay.

So far, so good.

\hypertarget{check-trialstimuli-per-category-per-frequency-per-subject}{%
\subsubsection{Check trial/stimuli per category, per frequency, per
subject}\label{check-trialstimuli-per-category-per-frequency-per-subject}}

We have 24 trials per participant, but within those trials we
\emph{should} have 8 trials per category, 4 low frequency and 4 high
frequency trials.

\begin{Shaded}
\begin{Highlighting}[]
\KeywordTok{head}\NormalTok{(}\KeywordTok{table}\NormalTok{(generalizationPL}\OperatorTok{$}\NormalTok{subjID, generalizationPL}\OperatorTok{$}\NormalTok{category, generalizationPL}\OperatorTok{$}\NormalTok{frequency))}
\end{Highlighting}
\end{Shaded}

\begin{verbatim}
## , ,  = 25
## 
##          
##           1 2 3
##   1414932 4 4 4
##   1414933 4 4 4
##   1414937 4 4 4
##   1414945 4 4 4
##   1414957 4 4 4
##   1415040 4 4 4
## 
## , ,  = 75
## 
##          
##           1 2 3
##   1414932 4 4 4
##   1414933 4 4 4
##   1414937 4 4 4
##   1414945 4 4 4
##   1414957 4 4 4
##   1415040 4 4 4
\end{verbatim}

Let's check the second task.

\hypertarget{check-test-2-generalization-from-label-to-pictures}{%
\subsection{Check test 2: Generalization from label to
pictures}\label{check-test-2-generalization-from-label-to-pictures}}

\begin{Shaded}
\begin{Highlighting}[]
\NormalTok{generalizationLP <-}\StringTok{ }\NormalTok{testing }\OperatorTok
\StringTok{  }\KeywordTok{filter}\NormalTok{(task }\OperatorTok{==}\StringTok{ 'generalizationLP'}\NormalTok{) }
\NormalTok{generalizationLP <-}\StringTok{ }\KeywordTok{droplevels}\NormalTok{(generalizationLP)}
\end{Highlighting}
\end{Shaded}

\hypertarget{how-many-trials-per-participant-1}{%
\subsubsection{How many trials per
participant?}\label{how-many-trials-per-participant-1}}

\begin{Shaded}
\begin{Highlighting}[]
\NormalTok{generalizationLP }\OperatorTok\StringTok{                             }
\StringTok{  }\KeywordTok{group_by}\NormalTok{(subjID) }\OperatorTok\StringTok{  }
\StringTok{  }\KeywordTok{count}\NormalTok{() }
\end{Highlighting}
\end{Shaded}

\begin{verbatim}
## # A tibble: 120 x 2
## # Groups:   subjID [120]
##     subjID     n
##      <int> <int>
##  1 1414932    24
##  2 1414933    24
##  3 1414937    24
##  4 1414945    24
##  5 1414957    24
##  6 1415040    24
##  7 1420163    24
##  8 1420165    24
##  9 1420169    24
## 10 1420171    24
## # ... with 110 more rows
\end{verbatim}

24 trials, great.

\hypertarget{check-whether-participants-saw-a-unique-fribble-1}{%
\subsubsection{Check whether participants saw a unique
fribble}\label{check-whether-participants-saw-a-unique-fribble-1}}

First let's clean the rows from Gorilla gibberish.

\begin{Shaded}
\begin{Highlighting}[]
\KeywordTok{as.factor}\NormalTok{(}\KeywordTok{gsub}\NormalTok{(}\StringTok{'[[:punct:]]|"'}\NormalTok{, }\StringTok{""}\NormalTok{, generalizationLP}\OperatorTok{$}\NormalTok{fribbleID))->}\StringTok{ }\NormalTok{generalizationLP}\OperatorTok{$}\NormalTok{fribbleID }

\KeywordTok{as.factor}\NormalTok{(}\KeywordTok{gsub}\NormalTok{(}\StringTok{'jpg'}\NormalTok{, }\StringTok{"_"}\NormalTok{, generalizationLP}\OperatorTok{$}\NormalTok{fribbleID))->}\StringTok{ }\NormalTok{generalizationLP}\OperatorTok{$}\NormalTok{fribbleID}

\KeywordTok{as.factor}\NormalTok{(}\KeywordTok{gsub}\NormalTok{(}\StringTok{"/task/70033/56/asset/|/task/70033/57/asset/|/task/70033/58/asset/"}\NormalTok{, }\StringTok{""}\NormalTok{, generalizationLP}\OperatorTok{$}\NormalTok{resp))->}\StringTok{ }\NormalTok{generalizationLP}\OperatorTok{$}\NormalTok{resp}

\KeywordTok{as.factor}\NormalTok{(}\KeywordTok{gsub}\NormalTok{(}\StringTok{".jpg"}\NormalTok{, }\StringTok{""}\NormalTok{, generalizationLP}\OperatorTok{$}\NormalTok{resp))->}\StringTok{ }\NormalTok{generalizationLP}\OperatorTok{$}\NormalTok{resp}
\end{Highlighting}
\end{Shaded}

Then check for duplicates:

\begin{Shaded}
\begin{Highlighting}[]
\KeywordTok{substr}\NormalTok{(}\KeywordTok{as.character}\NormalTok{(generalizationLP}\OperatorTok{$}\NormalTok{fribbleID), }\DecValTok{1}\NormalTok{, }\DecValTok{5}\NormalTok{)->}\StringTok{ }\NormalTok{temp}
\KeywordTok{substr}\NormalTok{(}\KeywordTok{as.character}\NormalTok{(generalizationLP}\OperatorTok{$}\NormalTok{fribbleID), }\DecValTok{7}\NormalTok{, }\DecValTok{11}\NormalTok{)->}\StringTok{ }\NormalTok{temp2}
\KeywordTok{substr}\NormalTok{(}\KeywordTok{as.character}\NormalTok{(generalizationLP}\OperatorTok{$}\NormalTok{fribbleID), }\DecValTok{13}\NormalTok{, }\DecValTok{17}\NormalTok{)->}\StringTok{ }\NormalTok{temp3}

\NormalTok{fribblePresented <-}\StringTok{ }\KeywordTok{c}\NormalTok{(temp,temp2,temp3)}
\KeywordTok{unique}\NormalTok{(generalizationLP}\OperatorTok{$}\NormalTok{subjID)->}\StringTok{ }\NormalTok{subj}

\NormalTok{duplicatedFribbles <-}\StringTok{ }\OtherTok{NA}\NormalTok{;}
\ControlFlowTok{for}\NormalTok{ (i }\ControlFlowTok{in} \DecValTok{1}\OperatorTok{:}\KeywordTok{length}\NormalTok{(subj))\{}
  \KeywordTok{substr}\NormalTok{(}\KeywordTok{as.character}\NormalTok{(generalizationLP[generalizationLP}\OperatorTok{$}\NormalTok{subjID}\OperatorTok{==}\NormalTok{subj[i],]}\OperatorTok{$}\NormalTok{fribbleID), }\DecValTok{1}\NormalTok{, }\DecValTok{5}\NormalTok{)->}\StringTok{ }\NormalTok{temp}
  \KeywordTok{substr}\NormalTok{(}\KeywordTok{as.character}\NormalTok{(generalizationLP[generalizationLP}\OperatorTok{$}\NormalTok{subjID}\OperatorTok{==}\NormalTok{subj[i],]}\OperatorTok{$}\NormalTok{fribbleID), }\DecValTok{7}\NormalTok{, }\DecValTok{11}\NormalTok{)->}\StringTok{ }\NormalTok{temp2}
  \KeywordTok{substr}\NormalTok{(}\KeywordTok{as.character}\NormalTok{(generalizationLP[generalizationLP}\OperatorTok{$}\NormalTok{subjID}\OperatorTok{==}\NormalTok{subj[i],]}\OperatorTok{$}\NormalTok{fribbleID), }\DecValTok{13}\NormalTok{, }\DecValTok{17}\NormalTok{)->}\StringTok{ }\NormalTok{temp3}
\NormalTok{  fribblePresented <-}\StringTok{ }\KeywordTok{c}\NormalTok{(temp,temp2,temp3)}
\NormalTok{  dup <-}\StringTok{ }\NormalTok{fribblePresented[}\KeywordTok{duplicated}\NormalTok{(fribblePresented)] }\CommentTok{#extract duplicated elements}
  \KeywordTok{print}\NormalTok{(subj[i])}
  
  \ControlFlowTok{if}\NormalTok{ (}\KeywordTok{length}\NormalTok{(dup)}\OperatorTok{>}\DecValTok{0}\NormalTok{)\{}
    \KeywordTok{print}\NormalTok{(dup)}
\NormalTok{  \} }\ControlFlowTok{else}\NormalTok{ \{}
    \KeywordTok{print}\NormalTok{(}\KeywordTok{length}\NormalTok{(dup))}
\NormalTok{  \}}
  
\NormalTok{\};}
\end{Highlighting}
\end{Shaded}

\begin{verbatim}
## [1] 1414937
## [1] 0
## [1] 1414945
## [1] 0
## [1] 1414957
## [1] 0
## [1] 1415040
## [1] 0
## [1] 1431949
## [1] 0
## [1] 1431944
## [1] 0
## [1] 1431953
## [1] 0
## [1] 1431958
## [1] 0
## [1] 1431965
## [1] 0
## [1] 1431946
## [1] 0
## [1] 1431957
## [1] 0
## [1] 1431961
## [1] 0
## [1] 1431969
## [1] 0
## [1] 1431978
## [1] 0
## [1] 1431979
## [1] 0
## [1] 1422477
## [1] 0
## [1] 1422675
## [1] 0
## [1] 1422677
## [1] 0
## [1] 1422679
## [1] 0
## [1] 1422689
## [1] 0
## [1] 1422716
## [1] 0
## [1] 1431972
## [1] 0
## [1] 1431974
## [1] 0
## [1] 1431984
## [1] 0
## [1] 1431992
## [1] 0
## [1] 1431997
## [1] 0
## [1] 1432009
## [1] 0
## [1] 1432011
## [1] 0
## [1] 1432052
## [1] 0
## [1] 1432075
## [1] 0
## [1] 1420163
## [1] 0
## [1] 1420165
## [1] 0
## [1] 1420177
## [1] 0
## [1] 1420180
## [1] 0
## [1] 1420199
## [1] 0
## [1] 1420580
## [1] 0
## [1] 1420622
## [1] 0
## [1] 1422463
## [1] 0
## [1] 1422465
## [1] 0
## [1] 1422466
## [1] 0
## [1] 1422472
## [1] 0
## [1] 1459007
## [1] 0
## [1] 1459002
## [1] 0
## [1] 1459009
## [1] 0
## [1] 1459001
## [1] 0
## [1] 1459003
## [1] 0
## [1] 1459013
## [1] 0
## [1] 1459029
## [1] 0
## [1] 1458992
## [1] 0
## [1] 1459018
## [1] 0
## [1] 1459024
## [1] 0
## [1] 1459047
## [1] 0
## [1] 1459052
## [1] 0
## [1] 1459064
## [1] 0
## [1] 1459048
## [1] 0
## [1] 1459057
## [1] 0
## [1] 1459697
## [1] 0
## [1] 1459696
## [1] 0
## [1] 1459706
## [1] 0
## [1] 1459702
## [1] 0
## [1] 1459708
## [1] 0
## [1] 1459703
## [1] 0
## [1] 1459767
## [1] 0
## [1] 1414933
## [1] 0
## [1] 1414932
## [1] 0
## [1] 1420169
## [1] 0
## [1] 1420171
## [1] 0
## [1] 1420577
## [1] 0
## [1] 1422467
## [1] 0
## [1] 1422475
## [1] 0
## [1] 1422678
## [1] 0
## [1] 1422680
## [1] 0
## [1] 1422681
## [1] 0
## [1] 1431942
## [1] 0
## [1] 1431948
## [1] 0
## [1] 1431966
## [1] 0
## [1] 1431968
## [1] 0
## [1] 1431952
## [1] 0
## [1] 1431954
## [1] 0
## [1] 1431956
## [1] 0
## [1] 1431959
## [1] 0
## [1] 1431960
## [1] 0
## [1] 1431963
## [1] 0
## [1] 1431970
## [1] 0
## [1] 1431981
## [1] 0
## [1] 1431989
## [1] 0
## [1] 1431998
## [1] 0
## [1] 1431999
## [1] 0
## [1] 1432003
## [1] 0
## [1] 1432007
## [1] 0
## [1] 1432030
## [1] 0
## [1] 1420185
## [1] 0
## [1] 1420204
## [1] 0
## [1] 1420552
## [1] 0
## [1] 1420573
## [1] 0
## [1] 1422470
## [1] 0
## [1] 1422473
## [1] 0
## [1] 1422476
## [1] 0
## [1] 1422676
## [1] 0
## [1] 1422715
## [1] 0
## [1] 1432301
## [1] 0
## [1] 1432323
## [1] 0
## [1] 1457883
## [1] 0
## [1] 1458997
## [1] 0
## [1] 1459015
## [1] 0
## [1] 1459025
## [1] 0
## [1] 1458998
## [1] 0
## [1] 1458996
## [1] 0
## [1] 1459043
## [1] 0
## [1] 1459036
## [1] 0
## [1] 1459039
## [1] 0
## [1] 1459046
## [1] 0
## [1] 1459067
## [1] 0
## [1] 1459020
## [1] 0
## [1] 1459078
## [1] 0
## [1] 1459109
## [1] 0
## [1] 1459701
## [1] 0
## [1] 1459700
## [1] 0
## [1] 1459709
## [1] 0
## [1] 1459699
## [1] 0
\end{verbatim}

\begin{Shaded}
\begin{Highlighting}[]
\KeywordTok{rm}\NormalTok{(subj, temp, temp2, temp3, i, fribblePresented, duplicatedFribbles, dup)}
\end{Highlighting}
\end{Shaded}

Great! participants saw always different fribble.

\hypertarget{check-whether-fribbles-presented-were-either-high-or-low-frequency-within-trials}{%
\subsubsection{Check whether fribbles presented were either high or low
frequency within
trials}\label{check-whether-fribbles-presented-were-either-high-or-low-frequency-within-trials}}

In this task we have three pictures and one label pronounced. This means
that the fribbleID column contains 3 images. I'm going to cycle over the
dataset, and break the fribbleID column in three, then I'm going to
print the fribble that within the same trial has a different frequency.
I'm going to print the fribbles that are presented wrongly, e.g., ``low
high low'' etc. If all fribbles are presented correctly: , e.g., ``low
low low'' and ``high high high'', then the output is empty.

\begin{Shaded}
\begin{Highlighting}[]
\KeywordTok{unique}\NormalTok{(generalizationLP}\OperatorTok{$}\NormalTok{subjID)->}\StringTok{ }\NormalTok{subj;}

\NormalTok{trials <-}\StringTok{ }\OtherTok{NULL}\NormalTok{;}
\NormalTok{task <-}\StringTok{ }\OtherTok{NULL}\NormalTok{;}

\ControlFlowTok{for}\NormalTok{ (i }\ControlFlowTok{in} \DecValTok{1}\OperatorTok{:}\KeywordTok{length}\NormalTok{(subj))\{}
  \KeywordTok{as.integer}\NormalTok{(}\KeywordTok{substr}\NormalTok{(}\KeywordTok{as.character}\NormalTok{(generalizationLP[generalizationLP}\OperatorTok{$}\NormalTok{subjID}\OperatorTok{==}\NormalTok{subj[i],]}\OperatorTok{$}\NormalTok{fribbleID), }\DecValTok{4}\NormalTok{, }\DecValTok{5}\NormalTok{))->}\StringTok{ }\NormalTok{temp }\CommentTok{#first fribble}
  \KeywordTok{as.integer}\NormalTok{(}\KeywordTok{substr}\NormalTok{(}\KeywordTok{as.character}\NormalTok{(generalizationLP[generalizationLP}\OperatorTok{$}\NormalTok{subjID}\OperatorTok{==}\NormalTok{subj[i],]}\OperatorTok{$}\NormalTok{fribbleID), }\DecValTok{10}\NormalTok{, }\DecValTok{11}\NormalTok{))->}\StringTok{ }\NormalTok{temp2 }\CommentTok{#second fribble}
  \KeywordTok{as.integer}\NormalTok{(}\KeywordTok{substr}\NormalTok{(}\KeywordTok{as.character}\NormalTok{(generalizationLP[generalizationLP}\OperatorTok{$}\NormalTok{subjID}\OperatorTok{==}\NormalTok{subj[i],]}\OperatorTok{$}\NormalTok{fribbleID), }\DecValTok{16}\NormalTok{, }\DecValTok{17}\NormalTok{))->}\StringTok{ }\NormalTok{temp3 }\CommentTok{#third fribble}
\NormalTok{trials <-}\StringTok{ }\KeywordTok{cbind}\NormalTok{(temp, temp2, temp3, }\KeywordTok{as.integer}\NormalTok{(subj[i])) }\CommentTok{# store it in columns along with subj info}
\NormalTok{task <-}\StringTok{ }\KeywordTok{rbind}\NormalTok{(task, trials) }\CommentTok{#store all subjs}
\NormalTok{\};}

\ControlFlowTok{for}\NormalTok{ (i }\ControlFlowTok{in} \DecValTok{1}\OperatorTok{:}\KeywordTok{nrow}\NormalTok{(task))\{ }\CommentTok{#check by rows whether there is a unique number, print the row if wrong}
  \ControlFlowTok{if}\NormalTok{ ((task[i,}\DecValTok{1}\NormalTok{] }\OperatorTok{==}\StringTok{ }\NormalTok{task[i,}\DecValTok{2}\NormalTok{] }\OperatorTok{&}\StringTok{ }\NormalTok{task[i,}\DecValTok{3}\NormalTok{])}\OperatorTok{==}\StringTok{ }\OtherTok{FALSE}\NormalTok{) \{}
    \KeywordTok{print}\NormalTok{(}\StringTok{'wrong frequency fribble:'}\NormalTok{)}
    \KeywordTok{print}\NormalTok{(task[i,}\DecValTok{1}\NormalTok{], task[i,}\DecValTok{2}\NormalTok{], task[i,}\DecValTok{3}\NormalTok{])}
\NormalTok{  \} }
\NormalTok{\};}

\NormalTok{frequency <-}\StringTok{ }\KeywordTok{ifelse}\NormalTok{(}\KeywordTok{substr}\NormalTok{(}\KeywordTok{as.character}\NormalTok{(task[,}\DecValTok{1}\NormalTok{]), }\DecValTok{1}\NormalTok{, }\DecValTok{1}\NormalTok{)}\OperatorTok{==}\DecValTok{2}\NormalTok{, }\StringTok{'low'}\NormalTok{, }\StringTok{'high'}\NormalTok{)}
\KeywordTok{cbind}\NormalTok{(task, frequency)->task}
\KeywordTok{as.data.frame}\NormalTok{(task)->}\StringTok{ }\NormalTok{task}
\KeywordTok{rm}\NormalTok{(trials, i, subj, temp, temp2, temp3);}
\end{Highlighting}
\end{Shaded}

Great, fribbles presented were either low or high frequency. Check
whether participants saw 4 trials with low and 4 trials with high
frequency:

\hypertarget{check-trial-distribution-per-frequency}{%
\subsubsection{Check trial distribution per
frequency:}\label{check-trial-distribution-per-frequency}}

\begin{Shaded}
\begin{Highlighting}[]
\KeywordTok{head}\NormalTok{(}\KeywordTok{table}\NormalTok{(task}\OperatorTok{$}\NormalTok{V4, task}\OperatorTok{$}\NormalTok{frequency))}
\end{Highlighting}
\end{Shaded}

\begin{verbatim}
##          
##           high low
##   1414932   12  12
##   1414933   12  12
##   1414937   12  12
##   1414945   12  12
##   1414957   12  12
##   1415040   12  12
\end{verbatim}

I'm going to merge the stimuli set now.

When we do it, this time we need to merge by resp and not by fribbleID,
because our fribble selected is coded in this column:

\begin{Shaded}
\begin{Highlighting}[]
\NormalTok{fribbleSet}\OperatorTok{$}\NormalTok{resp <-}\StringTok{ }\NormalTok{fribbleSet}\OperatorTok{$}\NormalTok{fribbleID }\CommentTok{# column's name needs to be the same in order to merge}
\KeywordTok{merge}\NormalTok{(generalizationLP, fribbleSet, }\DataTypeTok{by =} \StringTok{'resp'}\NormalTok{, }\DataTypeTok{all.x =}\NormalTok{ T)->}\StringTok{ }\NormalTok{generalizationLP;}
\NormalTok{fribbleSet}\OperatorTok{$}\NormalTok{resp <-}\StringTok{ }\OtherTok{NULL}\NormalTok{;}
\NormalTok{generalizationLP}\OperatorTok{$}\NormalTok{fribbleID.y <-}\StringTok{ }\OtherTok{NULL}\NormalTok{;}
\NormalTok{generalizationLP}\OperatorTok{$}\NormalTok{label.y <-}\StringTok{ }\OtherTok{NULL}\NormalTok{;}
\NormalTok{generalizationLP <-}\StringTok{ }\KeywordTok{rename}\NormalTok{(generalizationLP, }\DataTypeTok{label =}\NormalTok{ label.x);}
\NormalTok{generalizationLP <-}\StringTok{ }\KeywordTok{rename}\NormalTok{(generalizationLP, }\DataTypeTok{fribbleID =}\NormalTok{ fribbleID.x);}
\end{Highlighting}
\end{Shaded}

\hypertarget{check-responses-distribution-by-category}{%
\subsubsection{Check responses distribution by
category:}\label{check-responses-distribution-by-category}}

\begin{Shaded}
\begin{Highlighting}[]
\NormalTok{generalizationLP }\OperatorTok\StringTok{                             }
\StringTok{  }\KeywordTok{group_by}\NormalTok{(subjID, category) }\OperatorTok\StringTok{  }
\StringTok{  }\KeywordTok{count}\NormalTok{()}
\end{Highlighting}
\end{Shaded}

\begin{verbatim}
## # A tibble: 427 x 3
## # Groups:   subjID, category [427]
##     subjID category     n
##      <int>    <int> <int>
##  1 1414932        1     7
##  2 1414932        2    11
##  3 1414932        3     2
##  4 1414932       NA     4
##  5 1414933        1     8
##  6 1414933        2     5
##  7 1414933        3    10
##  8 1414933       NA     1
##  9 1414937        1     7
## 10 1414937        2     7
## # ... with 417 more rows
\end{verbatim}

Cool.

\hypertarget{check-responses-distribution-by-frequency}{%
\subsubsection{Check responses distribution by
frequency:}\label{check-responses-distribution-by-frequency}}

\begin{Shaded}
\begin{Highlighting}[]
\NormalTok{generalizationLP }\OperatorTok\StringTok{                             }
\StringTok{  }\KeywordTok{group_by}\NormalTok{(subjID, label, frequency) }\OperatorTok\StringTok{  }
\StringTok{  }\KeywordTok{count}\NormalTok{()}
\end{Highlighting}
\end{Shaded}

\begin{verbatim}
## # A tibble: 837 x 4
## # Groups:   subjID, label, frequency [837]
##     subjID label frequency     n
##      <int> <fct>     <int> <int>
##  1 1414932 bim          25     3
##  2 1414932 bim          75     4
##  3 1414932 bim          NA     1
##  4 1414932 dep          25     3
##  5 1414932 dep          75     3
##  6 1414932 dep          NA     2
##  7 1414932 tob          25     3
##  8 1414932 tob          75     4
##  9 1414932 tob          NA     1
## 10 1414933 bim          25     4
## # ... with 827 more rows
\end{verbatim}

\hypertarget{check-test-3-contingency-judgement-task}{%
\subsection{Check test 3: Contingency Judgement
task}\label{check-test-3-contingency-judgement-task}}

\begin{Shaded}
\begin{Highlighting}[]
\NormalTok{contingencyJudgement <-}\StringTok{ }\NormalTok{testing }\OperatorTok
\StringTok{  }\KeywordTok{filter}\NormalTok{(task }\OperatorTok{==}\StringTok{ 'contingencyJudgement'}\NormalTok{) }
\NormalTok{contingencyJudgement <-}\StringTok{ }\KeywordTok{droplevels}\NormalTok{(contingencyJudgement)}
\end{Highlighting}
\end{Shaded}

\hypertarget{how-many-trials-per-participant-2}{%
\subsubsection{How many trials per
participant?}\label{how-many-trials-per-participant-2}}

\begin{Shaded}
\begin{Highlighting}[]
\NormalTok{contingencyJudgement }\OperatorTok\StringTok{                             }
\StringTok{  }\KeywordTok{group_by}\NormalTok{(subjID) }\OperatorTok\StringTok{  }
\StringTok{  }\KeywordTok{count}\NormalTok{() }
\end{Highlighting}
\end{Shaded}

\begin{verbatim}
## # A tibble: 120 x 2
## # Groups:   subjID [120]
##     subjID     n
##      <int> <int>
##  1 1414932    24
##  2 1414933    24
##  3 1414937    24
##  4 1414945    24
##  5 1414957    24
##  6 1415040    24
##  7 1420163    24
##  8 1420165    24
##  9 1420169    24
## 10 1420171    24
## # ... with 110 more rows
\end{verbatim}

Very good.

\hypertarget{did-participants-see-a-fribble-more-than-once}{%
\subsubsection{Did participants see a fribble more than
once?}\label{did-participants-see-a-fribble-more-than-once}}

\begin{Shaded}
\begin{Highlighting}[]
\KeywordTok{droplevels}\NormalTok{(contingencyJudgement) }\OperatorTok\StringTok{                             }
\StringTok{  }\KeywordTok{group_by}\NormalTok{(subjID, fribbleID) }\OperatorTok\StringTok{  }
\StringTok{  }\KeywordTok{count}\NormalTok{() }\OperatorTok
\StringTok{  }\KeywordTok{filter}\NormalTok{( n }\OperatorTok{>}\StringTok{ }\DecValTok{1}\NormalTok{)}
\end{Highlighting}
\end{Shaded}

\begin{verbatim}
## Warning: Factor `fribbleID` contains implicit NA, consider using
## `forcats::fct_explicit_na`
\end{verbatim}

\begin{verbatim}
## # A tibble: 0 x 3
## # Groups:   subjID, fribbleID [1]
## # ... with 3 variables: subjID <int>, fribbleID <fct>, n <int>
\end{verbatim}

No! that's great.

\hypertarget{are-labels-repeated-equally}{%
\subsubsection{Are labels repeated
equally?}\label{are-labels-repeated-equally}}

\begin{Shaded}
\begin{Highlighting}[]
\KeywordTok{table}\NormalTok{(contingencyJudgement}\OperatorTok{$}\NormalTok{subjID, contingencyJudgement}\OperatorTok{$}\NormalTok{label)}
\end{Highlighting}
\end{Shaded}

\begin{verbatim}
##          
##           bim dep tob
##   1414932   8   8   8
##   1414933   8   8   8
##   1414937   8   8   8
##   1414945   8   8   8
##   1414957   8   8   8
##   1415040   8   8   8
##   1420163   8   8   8
##   1420165   8   8   8
##   1420169   8   8   8
##   1420171   8   8   8
##   1420177   8   8   8
##   1420180   8   8   8
##   1420185   8   8   8
##   1420199   8   8   8
##   1420204   8   8   8
##   1420552   8   8   8
##   1420573   8   8   8
##   1420577   8   8   8
##   1420580   8   8   8
##   1420622   8   8   8
##   1422463   8   8   8
##   1422465   8   8   8
##   1422466   8   8   8
##   1422467   8   8   8
##   1422470   8   8   8
##   1422472   8   8   8
##   1422473   8   8   8
##   1422475   8   8   8
##   1422476   8   8   8
##   1422477   8   8   8
##   1422675   8   8   8
##   1422676   8   8   8
##   1422677   8   8   8
##   1422678   8   8   8
##   1422679   8   8   8
##   1422680   8   8   8
##   1422681   8   8   8
##   1422689   8   8   8
##   1422715   8   8   8
##   1422716   8   8   8
##   1431942   8   8   8
##   1431944   8   8   8
##   1431946   8   8   8
##   1431948   8   8   8
##   1431949   8   8   8
##   1431952   8   8   8
##   1431953   8   8   8
##   1431954   8   8   8
##   1431956   8   8   8
##   1431957   8   8   8
##   1431958   8   8   8
##   1431959   8   8   8
##   1431960   8   8   8
##   1431961   8   8   8
##   1431963   8   8   8
##   1431965   8   8   8
##   1431966   8   8   8
##   1431968   8   8   8
##   1431969   8   8   8
##   1431970   8   8   8
##   1431972   8   8   8
##   1431974   8   8   8
##   1431978   8   8   8
##   1431979   8   8   8
##   1431981   8   8   8
##   1431984   8   8   8
##   1431989   8   8   8
##   1431992   8   8   8
##   1431997   8   8   8
##   1431998   8   8   8
##   1431999   8   8   8
##   1432003   8   8   8
##   1432007   8   8   8
##   1432009   8   8   8
##   1432011   8   8   8
##   1432030   8   8   8
##   1432052   8   8   8
##   1432075   8   8   8
##   1432301   8   8   8
##   1432323   8   8   8
##   1457883   8   8   8
##   1458992   8   8   8
##   1458996   8   8   8
##   1458997   8   8   8
##   1458998   8   8   8
##   1459001   8   8   8
##   1459002   8   8   8
##   1459003   8   8   8
##   1459007   8   8   8
##   1459009   8   8   8
##   1459013   8   8   8
##   1459015   8   8   8
##   1459018   8   8   8
##   1459020   8   8   8
##   1459024   8   8   8
##   1459025   8   8   8
##   1459029   8   8   8
##   1459036   8   8   8
##   1459039   8   8   8
##   1459043   8   8   8
##   1459046   8   8   8
##   1459047   8   8   8
##   1459048   8   8   8
##   1459052   8   8   8
##   1459057   8   8   8
##   1459064   8   8   8
##   1459067   8   8   8
##   1459078   8   8   8
##   1459109   8   8   8
##   1459696   8   8   8
##   1459697   8   8   8
##   1459699   8   8   8
##   1459700   8   8   8
##   1459701   8   8   8
##   1459702   8   8   8
##   1459703   8   8   8
##   1459706   8   8   8
##   1459708   8   8   8
##   1459709   8   8   8
##   1459767   8   8   8
\end{verbatim}

good

\begin{Shaded}
\begin{Highlighting}[]
\KeywordTok{merge}\NormalTok{(contingencyJudgement, fribbleSet, }\DataTypeTok{by =} \StringTok{'fribbleID'}\NormalTok{)->}\StringTok{ }\NormalTok{contingencyJudgement}
\NormalTok{contingencyJudgement}\OperatorTok{$}\NormalTok{label.y <-}\StringTok{ }\OtherTok{NULL}\NormalTok{;}
\NormalTok{contingencyJudgement <-}\StringTok{ }\KeywordTok{rename}\NormalTok{(contingencyJudgement, }\DataTypeTok{label =}\NormalTok{ label.x)}
\end{Highlighting}
\end{Shaded}

\hypertarget{check-category-presentation}{%
\subsubsection{Check category
presentation:}\label{check-category-presentation}}

\begin{Shaded}
\begin{Highlighting}[]
\NormalTok{contingencyJudgement }\OperatorTok\StringTok{                             }
\StringTok{  }\KeywordTok{group_by}\NormalTok{(subjID, category) }\OperatorTok\StringTok{  }
\StringTok{  }\KeywordTok{count}\NormalTok{()}
\end{Highlighting}
\end{Shaded}

\begin{verbatim}
## # A tibble: 360 x 3
## # Groups:   subjID, category [360]
##     subjID category     n
##      <int>    <int> <int>
##  1 1414932        1     8
##  2 1414932        2     8
##  3 1414932        3     8
##  4 1414933        1     8
##  5 1414933        2     8
##  6 1414933        3     8
##  7 1414937        1     8
##  8 1414937        2     8
##  9 1414937        3     8
## 10 1414945        1     8
## # ... with 350 more rows
\end{verbatim}

\begin{Shaded}
\begin{Highlighting}[]
\KeywordTok{table}\NormalTok{(contingencyJudgement}\OperatorTok{$}\NormalTok{category, contingencyJudgement}\OperatorTok{$}\NormalTok{label)}
\end{Highlighting}
\end{Shaded}

\begin{verbatim}
##    
##     bim dep tob
##   1 312 312 336
##   2 384 288 288
##   3 264 360 336
\end{verbatim}

\hypertarget{check-test-4-random-dot-task}{%
\subsection{Check test 4: Random dot
task}\label{check-test-4-random-dot-task}}

Let's check our random dot task. This was inserted randomly during
trials 4 times. 5 trials each time, plus 4 practice trials.

\begin{Shaded}
\begin{Highlighting}[]
\NormalTok{randomDot <-}\StringTok{ }\NormalTok{testing }\OperatorTok
\StringTok{  }\KeywordTok{filter}\NormalTok{(task }\OperatorTok{==}\StringTok{ 'randomDot'}\NormalTok{) }
\end{Highlighting}
\end{Shaded}

\hypertarget{how-many-trials-per-participant-3}{%
\subsubsection{How many trials per
participant?}\label{how-many-trials-per-participant-3}}

\begin{Shaded}
\begin{Highlighting}[]
\NormalTok{randomDot }\OperatorTok\StringTok{                             }
\StringTok{  }\KeywordTok{group_by}\NormalTok{(subjID) }\OperatorTok\StringTok{  }
\StringTok{  }\KeywordTok{count}\NormalTok{() }
\end{Highlighting}
\end{Shaded}

\begin{verbatim}
## # A tibble: 120 x 2
## # Groups:   subjID [120]
##     subjID     n
##      <int> <int>
##  1 1414932    26
##  2 1414933    26
##  3 1414937    26
##  4 1414945    26
##  5 1414957    26
##  6 1415040    26
##  7 1420163    26
##  8 1420165    26
##  9 1420169    26
## 10 1420171    26
## # ... with 110 more rows
\end{verbatim}

we have 5 trials repeated during learning four times (20) plus 4
practice trials.

\hypertarget{how-was-accuracy-distributed-across-participants}{%
\subsubsection{How was accuracy distributed across
participants?}\label{how-was-accuracy-distributed-across-participants}}

First, let's consider that when we have a timeout, the output is -1

\begin{Shaded}
\begin{Highlighting}[]
\NormalTok{randomDot }\OperatorTok\StringTok{                             }
\StringTok{  }\KeywordTok{group_by}\NormalTok{(subjID, resp) }\OperatorTok\StringTok{ }
\StringTok{  }\KeywordTok{filter}\NormalTok{(rt }\OperatorTok{==}\StringTok{ }\DecValTok{-1}\NormalTok{) }\OperatorTok
\StringTok{  }\KeywordTok{count}\NormalTok{()}
\end{Highlighting}
\end{Shaded}

\begin{verbatim}
## # A tibble: 83 x 3
## # Groups:   subjID, resp [83]
##     subjID resp      n
##      <int> <fct> <int>
##  1 1414932 -1       10
##  2 1414933 -1        1
##  3 1414945 -1        3
##  4 1415040 -1        1
##  5 1420163 -1        2
##  6 1420165 -1        1
##  7 1420180 -1        2
##  8 1420185 -1        1
##  9 1420204 -1        1
## 10 1420552 -1        3
## # ... with 73 more rows
\end{verbatim}

Here we can see that some participant missed some trials.

Let's see how accuracy is coded when response is -1:

\begin{Shaded}
\begin{Highlighting}[]
\KeywordTok{head}\NormalTok{(randomDot[randomDot}\OperatorTok{$}\NormalTok{rt }\OperatorTok{==}\StringTok{ }\DecValTok{-1}\NormalTok{,]}\OperatorTok{$}\NormalTok{acc)}
\end{Highlighting}
\end{Shaded}

\begin{verbatim}
## [1] NA NA NA NA NA NA
\end{verbatim}

So it is coded as ``NA'', great. However:

\begin{Shaded}
\begin{Highlighting}[]
\KeywordTok{nrow}\NormalTok{(randomDot[}\KeywordTok{is.na}\NormalTok{(randomDot}\OperatorTok{$}\NormalTok{acc),]) }\CommentTok{#total of NA}
\end{Highlighting}
\end{Shaded}

\begin{verbatim}
## [1] 325
\end{verbatim}

\begin{Shaded}
\begin{Highlighting}[]
\KeywordTok{nrow}\NormalTok{(randomDot[randomDot}\OperatorTok{$}\NormalTok{resp }\OperatorTok{==}\StringTok{ }\DecValTok{-1}\NormalTok{,]) }\CommentTok{# total of timeouts}
\end{Highlighting}
\end{Shaded}

\begin{verbatim}
## [1] 196
\end{verbatim}

There are more NA's in acc than can be explained by timeouts. This means
that also wrong responses are coded as NA. We need to recode those.

\begin{Shaded}
\begin{Highlighting}[]
\NormalTok{randomDot[}\KeywordTok{is.na}\NormalTok{(randomDot}\OperatorTok{$}\NormalTok{acc),]}\OperatorTok{$}\NormalTok{acc <-}\StringTok{ }\DecValTok{0} \CommentTok{#recode everything that is wrong or timeout as 0}
\end{Highlighting}
\end{Shaded}

\hypertarget{check-the-overall-accuracy-of-participants-filtering-by-timeouts}{%
\subsubsection{Check the overall accuracy of participants, filtering by
timeouts:}\label{check-the-overall-accuracy-of-participants-filtering-by-timeouts}}

\begin{Shaded}
\begin{Highlighting}[]
\KeywordTok{aggregate}\NormalTok{(acc }\OperatorTok{~}\StringTok{ }\NormalTok{subjID, }\DataTypeTok{data =}\NormalTok{ randomDot[}\OperatorTok{!}\NormalTok{(randomDot}\OperatorTok{$}\NormalTok{resp }\OperatorTok{==}\StringTok{ }\DecValTok{-1}\NormalTok{),], }\DataTypeTok{FUN =}\NormalTok{ mean)}\CommentTok{# without timeouts}
\end{Highlighting}
\end{Shaded}

\begin{verbatim}
##      subjID       acc
## 1   1414932 0.6875000
## 2   1414933 1.0000000
## 3   1414937 1.0000000
## 4   1414945 1.0000000
## 5   1414957 1.0000000
## 6   1415040 1.0000000
## 7   1420163 0.9583333
## 8   1420165 0.9600000
## 9   1420169 1.0000000
## 10  1420171 1.0000000
## 11  1420177 1.0000000
## 12  1420180 0.9583333
## 13  1420185 1.0000000
## 14  1420199 1.0000000
## 15  1420204 1.0000000
## 16  1420552 1.0000000
## 17  1420573 1.0000000
## 18  1420577 0.9583333
## 19  1420580 1.0000000
## 20  1420622 1.0000000
## 21  1422463 1.0000000
## 22  1422465 1.0000000
## 23  1422466 0.9565217
## 24  1422467 1.0000000
## 25  1422470 0.7600000
## 26  1422472 1.0000000
## 27  1422473 1.0000000
## 28  1422475 0.5200000
## 29  1422476 0.9600000
## 30  1422477 1.0000000
## 31  1422675 1.0000000
## 32  1422676 0.9615385
## 33  1422677 0.9047619
## 34  1422678 0.9600000
## 35  1422679 0.9565217
## 36  1422680 1.0000000
## 37  1422681 1.0000000
## 38  1422689 0.6000000
## 39  1422715 1.0000000
## 40  1422716 1.0000000
## 41  1431942 0.8461538
## 42  1431944 0.7619048
## 43  1431946 1.0000000
## 44  1431948 0.9600000
## 45  1431949 1.0000000
## 46  1431952 0.9565217
## 47  1431953 0.9615385
## 48  1431954 1.0000000
## 49  1431956 0.9166667
## 50  1431957 1.0000000
## 51  1431958 0.9615385
## 52  1431959 1.0000000
## 53  1431960 1.0000000
## 54  1431961 1.0000000
## 55  1431963 1.0000000
## 56  1431965 1.0000000
## 57  1431966 0.9600000
## 58  1431968 1.0000000
## 59  1431969 1.0000000
## 60  1431970 0.9565217
## 61  1431972 0.9600000
## 62  1431974 1.0000000
## 63  1431978 1.0000000
## 64  1431979 1.0000000
## 65  1431981 1.0000000
## 66  1431984 0.9600000
## 67  1431989 1.0000000
## 68  1431992 1.0000000
## 69  1431997 1.0000000
## 70  1431998 1.0000000
## 71  1431999 1.0000000
## 72  1432003 0.9130435
## 73  1432007 1.0000000
## 74  1432009 0.9600000
## 75  1432011 0.9090909
## 76  1432030 1.0000000
## 77  1432052 0.9166667
## 78  1432075 0.9600000
## 79  1432301 1.0000000
## 80  1432323 1.0000000
## 81  1457883 1.0000000
## 82  1458992 1.0000000
## 83  1458996 1.0000000
## 84  1458997 1.0000000
## 85  1458998 1.0000000
## 86  1459001 0.6521739
## 87  1459002 1.0000000
## 88  1459003 0.9600000
## 89  1459007 1.0000000
## 90  1459009 0.2916667
## 91  1459013 0.9600000
## 92  1459015 0.9565217
## 93  1459018 1.0000000
## 94  1459020 1.0000000
## 95  1459024 1.0000000
## 96  1459025 1.0000000
## 97  1459029 1.0000000
## 98  1459036 0.5652174
## 99  1459039 1.0000000
## 100 1459043 1.0000000
## 101 1459046 1.0000000
## 102 1459047 1.0000000
## 103 1459048 0.9523810
## 104 1459052 1.0000000
## 105 1459057 0.9166667
## 106 1459064 1.0000000
## 107 1459067 1.0000000
## 108 1459078 0.8800000
## 109 1459109 0.9583333
## 110 1459696 0.8333333
## 111 1459697 1.0000000
## 112 1459699 1.0000000
## 113 1459700 0.9600000
## 114 1459701 0.9615385
## 115 1459702 1.0000000
## 116 1459703 0.6956522
## 117 1459706 0.9565217
## 118 1459708 1.0000000
## 119 1459709 1.0000000
## 120 1459767 1.0000000
\end{verbatim}

Now that we have all tests separated, better to remove this file:

\hypertarget{data-visualization}{%
\section{Data visualization}\label{data-visualization}}

Okay, from the sanity checks done above we can draw two conclusions:

\begin{enumerate}
\def\labelenumi{\arabic{enumi}.}
\item
  Learning and Testing was presented as it was supposed to be and
\item
  data was stored correctly
\end{enumerate}

Let's see now if data makes sense.

\hypertarget{select-the-version-of-the-experiment}{%
\subsection{Select the version of the
experiment}\label{select-the-version-of-the-experiment}}

Select the version of the experiment you want:

\begin{itemize}
\item
  Version 14 has 80 subjects, label picture task has 2500ms as timeout
\item
  Version 15 has 42 subjects, label picture task has 3500ms as timeout
\end{itemize}

\begin{Shaded}
\begin{Highlighting}[]
\NormalTok{ver2 <-}\StringTok{ }\KeywordTok{c}\NormalTok{(}\DecValTok{14}\NormalTok{)}
\end{Highlighting}
\end{Shaded}

\hypertarget{reaction-times}{%
\subsection{Reaction times}\label{reaction-times}}

\begin{Shaded}
\begin{Highlighting}[]
\KeywordTok{rbind}\NormalTok{(generalizationPL, generalizationLP, contingencyJudgement)->}\StringTok{ }\NormalTok{alltasks}
\NormalTok{alltasks <-}\StringTok{ }\KeywordTok{droplevels}\NormalTok{(alltasks)}
\end{Highlighting}
\end{Shaded}

\begin{Shaded}
\begin{Highlighting}[]
\KeywordTok{gghistogram}\NormalTok{(alltasks,}
       \DataTypeTok{x =} \StringTok{"rt"}\NormalTok{,}
       \DataTypeTok{y =} \StringTok{"..count.."}\NormalTok{,}
       \DataTypeTok{xlab =} \StringTok{"rt"}\NormalTok{, }
       \DataTypeTok{color =} \StringTok{"task"}\NormalTok{, }
       \DataTypeTok{fill =} \StringTok{"task"}\NormalTok{,}
       \DataTypeTok{bins =} \DecValTok{40}\NormalTok{,}
       \DataTypeTok{palette =} \StringTok{"jco"}\NormalTok{,}
       \DataTypeTok{add =} \StringTok{"median"}
\NormalTok{)}
\end{Highlighting}
\end{Shaded}

\begin{verbatim}
## Warning: Removed 934 rows containing non-finite values (stat_bin).
\end{verbatim}

\includegraphics{preProcessing_files/figure-latex/RT hist-1.pdf}

The two generalization tasks looks quite different. I'm going to plot it
separately for a better inspection:

\begin{Shaded}
\begin{Highlighting}[]
\NormalTok{p<-}\StringTok{ }\KeywordTok{gghistogram}\NormalTok{(alltasks, }\CommentTok{#will throw warnings related to non responses but that's okay, ggplot simply removes them}
       \DataTypeTok{x =} \StringTok{"rt"}\NormalTok{,}
       \DataTypeTok{y =} \StringTok{"..count.."}\NormalTok{,}
       \DataTypeTok{xlab =} \StringTok{"rt"}\NormalTok{,}
       \DataTypeTok{facet.by =} \StringTok{"task"}\NormalTok{,}
       \DataTypeTok{add =} \StringTok{"median"}\NormalTok{,}
       \DataTypeTok{bins =} \DecValTok{40}
\NormalTok{)}

\KeywordTok{facet}\NormalTok{(p, }\DataTypeTok{facet.by =} \StringTok{"task"}\NormalTok{,}
      \DataTypeTok{nrow =} \DecValTok{3}\NormalTok{,}
      \DataTypeTok{ncol =} \DecValTok{1}\NormalTok{)}
\end{Highlighting}
\end{Shaded}

\begin{verbatim}
## Warning: Removed 934 rows containing non-finite values (stat_bin).
\end{verbatim}

\begin{center}\includegraphics{preProcessing_files/figure-latex/histogram separated by generalization tasks-1} \end{center}

The tails of the first two tasks don't end smoothly, especially in task
2.

\hypertarget{accuracy}{%
\subsection{accuracy}\label{accuracy}}

\hypertarget{randomdot}{%
\subsubsection{RandomDot}\label{randomdot}}

\begin{Shaded}
\begin{Highlighting}[]
\KeywordTok{unique}\NormalTok{(randomDot}\OperatorTok{$}\NormalTok{subjID)->}\StringTok{ }\NormalTok{subj;}
\NormalTok{randomDot->}\StringTok{ }\NormalTok{randomTask}

\NormalTok{trials <-}\StringTok{ }\KeywordTok{c}\NormalTok{(}\KeywordTok{rep}\NormalTok{(}\StringTok{'0'}\NormalTok{, }\DecValTok{6}\NormalTok{), }\KeywordTok{rep}\NormalTok{(}\StringTok{'1'}\NormalTok{, }\DecValTok{5}\NormalTok{), }
              \KeywordTok{rep}\NormalTok{(}\StringTok{'2'}\NormalTok{, }\DecValTok{5}\NormalTok{), }\KeywordTok{rep}\NormalTok{(}\StringTok{'3'}\NormalTok{, }\DecValTok{5}\NormalTok{), }
              \KeywordTok{rep}\NormalTok{(}\StringTok{'4'}\NormalTok{, }\DecValTok{5}\NormalTok{))}

\NormalTok{trialstot <-}\StringTok{ }\KeywordTok{as.factor}\NormalTok{(}\KeywordTok{rep}\NormalTok{(trials, }\KeywordTok{length}\NormalTok{(subj)))}

\NormalTok{randomTask}\OperatorTok{$}\NormalTok{blocks <-}\StringTok{ }\NormalTok{trialstot}
\end{Highlighting}
\end{Shaded}

\hypertarget{how-many-timeouts-by-participant}{%
\paragraph{How many timeouts by
participant?}\label{how-many-timeouts-by-participant}}

\begin{Shaded}
\begin{Highlighting}[]
\NormalTok{randomTask}\OperatorTok{$}\NormalTok{timeout <-}\StringTok{ }\KeywordTok{ifelse}\NormalTok{(randomTask}\OperatorTok{$}\NormalTok{resp}\OperatorTok{==}\StringTok{ }\DecValTok{-1}\NormalTok{, }\DecValTok{1}\NormalTok{, }\DecValTok{0}\NormalTok{)}
\end{Highlighting}
\end{Shaded}

\begin{Shaded}
\begin{Highlighting}[]
\NormalTok{temp<-randomTask }\OperatorTok
\StringTok{  }\KeywordTok{count}\NormalTok{(timeout, subjID) }\OperatorTok
\StringTok{  }\KeywordTok{filter}\NormalTok{(timeout }\OperatorTok{==}\StringTok{ }\DecValTok{1}\NormalTok{)}

\KeywordTok{unique}\NormalTok{(temp}\OperatorTok{$}\NormalTok{subjID)->}\StringTok{ }\NormalTok{subjs}

\NormalTok{temp2<-randomTask[}\OperatorTok{!}\NormalTok{(randomTask}\OperatorTok{$}\NormalTok{subjID }\OperatorTok\StringTok{ }\NormalTok{subjs),] }\OperatorTok
\StringTok{  }\KeywordTok{count}\NormalTok{(timeout, subjID,  ) }\OperatorTok
\StringTok{  }\KeywordTok{filter}\NormalTok{(timeout }\OperatorTok{==}\StringTok{ }\DecValTok{0}\NormalTok{)}

\NormalTok{temp2[temp2}\OperatorTok{$}\NormalTok{timeout}\OperatorTok{==}\DecValTok{0}\NormalTok{,]}\OperatorTok{$}\NormalTok{n <-}\StringTok{ }\DecValTok{0}

\KeywordTok{rbind}\NormalTok{(temp,temp2)->}\StringTok{ }\NormalTok{timeout}
\end{Highlighting}
\end{Shaded}

Histrogram by participant:

\begin{Shaded}
\begin{Highlighting}[]
\KeywordTok{hist}\NormalTok{(timeout}\OperatorTok{$}\NormalTok{n, }\DataTypeTok{xlab =} \StringTok{'number of timeouts'}\NormalTok{, }
     \DataTypeTok{main =} \StringTok{''}\NormalTok{, }
     \DataTypeTok{col=}\KeywordTok{grey}\NormalTok{(.}\DecValTok{80}\NormalTok{), }
     \DataTypeTok{border=}\KeywordTok{grey}\NormalTok{(}\DecValTok{0}\NormalTok{),}
     \DataTypeTok{breaks =} \KeywordTok{seq}\NormalTok{(}\DecValTok{0}\NormalTok{,}\KeywordTok{max}\NormalTok{(timeout}\OperatorTok{$}\NormalTok{n),}\DecValTok{1}\NormalTok{))}
\end{Highlighting}
\end{Shaded}

\begin{center}\includegraphics{preProcessing_files/figure-latex/unnamed-chunk-34-1} \end{center}

\begin{Shaded}
\begin{Highlighting}[]
\NormalTok{timeout <-}\StringTok{ }\NormalTok{randomTask }\OperatorTok
\StringTok{  }\KeywordTok{group_by}\NormalTok{(subjID, blocks) }\OperatorTok
\StringTok{  }\KeywordTok{filter}\NormalTok{(resp }\OperatorTok{==}\StringTok{ }\DecValTok{-1}\NormalTok{) }\OperatorTok
\StringTok{  }\KeywordTok{count}\NormalTok{() }


\KeywordTok{ggbarplot}\NormalTok{(timeout[timeout}\OperatorTok{$}\NormalTok{n}\OperatorTok{>}\DecValTok{1}\NormalTok{ ,], }\DataTypeTok{x =} \StringTok{"blocks"}\NormalTok{, }\DataTypeTok{y =} \StringTok{"n"}\NormalTok{,}
          \DataTypeTok{facet.by =} \StringTok{"subjID"}\NormalTok{,}
          \DataTypeTok{sort.by.groups =} \OtherTok{TRUE}\NormalTok{,     }\CommentTok{# Sort inside each group}
          \DataTypeTok{ylab =} \StringTok{"num of timeouts"}\NormalTok{)}
\end{Highlighting}
\end{Shaded}

\includegraphics{preProcessing_files/figure-latex/unnamed-chunk-35-1.pdf}
\#\#\#\# Subjects that made more than 3 timeouts

\begin{Shaded}
\begin{Highlighting}[]
\KeywordTok{unique}\NormalTok{(timeout[timeout}\OperatorTok{$}\NormalTok{n}\OperatorTok{>}\DecValTok{3}\NormalTok{,]}\OperatorTok{$}\NormalTok{subjID) ->}\StringTok{ }\NormalTok{problematicPeople}
\end{Highlighting}
\end{Shaded}

\begin{Shaded}
\begin{Highlighting}[]
\NormalTok{accdistr <-}\StringTok{ }\NormalTok{randomTask[}\OperatorTok{!}\NormalTok{(randomTask}\OperatorTok{$}\NormalTok{resp }\OperatorTok{==}\StringTok{ }\DecValTok{-1}\NormalTok{),] }\OperatorTok
\StringTok{  }\KeywordTok{group_by}\NormalTok{(subjID, blocks,  ) }\OperatorTok
\StringTok{  }\KeywordTok{summarise}\NormalTok{(}\DataTypeTok{m =} \KeywordTok{mean}\NormalTok{(acc))}
\end{Highlighting}
\end{Shaded}

\begin{Shaded}
\begin{Highlighting}[]
\KeywordTok{ggstripchart}\NormalTok{(accdistr, }\DataTypeTok{x =} \StringTok{"blocks"}\NormalTok{, }\DataTypeTok{y =} \StringTok{"m"}\NormalTok{,}
             \DataTypeTok{xlab =} \StringTok{"blocks"}\NormalTok{,}
             \DataTypeTok{ylab =} \StringTok{"accuracy"}\NormalTok{,}
             \DataTypeTok{add =} \StringTok{"mean_ci"}\NormalTok{,}
             \DataTypeTok{size =} \DecValTok{2}\NormalTok{,}
             \DataTypeTok{color =} \StringTok{"darkgray"}\NormalTok{,}
             \DataTypeTok{shape =} \DecValTok{21}\NormalTok{,}
             \DataTypeTok{fill =} \StringTok{"gray"}\NormalTok{,}
             \DataTypeTok{error.plot =} \StringTok{"pointrange"}\NormalTok{,}
             \DataTypeTok{add.params =} \KeywordTok{list}\NormalTok{(}\DataTypeTok{color =} \StringTok{"black"}\NormalTok{,}
                               \DataTypeTok{size =} \FloatTok{0.7}\NormalTok{)) }\OperatorTok{+}
\StringTok{  }\KeywordTok{scale_y_continuous}\NormalTok{(}\DataTypeTok{limits =} \KeywordTok{c}\NormalTok{(}\FloatTok{0.1}\NormalTok{, }\DecValTok{1}\NormalTok{), }\DataTypeTok{oob =}\NormalTok{ scales}\OperatorTok{::}\NormalTok{squish) }\OperatorTok{+}\StringTok{ }\CommentTok{#to prevent jitter to move above 100%}
\StringTok{  }\KeywordTok{geom_hline}\NormalTok{(}\DataTypeTok{yintercept =} \FloatTok{.50}\NormalTok{, }\DataTypeTok{col=}\StringTok{'red'}\NormalTok{, }\DataTypeTok{lwd=}\DecValTok{1}\NormalTok{);}
\end{Highlighting}
\end{Shaded}

\includegraphics{preProcessing_files/figure-latex/unnamed-chunk-38-1.pdf}

\begin{Shaded}
\begin{Highlighting}[]
\NormalTok{accdistr <-}\StringTok{ }\NormalTok{randomTask[}\OperatorTok{!}\NormalTok{(randomTask}\OperatorTok{$}\NormalTok{resp }\OperatorTok{==}\StringTok{ }\DecValTok{-1}\NormalTok{),] }\OperatorTok
\StringTok{  }\KeywordTok{group_by}\NormalTok{(subjID, blocks) }\OperatorTok
\StringTok{  }\KeywordTok{summarise}\NormalTok{(}\DataTypeTok{m =} \KeywordTok{mean}\NormalTok{(acc))}

\NormalTok{accdistr[accdistr}\OperatorTok{$}\NormalTok{m}\OperatorTok{<=}\NormalTok{.}\DecValTok{5}\NormalTok{,]}
\end{Highlighting}
\end{Shaded}

\begin{verbatim}
## # A tibble: 21 x 3
## # Groups:   subjID [11]
##     subjID blocks     m
##      <int> <fct>  <dbl>
##  1 1414932 4       0.25
##  2 1422470 1       0.4 
##  3 1422475 2       0.5 
##  4 1422475 3       0.2 
##  5 1422475 4       0   
##  6 1422689 3       0   
##  7 1422689 4       0.4 
##  8 1431942 4       0.4 
##  9 1459001 3       0.5 
## 10 1459001 4       0.2 
## # ... with 11 more rows
\end{verbatim}

\hypertarget{people-that-scored-less-than-70}{%
\paragraph{People that scored less than
70\%:}\label{people-that-scored-less-than-70}}

\begin{Shaded}
\begin{Highlighting}[]
\KeywordTok{unique}\NormalTok{(accdistr[accdistr}\OperatorTok{$}\NormalTok{m}\OperatorTok{<}\NormalTok{.}\DecValTok{7}\NormalTok{,]}\OperatorTok{$}\NormalTok{subjID) ->}\StringTok{ }\NormalTok{dumbPeople}
\end{Highlighting}
\end{Shaded}

\begin{Shaded}
\begin{Highlighting}[]
\KeywordTok{setdiff}\NormalTok{(dumbPeople, problematicPeople)->}\StringTok{ }\NormalTok{dumbPeople}
\end{Highlighting}
\end{Shaded}

Let's consider them as bad subjects.

\begin{Shaded}
\begin{Highlighting}[]
\KeywordTok{c}\NormalTok{(problematicPeople, dumbPeople)->badsubjs}
\end{Highlighting}
\end{Shaded}

\begin{Shaded}
\begin{Highlighting}[]
\KeywordTok{rm}\NormalTok{(temp, temp2, timeout, subj, subjs, trials, trialstot, accdistr)}
\end{Highlighting}
\end{Shaded}

\hypertarget{task-1-from-picture-to-labels}{%
\subsubsection{Task 1: from picture to
labels}\label{task-1-from-picture-to-labels}}

The column fribbleID stores the fribble presented, while the column
label stores the labels presented. Resp column in this task refers to
the label selected. Category and frequency refers to the fribbleID
column.

I'm going to add 1 in the accuracy column for every instance where
response matches the category column, i.e., the participant correctly
associated the fribble to its label.

I remove the no-response, and compute accuracy based on category and
response.

\hypertarget{how-many-participants-do-we-have-per-learning}{%
\paragraph{How many participants do we have per
learning?}\label{how-many-participants-do-we-have-per-learning}}

\begin{Shaded}
\begin{Highlighting}[]
\KeywordTok{length}\NormalTok{(}\KeywordTok{unique}\NormalTok{(generalizationPL}\OperatorTok{$}\NormalTok{subjID))}
\end{Highlighting}
\end{Shaded}

\begin{verbatim}
## [1] 120
\end{verbatim}

\begin{Shaded}
\begin{Highlighting}[]
\NormalTok{fl<-}\StringTok{ }\KeywordTok{length}\NormalTok{(}\KeywordTok{unique}\NormalTok{(generalizationPL[generalizationPL}\OperatorTok{$}\NormalTok{learning}\OperatorTok{==}\StringTok{'FL'} \OperatorTok{&}\StringTok{ }
\StringTok{                                      }\OperatorTok{!}\NormalTok{(generalizationPL}\OperatorTok{$}\NormalTok{subjID }\OperatorTok\StringTok{ }\NormalTok{badsubjs),]}\OperatorTok{$}\NormalTok{subjID))}
\NormalTok{lf<-}\StringTok{ }\KeywordTok{length}\NormalTok{(}\KeywordTok{unique}\NormalTok{(generalizationPL[generalizationPL}\OperatorTok{$}\NormalTok{learning}\OperatorTok{==}\StringTok{'LF'} \OperatorTok{&}
\StringTok{                                      }\OperatorTok{!}\NormalTok{(generalizationPL}\OperatorTok{$}\NormalTok{subjID }\OperatorTok\StringTok{ }\NormalTok{badsubjs),]}\OperatorTok{$}\NormalTok{subjID))}
\NormalTok{fl}
\end{Highlighting}
\end{Shaded}

\begin{verbatim}
## [1] 55
\end{verbatim}

\begin{Shaded}
\begin{Highlighting}[]
\NormalTok{lf}
\end{Highlighting}
\end{Shaded}

\begin{verbatim}
## [1] 47
\end{verbatim}

We have 55 for feature-label learning, and 47 for label-feature
learning.

\hypertarget{check-tails-of-the-rt-distribution}{%
\paragraph{Check tails of the rt
distribution}\label{check-tails-of-the-rt-distribution}}

The point is that we can't rely on responses made very early, because
these might be simply mistakes or technical errors.

\begin{Shaded}
\begin{Highlighting}[]
\KeywordTok{par}\NormalTok{(}\DataTypeTok{mfrow=}\KeywordTok{c}\NormalTok{(}\DecValTok{1}\NormalTok{,}\DecValTok{2}\NormalTok{))}
\KeywordTok{hist}\NormalTok{(generalizationPL[generalizationPL}\OperatorTok{$}\NormalTok{rt}\OperatorTok{<}\DecValTok{600} \OperatorTok{&}\StringTok{ }
\StringTok{                        }\OperatorTok{!}\NormalTok{(generalizationPL}\OperatorTok{$}\NormalTok{subjID }\OperatorTok\StringTok{ }\NormalTok{badsubjs),]}\OperatorTok{$}\NormalTok{rt, }\DataTypeTok{main =} \StringTok{'rt < 600ms'}\NormalTok{, }\DataTypeTok{xlab =} \StringTok{'trials'}\NormalTok{);}

\KeywordTok{hist}\NormalTok{(generalizationPL[generalizationPL}\OperatorTok{$}\NormalTok{rt}\OperatorTok{>}\DecValTok{2000} \OperatorTok{&}\StringTok{ }
\StringTok{                        }\OperatorTok{!}\NormalTok{(generalizationPL}\OperatorTok{$}\NormalTok{subjID }\OperatorTok\StringTok{ }\NormalTok{badsubjs),]}\OperatorTok{$}\NormalTok{rt, }\DataTypeTok{main =} \StringTok{'rt > 2000ms'}\NormalTok{, }\DataTypeTok{xlab =} \StringTok{'trials'}\NormalTok{);}
\end{Highlighting}
\end{Shaded}

\includegraphics{preProcessing_files/figure-latex/hist of rt-1.pdf}

\begin{Shaded}
\begin{Highlighting}[]
\KeywordTok{par}\NormalTok{(}\DataTypeTok{mfrow=}\KeywordTok{c}\NormalTok{(}\DecValTok{1}\NormalTok{,}\DecValTok{1}\NormalTok{))}
\end{Highlighting}
\end{Shaded}

I would remove rt \textless100ms for all tasks.

\hypertarget{how-many-what-type-of-trials-do-we-have}{%
\paragraph{How many, what type of trials do we
have?}\label{how-many-what-type-of-trials-do-we-have}}

\begin{Shaded}
\begin{Highlighting}[]
\KeywordTok{round}\NormalTok{(}\KeywordTok{nrow}\NormalTok{(generalizationPL[generalizationPL}\OperatorTok{$}\NormalTok{rt}\OperatorTok{<}\DecValTok{100} \OperatorTok{&}\StringTok{ }
\StringTok{                              }\OperatorTok{!}\NormalTok{(generalizationPL}\OperatorTok{$}\NormalTok{subjID }\OperatorTok\StringTok{ }\NormalTok{badsubjs),]) }\OperatorTok{/}\StringTok{ }\KeywordTok{nrow}\NormalTok{(generalizationPL[}\OperatorTok{!}\NormalTok{(generalizationPL}\OperatorTok{$}\NormalTok{subjID }\OperatorTok\StringTok{ }\NormalTok{badsubjs),])}\OperatorTok{*}\DecValTok{100}\NormalTok{,}\DecValTok{2}\NormalTok{)}
\end{Highlighting}
\end{Shaded}

\begin{verbatim}
## [1] 6.74
\end{verbatim}

\begin{Shaded}
\begin{Highlighting}[]
\KeywordTok{rm}\NormalTok{(fl,lf)}
\NormalTok{pictureLabel <-}\StringTok{ }\NormalTok{generalizationPL[}\OperatorTok{!}\NormalTok{(}\KeywordTok{is.na}\NormalTok{(generalizationPL}\OperatorTok{$}\NormalTok{resp)),]}

\NormalTok{pictureLabel}\OperatorTok{$}\NormalTok{acc <-}\StringTok{ }\DecValTok{0}\NormalTok{;}
\NormalTok{pictureLabel[pictureLabel}\OperatorTok{$}\NormalTok{category}\OperatorTok{==}\DecValTok{1} \OperatorTok{&}\StringTok{ }\NormalTok{pictureLabel}\OperatorTok{$}\NormalTok{resp}\OperatorTok{==}\StringTok{'dep'}\NormalTok{,]}\OperatorTok{$}\NormalTok{acc <-}\StringTok{ }\DecValTok{1}\NormalTok{;}

\NormalTok{pictureLabel[pictureLabel}\OperatorTok{$}\NormalTok{category}\OperatorTok{==}\DecValTok{2} \OperatorTok{&}\StringTok{ }\NormalTok{pictureLabel}\OperatorTok{$}\NormalTok{resp}\OperatorTok{==}\StringTok{'bim'}\NormalTok{,]}\OperatorTok{$}\NormalTok{acc <-}\StringTok{ }\DecValTok{1}\NormalTok{;}

\NormalTok{pictureLabel[pictureLabel}\OperatorTok{$}\NormalTok{category}\OperatorTok{==}\DecValTok{3} \OperatorTok{&}\StringTok{ }\NormalTok{pictureLabel}\OperatorTok{$}\NormalTok{resp}\OperatorTok{==}\StringTok{'tob'}\NormalTok{,]}\OperatorTok{$}\NormalTok{acc <-}\StringTok{ }\DecValTok{1}\NormalTok{;}
\end{Highlighting}
\end{Shaded}

\begin{Shaded}
\begin{Highlighting}[]
\NormalTok{n <-}\StringTok{ }\KeywordTok{length}\NormalTok{(}\KeywordTok{unique}\NormalTok{(pictureLabel[}\OperatorTok{!}\NormalTok{(pictureLabel}\OperatorTok{$}\NormalTok{subjID }\OperatorTok\StringTok{ }\NormalTok{badsubjs),]}\OperatorTok{$}\NormalTok{subjID))}

\NormalTok{nrows <-}\StringTok{ }\NormalTok{(}\KeywordTok{nrow}\NormalTok{(generalizationPL[}\OperatorTok{!}\NormalTok{(generalizationPL}\OperatorTok{$}\NormalTok{subjID }\OperatorTok\StringTok{ }\NormalTok{badsubjs),])) }\OperatorTok{-}\StringTok{ }\NormalTok{(}\KeywordTok{nrow}\NormalTok{(pictureLabel[}\OperatorTok{!}\NormalTok{(pictureLabel}\OperatorTok{$}\NormalTok{subjID }\OperatorTok\StringTok{ }\NormalTok{badsubjs),]))}
\end{Highlighting}
\end{Shaded}

\begin{Shaded}
\begin{Highlighting}[]
\KeywordTok{sort}\NormalTok{(}\KeywordTok{unique}\NormalTok{(pictureLabel[}\OperatorTok{!}\NormalTok{(pictureLabel}\OperatorTok{$}\NormalTok{subjID }\OperatorTok\StringTok{ }\NormalTok{badsubjs),]}\OperatorTok{$}\NormalTok{subjID))->}\StringTok{ }\NormalTok{subjs;}
\KeywordTok{sort}\NormalTok{(}\KeywordTok{unique}\NormalTok{(generalizationPL[}\OperatorTok{!}\NormalTok{(generalizationPL}\OperatorTok{$}\NormalTok{subjID }\OperatorTok\StringTok{ }\NormalTok{badsubjs),]}\OperatorTok{$}\NormalTok{subjID)) ->totsubjs;}

\NormalTok{subjmissed<-}\StringTok{ }\KeywordTok{setdiff}\NormalTok{(totsubjs, subjs);}

\KeywordTok{rm}\NormalTok{(subjs, totsubjs);}
\end{Highlighting}
\end{Shaded}

We have 101 participants in this task, this is -1 compared to our total
number of participants. The subject(s) that didn't answer at all the
task is: 1420171. We have lost also 145 responses, that is 5.0347222
over the total: 2880.

How many trials per participant do we have now?

\begin{Shaded}
\begin{Highlighting}[]
\NormalTok{pictureLabel }\OperatorTok
\StringTok{  }\KeywordTok{group_by}\NormalTok{(subjID) }\OperatorTok
\StringTok{  }\KeywordTok{count}\NormalTok{() }\OperatorTok\StringTok{ }\KeywordTok{filter}\NormalTok{(n}\OperatorTok{<=}\DecValTok{18}\NormalTok{)}
\end{Highlighting}
\end{Shaded}

\begin{verbatim}
## # A tibble: 2 x 2
## # Groups:   subjID [2]
##    subjID     n
##     <int> <int>
## 1 1422475    18
## 2 1432075    18
\end{verbatim}

No one had less than 18 trials, over the total (24). That's fine!

\hypertarget{barplot-accuracy-by-category-frequency-learning}{%
\paragraph{Barplot accuracy by category + frequency +
learning}\label{barplot-accuracy-by-category-frequency-learning}}

picture label

\begin{Shaded}
\begin{Highlighting}[]
\KeywordTok{c}\NormalTok{(badsubjs, subjmissed) ->}\StringTok{ }\NormalTok{badsubjs}

\KeywordTok{rm}\NormalTok{(n, subjmissed, nrows)}

\NormalTok{ss_prop<-}\KeywordTok{aggregate}\NormalTok{(acc }\OperatorTok{~}\StringTok{ }\NormalTok{frequency}\OperatorTok{+}\NormalTok{category}\OperatorTok{+}\NormalTok{subjID}\OperatorTok{+}\NormalTok{learning, }
                   \DataTypeTok{data =}\NormalTok{ pictureLabel[pictureLabel}\OperatorTok{$}\NormalTok{rt }\OperatorTok{>}\StringTok{ }\DecValTok{100} \OperatorTok{&}
\StringTok{                                       }\OperatorTok{!}\NormalTok{(pictureLabel}\OperatorTok{$}\NormalTok{subjID }\OperatorTok\StringTok{ }\NormalTok{badsubjs),], }\DataTypeTok{FUN =}\NormalTok{ mean)}
\end{Highlighting}
\end{Shaded}

Plot aggregated over subjs. To see accuracy distributed over categories.

\begin{Shaded}
\begin{Highlighting}[]
\NormalTok{ms <-}\StringTok{ }\NormalTok{ss_prop }\OperatorTok
\StringTok{  }\KeywordTok{group_by}\NormalTok{( category, frequency, learning) }\OperatorTok
\StringTok{  }\KeywordTok{summarise}\NormalTok{(}\DataTypeTok{n=}\KeywordTok{n}\NormalTok{(),}
    \DataTypeTok{mean=}\KeywordTok{mean}\NormalTok{(acc),}
    \DataTypeTok{sd=}\KeywordTok{sd}\NormalTok{(acc)}
\NormalTok{  ) }\OperatorTok
\StringTok{  }\KeywordTok{mutate}\NormalTok{( }\DataTypeTok{se=}\NormalTok{sd}\OperatorTok{/}\KeywordTok{sqrt}\NormalTok{(n))  }\OperatorTok\StringTok{ }
\StringTok{  }\KeywordTok{mutate}\NormalTok{( }\DataTypeTok{ci=}\NormalTok{se }\OperatorTok{*}\StringTok{ }\KeywordTok{qt}\NormalTok{((}\DecValTok{1}\FloatTok{-0.05}\NormalTok{)}\OperatorTok{/}\DecValTok{2} \OperatorTok{+}\StringTok{ }\FloatTok{.5}\NormalTok{, n}\DecValTok{-1}\NormalTok{))}

\NormalTok{ms}\OperatorTok{$}\NormalTok{frequency <-}\StringTok{ }\KeywordTok{as.factor}\NormalTok{(ms}\OperatorTok{$}\NormalTok{frequency)}
\NormalTok{plyr}\OperatorTok{::}\KeywordTok{revalue}\NormalTok{(ms}\OperatorTok{$}\NormalTok{frequency, }\KeywordTok{c}\NormalTok{(}\StringTok{"25"}\NormalTok{=}\StringTok{"low"}\NormalTok{))->}\StringTok{ }\NormalTok{ms}\OperatorTok{$}\NormalTok{frequency;}
\NormalTok{plyr}\OperatorTok{::}\KeywordTok{revalue}\NormalTok{(ms}\OperatorTok{$}\NormalTok{frequency, }\KeywordTok{c}\NormalTok{(}\StringTok{"75"}\NormalTok{=}\StringTok{"high"}\NormalTok{))->}\StringTok{ }\NormalTok{ms}\OperatorTok{$}\NormalTok{frequency;}

\KeywordTok{ggplot}\NormalTok{(}\KeywordTok{aes}\NormalTok{(}\DataTypeTok{x =}\NormalTok{ category, }\DataTypeTok{y =}\NormalTok{ mean, }\DataTypeTok{fill =}\NormalTok{ frequency), }\DataTypeTok{data =}\NormalTok{ ms) }\OperatorTok{+}
\StringTok{  }\KeywordTok{facet_grid}\NormalTok{( . }\OperatorTok{~}\StringTok{ }\NormalTok{learning) }\OperatorTok{+}\StringTok{ }
\StringTok{  }\KeywordTok{geom_bar}\NormalTok{(}\DataTypeTok{stat =} \StringTok{"identity"}\NormalTok{, }\DataTypeTok{color=}\StringTok{'white'}\NormalTok{, }\DataTypeTok{position=}\KeywordTok{position_dodge}\NormalTok{(), }\DataTypeTok{size=}\FloatTok{1.2}\NormalTok{) }\OperatorTok{+}
\StringTok{  }\KeywordTok{geom_errorbar}\NormalTok{(}\KeywordTok{aes}\NormalTok{(}\DataTypeTok{ymin=}\NormalTok{mean}\OperatorTok{-}\NormalTok{se, }\DataTypeTok{ymax=}\NormalTok{mean}\OperatorTok{+}\NormalTok{se), }\DataTypeTok{width=}\NormalTok{.}\DecValTok{15}\NormalTok{, }\DataTypeTok{size=}\DecValTok{1}\NormalTok{,}\DataTypeTok{position=}\KeywordTok{position_dodge}\NormalTok{(.}\DecValTok{9}\NormalTok{)) }\OperatorTok{+}
\StringTok{  }\KeywordTok{ylab}\NormalTok{(}\StringTok{"Accuracy "}\NormalTok{) }\OperatorTok{+}
\StringTok{  }\KeywordTok{xlab}\NormalTok{(}\StringTok{"category"}\NormalTok{) }\OperatorTok{+}
\StringTok{  }\KeywordTok{ggtitle}\NormalTok{(}\StringTok{'pictureLabels'}\NormalTok{) }\OperatorTok{+}
\StringTok{  }\KeywordTok{coord_cartesian}\NormalTok{(}\DataTypeTok{ylim =} \KeywordTok{c}\NormalTok{(}\DecValTok{0}\NormalTok{, }\DecValTok{1}\NormalTok{))}\OperatorTok{+}
\StringTok{  }\NormalTok{ggpubr}\OperatorTok{::}\KeywordTok{theme_pubclean}\NormalTok{() }\OperatorTok{+}\StringTok{ }
\StringTok{  }\KeywordTok{theme}\NormalTok{(}\DataTypeTok{legend.position=}\StringTok{"bottom"}\NormalTok{, }\DataTypeTok{legend.title =} \KeywordTok{element_blank}\NormalTok{()) }\OperatorTok{+}
\StringTok{  }\KeywordTok{theme}\NormalTok{(}\DataTypeTok{text =} \KeywordTok{element_text}\NormalTok{(}\DataTypeTok{size=}\DecValTok{10}\NormalTok{)) }\OperatorTok{+}
\StringTok{  }\KeywordTok{geom_hline}\NormalTok{(}\DataTypeTok{yintercept =} \FloatTok{.33}\NormalTok{, }\DataTypeTok{col=}\StringTok{'red'}\NormalTok{, }\DataTypeTok{lwd=}\DecValTok{1}\NormalTok{);}
\end{Highlighting}
\end{Shaded}

\includegraphics{preProcessing_files/figure-latex/unnamed-chunk-46-1.pdf}

\hypertarget{violin-plot-accuracy-by-category-frequency-learning}{%
\paragraph{Violin plot accuracy by category + frequency +
learning}\label{violin-plot-accuracy-by-category-frequency-learning}}

\begin{Shaded}
\begin{Highlighting}[]
\NormalTok{df <-}\StringTok{ }\KeywordTok{aggregate}\NormalTok{(acc }\OperatorTok{~}\StringTok{ }\NormalTok{subjID}\OperatorTok{+}\NormalTok{frequency}\OperatorTok{+}\NormalTok{learning}\OperatorTok{+}\NormalTok{category, }
                \DataTypeTok{data =}\NormalTok{ pictureLabel[pictureLabel}\OperatorTok{$}\NormalTok{rt }\OperatorTok{>}\StringTok{ }\DecValTok{100}  \OperatorTok{&}
\StringTok{                                       }\OperatorTok{!}\NormalTok{(pictureLabel}\OperatorTok{$}\NormalTok{subjID }\OperatorTok\StringTok{ }\NormalTok{badsubjs),], mean)}
\NormalTok{df}\OperatorTok{$}\NormalTok{frequency <-}\StringTok{ }\KeywordTok{as.factor}\NormalTok{(df}\OperatorTok{$}\NormalTok{frequency)}
\NormalTok{plyr}\OperatorTok{::}\KeywordTok{revalue}\NormalTok{(df}\OperatorTok{$}\NormalTok{frequency, }\KeywordTok{c}\NormalTok{(}\StringTok{"25"}\NormalTok{=}\StringTok{"low"}\NormalTok{))->}\StringTok{ }\NormalTok{df}\OperatorTok{$}\NormalTok{frequency;}
\NormalTok{plyr}\OperatorTok{::}\KeywordTok{revalue}\NormalTok{(df}\OperatorTok{$}\NormalTok{frequency, }\KeywordTok{c}\NormalTok{(}\StringTok{"75"}\NormalTok{=}\StringTok{"high"}\NormalTok{))->}\StringTok{ }\NormalTok{df}\OperatorTok{$}\NormalTok{frequency;}

\KeywordTok{ggviolin}\NormalTok{(df, }\DataTypeTok{x =} \StringTok{"frequency"}\NormalTok{, }\DataTypeTok{y =} \StringTok{"acc"}\NormalTok{, }\DataTypeTok{fill =} \StringTok{"frequency"}\NormalTok{,}
         \DataTypeTok{palette =} \KeywordTok{c}\NormalTok{(}\StringTok{"#00AFBB"}\NormalTok{, }\StringTok{"#E7B800"}\NormalTok{),}
         \DataTypeTok{add =} \StringTok{"boxplot"}\NormalTok{, }
         \DataTypeTok{add.params =} \KeywordTok{list}\NormalTok{(}\DataTypeTok{fill =} \StringTok{"white"}\NormalTok{),}
         \DataTypeTok{trim=}\OtherTok{TRUE}\NormalTok{) }\OperatorTok{+}
\StringTok{        }\KeywordTok{ggtitle}\NormalTok{(}\StringTok{'pictureLabels'}\NormalTok{) }\OperatorTok{+}
\StringTok{        }\KeywordTok{facet_grid}\NormalTok{( learning }\OperatorTok{~}\StringTok{ }\NormalTok{category) }\OperatorTok{+}
\StringTok{        }\KeywordTok{theme_pubclean}\NormalTok{()}\OperatorTok{+}
\StringTok{  }\KeywordTok{geom_hline}\NormalTok{(}\DataTypeTok{yintercept =} \FloatTok{.33}\NormalTok{, }\DataTypeTok{col=}\StringTok{'red'}\NormalTok{, }\DataTypeTok{lwd=}\DecValTok{1}\NormalTok{);}
\end{Highlighting}
\end{Shaded}

\begin{center}\includegraphics{preProcessing_files/figure-latex/unnamed-chunk-47-1} \end{center}

\hypertarget{violin-plot-accuracy-by-frequency-learning}{%
\paragraph{Violin plot accuracy by frequency +
learning}\label{violin-plot-accuracy-by-frequency-learning}}

Let's see how participants scored for the high/low frequency:

\begin{Shaded}
\begin{Highlighting}[]
\NormalTok{df <-}\StringTok{ }\KeywordTok{aggregate}\NormalTok{(acc }\OperatorTok{~}\StringTok{ }\NormalTok{subjID}\OperatorTok{+}\NormalTok{frequency}\OperatorTok{+}\NormalTok{learning, }
                \DataTypeTok{data =}\NormalTok{ pictureLabel[pictureLabel}\OperatorTok{$}\NormalTok{rt }\OperatorTok{>}\StringTok{ }\DecValTok{100}  \OperatorTok{&}
\StringTok{                                       }\OperatorTok{!}\NormalTok{(pictureLabel}\OperatorTok{$}\NormalTok{subjID }\OperatorTok\StringTok{ }\NormalTok{badsubjs),], mean)}
\NormalTok{df}\OperatorTok{$}\NormalTok{frequency <-}\StringTok{ }\KeywordTok{as.factor}\NormalTok{(df}\OperatorTok{$}\NormalTok{frequency)}
\NormalTok{plyr}\OperatorTok{::}\KeywordTok{revalue}\NormalTok{(df}\OperatorTok{$}\NormalTok{frequency, }\KeywordTok{c}\NormalTok{(}\StringTok{"25"}\NormalTok{=}\StringTok{"low"}\NormalTok{))->}\StringTok{ }\NormalTok{df}\OperatorTok{$}\NormalTok{frequency;}
\NormalTok{plyr}\OperatorTok{::}\KeywordTok{revalue}\NormalTok{(df}\OperatorTok{$}\NormalTok{frequency, }\KeywordTok{c}\NormalTok{(}\StringTok{"75"}\NormalTok{=}\StringTok{"high"}\NormalTok{))->}\StringTok{ }\NormalTok{df}\OperatorTok{$}\NormalTok{frequency;}

\KeywordTok{ggviolin}\NormalTok{(df, }\DataTypeTok{x =} \StringTok{"frequency"}\NormalTok{, }\DataTypeTok{y =} \StringTok{"acc"}\NormalTok{, }\DataTypeTok{fill =} \StringTok{"frequency"}\NormalTok{,}
         \DataTypeTok{palette =} \KeywordTok{c}\NormalTok{(}\StringTok{"#00AFBB"}\NormalTok{, }\StringTok{"#E7B800"}\NormalTok{),}
         \DataTypeTok{add =} \StringTok{"boxplot"}\NormalTok{, }
         \DataTypeTok{add.params =} \KeywordTok{list}\NormalTok{(}\DataTypeTok{fill =} \StringTok{"white"}\NormalTok{),}
         \DataTypeTok{trim=}\OtherTok{TRUE}\NormalTok{) }\OperatorTok{+}
\StringTok{        }\KeywordTok{ggtitle}\NormalTok{(}\StringTok{'pictureLabels'}\NormalTok{) }\OperatorTok{+}
\StringTok{        }\KeywordTok{facet_grid}\NormalTok{( . }\OperatorTok{~}\StringTok{ }\NormalTok{learning) }\OperatorTok{+}
\StringTok{        }\KeywordTok{theme_pubclean}\NormalTok{()}\OperatorTok{+}
\StringTok{  }\KeywordTok{geom_hline}\NormalTok{(}\DataTypeTok{yintercept =} \FloatTok{.33}\NormalTok{, }\DataTypeTok{col=}\StringTok{'red'}\NormalTok{, }\DataTypeTok{lwd=}\DecValTok{1}\NormalTok{);}
\end{Highlighting}
\end{Shaded}

\includegraphics{preProcessing_files/figure-latex/unnamed-chunk-48-1.pdf}

\begin{Shaded}
\begin{Highlighting}[]
\NormalTok{df }\OperatorTok
\StringTok{  }\KeywordTok{group_by}\NormalTok{(learning, frequency) }\OperatorTok
\StringTok{  }\KeywordTok{summarise}\NormalTok{(}\KeywordTok{mean}\NormalTok{(acc))}
\end{Highlighting}
\end{Shaded}

\begin{verbatim}
## # A tibble: 4 x 3
## # Groups:   learning [2]
##   learning frequency `mean(acc)`
##   <fct>    <fct>           <dbl>
## 1 FL       low             0.474
## 2 FL       high            0.721
## 3 LF       low             0.376
## 4 LF       high            0.656
\end{verbatim}

Closer inspection:

\begin{Shaded}
\begin{Highlighting}[]
\KeywordTok{par}\NormalTok{(}\DataTypeTok{mfrow=}\KeywordTok{c}\NormalTok{(}\DecValTok{2}\NormalTok{,}\DecValTok{2}\NormalTok{))}
\KeywordTok{hist}\NormalTok{(df[df}\OperatorTok{$}\NormalTok{frequency}\OperatorTok{==}\StringTok{'low'} \OperatorTok{&}\StringTok{ }\NormalTok{df}\OperatorTok{$}\NormalTok{learning}\OperatorTok{==}\StringTok{'FL'}\NormalTok{ ,]}\OperatorTok{$}\NormalTok{acc, }\DataTypeTok{xlab =} \StringTok{'acc'}\NormalTok{, }\DataTypeTok{main =} \StringTok{'low freq - FL '}\NormalTok{)}
\KeywordTok{hist}\NormalTok{(df[df}\OperatorTok{$}\NormalTok{frequency}\OperatorTok{==}\StringTok{'low'} \OperatorTok{&}\StringTok{ }\NormalTok{df}\OperatorTok{$}\NormalTok{learning}\OperatorTok{==}\StringTok{'LF'}\NormalTok{,]}\OperatorTok{$}\NormalTok{acc, }\DataTypeTok{xlab =} \StringTok{'acc'}\NormalTok{, }\DataTypeTok{main =} \StringTok{'low freq - LF '}\NormalTok{)}
\KeywordTok{hist}\NormalTok{(df[df}\OperatorTok{$}\NormalTok{frequency}\OperatorTok{==}\StringTok{'high'} \OperatorTok{&}\StringTok{ }\NormalTok{df}\OperatorTok{$}\NormalTok{learning}\OperatorTok{==}\StringTok{'FL'}\NormalTok{,]}\OperatorTok{$}\NormalTok{acc, }\DataTypeTok{xlab =} \StringTok{'acc'}\NormalTok{, }\DataTypeTok{main =} \StringTok{'high freq - FL '}\NormalTok{)}
\KeywordTok{hist}\NormalTok{(df[df}\OperatorTok{$}\NormalTok{frequency}\OperatorTok{==}\StringTok{'high'} \OperatorTok{&}\StringTok{ }\NormalTok{df}\OperatorTok{$}\NormalTok{learning}\OperatorTok{==}\StringTok{'LF'}\NormalTok{,]}\OperatorTok{$}\NormalTok{acc, }\DataTypeTok{xlab =} \StringTok{'acc'}\NormalTok{, }\DataTypeTok{main =} \StringTok{'high freq - LF '}\NormalTok{)}
\end{Highlighting}
\end{Shaded}

\includegraphics{preProcessing_files/figure-latex/unnamed-chunk-50-1.pdf}

\begin{Shaded}
\begin{Highlighting}[]
\KeywordTok{par}\NormalTok{(}\DataTypeTok{mfrow=}\KeywordTok{c}\NormalTok{(}\DecValTok{1}\NormalTok{,}\DecValTok{1}\NormalTok{))}
\end{Highlighting}
\end{Shaded}

\hypertarget{barplot-accuracy-by-frequency-learning}{%
\paragraph{Barplot accuracy by frequency +
learning}\label{barplot-accuracy-by-frequency-learning}}

\begin{Shaded}
\begin{Highlighting}[]
\CommentTok{#barPlot aggregated over categories:}

\NormalTok{ms <-}\StringTok{ }\KeywordTok{aggregate}\NormalTok{(acc }\OperatorTok{~}\StringTok{ }\NormalTok{subjID}\OperatorTok{+}\NormalTok{frequency}\OperatorTok{+}\NormalTok{learning, }
                \DataTypeTok{data=}\NormalTok{pictureLabel[pictureLabel}\OperatorTok{$}\NormalTok{rt }\OperatorTok{>}\StringTok{ }\DecValTok{100} \OperatorTok{&}
\StringTok{                                  }\OperatorTok{!}\NormalTok{(pictureLabel}\OperatorTok{$}\NormalTok{subjID }\OperatorTok\StringTok{ }\NormalTok{badsubjs)  ,], }\DataTypeTok{FUN=}\NormalTok{ mean)}

\NormalTok{df<-}\StringTok{ }\NormalTok{ms }\OperatorTok
\StringTok{  }\KeywordTok{group_by}\NormalTok{(frequency, learning)}\OperatorTok
\StringTok{  }\KeywordTok{summarise}\NormalTok{(}
    \DataTypeTok{mean =} \KeywordTok{mean}\NormalTok{(acc),}
    \DataTypeTok{sd =} \KeywordTok{sd}\NormalTok{(acc),}
    \DataTypeTok{n =} \KeywordTok{n}\NormalTok{()) }\OperatorTok
\StringTok{  }\KeywordTok{mutate}\NormalTok{( }\DataTypeTok{se=}\NormalTok{sd}\OperatorTok{/}\KeywordTok{sqrt}\NormalTok{(n))  }\OperatorTok\StringTok{ }
\StringTok{  }\KeywordTok{mutate}\NormalTok{( }\DataTypeTok{ci=}\NormalTok{se }\OperatorTok{*}\StringTok{ }\KeywordTok{qt}\NormalTok{((}\DecValTok{1}\FloatTok{-0.05}\NormalTok{)}\OperatorTok{/}\DecValTok{2} \OperatorTok{+}\StringTok{ }\FloatTok{.5}\NormalTok{, n}\DecValTok{-1}\NormalTok{))}

\NormalTok{df}\OperatorTok{$}\NormalTok{frequency <-}\StringTok{ }\KeywordTok{as.factor}\NormalTok{(df}\OperatorTok{$}\NormalTok{frequency)}
\NormalTok{plyr}\OperatorTok{::}\KeywordTok{revalue}\NormalTok{(df}\OperatorTok{$}\NormalTok{frequency, }\KeywordTok{c}\NormalTok{(}\StringTok{"25"}\NormalTok{=}\StringTok{"low"}\NormalTok{))->}\StringTok{ }\NormalTok{df}\OperatorTok{$}\NormalTok{frequency;}
\NormalTok{plyr}\OperatorTok{::}\KeywordTok{revalue}\NormalTok{(df}\OperatorTok{$}\NormalTok{frequency, }\KeywordTok{c}\NormalTok{(}\StringTok{"75"}\NormalTok{=}\StringTok{"high"}\NormalTok{))->}\StringTok{ }\NormalTok{df}\OperatorTok{$}\NormalTok{frequency;}


\NormalTok{pl<-}\KeywordTok{ggplot}\NormalTok{(}\KeywordTok{aes}\NormalTok{(}\DataTypeTok{x =}\NormalTok{ frequency, }\DataTypeTok{y =}\NormalTok{ mean, }\DataTypeTok{fill =}\NormalTok{ frequency), }\DataTypeTok{data =}\NormalTok{ df) }\OperatorTok{+}
\StringTok{  }\KeywordTok{facet_grid}\NormalTok{( . }\OperatorTok{~}\StringTok{ }\NormalTok{learning) }\OperatorTok{+}
\StringTok{  }\KeywordTok{geom_bar}\NormalTok{(}\DataTypeTok{stat =} \StringTok{"identity"}\NormalTok{, }\DataTypeTok{color=}\StringTok{'white'}\NormalTok{, }\DataTypeTok{position=}\KeywordTok{position_dodge}\NormalTok{(), }\DataTypeTok{size=}\FloatTok{1.2}\NormalTok{) }\OperatorTok{+}
\StringTok{  }\KeywordTok{geom_errorbar}\NormalTok{(}\KeywordTok{aes}\NormalTok{(}\DataTypeTok{ymin=}\NormalTok{mean}\OperatorTok{-}\NormalTok{se, }\DataTypeTok{ymax=}\NormalTok{mean}\OperatorTok{+}\NormalTok{se), }\DataTypeTok{width=}\NormalTok{.}\DecValTok{15}\NormalTok{, }\DataTypeTok{size=}\DecValTok{1}\NormalTok{,}\DataTypeTok{position=}\KeywordTok{position_dodge}\NormalTok{(.}\DecValTok{9}\NormalTok{)) }\OperatorTok{+}
\StringTok{  }\KeywordTok{ylab}\NormalTok{(}\StringTok{"Accuracy "}\NormalTok{) }\OperatorTok{+}
\StringTok{  }\KeywordTok{xlab}\NormalTok{(}\StringTok{"frequency"}\NormalTok{) }\OperatorTok{+}
\StringTok{  }\KeywordTok{ggtitle}\NormalTok{(}\StringTok{'pictureLabels'}\NormalTok{) }\OperatorTok{+}
\StringTok{  }\KeywordTok{coord_cartesian}\NormalTok{(}\DataTypeTok{ylim =} \KeywordTok{c}\NormalTok{(}\DecValTok{0}\NormalTok{, }\DecValTok{1}\NormalTok{))}\OperatorTok{+}
\StringTok{  }\NormalTok{ggpubr}\OperatorTok{::}\KeywordTok{theme_pubclean}\NormalTok{() }\OperatorTok{+}\StringTok{ }
\StringTok{  }\KeywordTok{theme}\NormalTok{(}\DataTypeTok{legend.position =} \StringTok{"none"}\NormalTok{) }\OperatorTok{+}
\StringTok{  }\KeywordTok{theme}\NormalTok{(}\DataTypeTok{text =} \KeywordTok{element_text}\NormalTok{(}\DataTypeTok{size=}\DecValTok{10}\NormalTok{)) }\OperatorTok{+}
\StringTok{  }\KeywordTok{geom_hline}\NormalTok{(}\DataTypeTok{yintercept =} \FloatTok{.33}\NormalTok{, }\DataTypeTok{col=}\StringTok{'red'}\NormalTok{, }\DataTypeTok{lwd=}\DecValTok{1}\NormalTok{);}
\end{Highlighting}
\end{Shaded}

\hypertarget{task-2-from-label-to-pictures}{%
\subsubsection{Task 2: from label to
pictures}\label{task-2-from-label-to-pictures}}

Let's check now the generalizaton from label to pictures:

\begin{Shaded}
\begin{Highlighting}[]
\KeywordTok{length}\NormalTok{(}\KeywordTok{unique}\NormalTok{(generalizationLP}\OperatorTok{$}\NormalTok{subjID))}
\end{Highlighting}
\end{Shaded}

\begin{verbatim}
## [1] 120
\end{verbatim}

\begin{Shaded}
\begin{Highlighting}[]
\NormalTok{fl<-}\StringTok{ }\KeywordTok{length}\NormalTok{(}\KeywordTok{unique}\NormalTok{(generalizationLP[generalizationLP}\OperatorTok{$}\NormalTok{learning}\OperatorTok{==}\StringTok{'FL'} \OperatorTok{&}\StringTok{ }
\StringTok{                                      }\OperatorTok{!}\NormalTok{(generalizationLP}\OperatorTok{$}\NormalTok{subjID }\OperatorTok\StringTok{ }\NormalTok{badsubjs),]}\OperatorTok{$}\NormalTok{subjID))}

\NormalTok{lf<-}\StringTok{ }\KeywordTok{length}\NormalTok{(}\KeywordTok{unique}\NormalTok{(generalizationLP[generalizationLP}\OperatorTok{$}\NormalTok{learning}\OperatorTok{==}\StringTok{'LF'} \OperatorTok{&}
\StringTok{                                      }\OperatorTok{!}\NormalTok{(generalizationLP}\OperatorTok{$}\NormalTok{subjID }\OperatorTok\StringTok{ }\NormalTok{badsubjs),]}\OperatorTok{$}\NormalTok{subjID))}
\NormalTok{fl}
\end{Highlighting}
\end{Shaded}

\begin{verbatim}
## [1] 55
\end{verbatim}

\begin{Shaded}
\begin{Highlighting}[]
\NormalTok{lf}
\end{Highlighting}
\end{Shaded}

\begin{verbatim}
## [1] 46
\end{verbatim}

\hypertarget{how-many-participants-do-we-have-per-learning-1}{%
\paragraph{How many participants do we have per
learning?}\label{how-many-participants-do-we-have-per-learning-1}}

We have 55 for feature-label learning, and 46 for label-feature
learning.

\begin{Shaded}
\begin{Highlighting}[]
\KeywordTok{rm}\NormalTok{(fl,lf)}
\NormalTok{labelPicture <-}\StringTok{ }\NormalTok{generalizationLP[}\OperatorTok{!}\NormalTok{(}\KeywordTok{is.na}\NormalTok{(generalizationLP}\OperatorTok{$}\NormalTok{resp)),]}
\NormalTok{n<-}\StringTok{ }\KeywordTok{length}\NormalTok{(}\KeywordTok{unique}\NormalTok{(labelPicture}\OperatorTok{$}\NormalTok{subjID))}
\NormalTok{nrows <-}\StringTok{ }\NormalTok{(}\KeywordTok{nrow}\NormalTok{(generalizationLP)) }\OperatorTok{-}\StringTok{ }\NormalTok{(}\KeywordTok{nrow}\NormalTok{(labelPicture))}

\KeywordTok{sort}\NormalTok{(}\KeywordTok{unique}\NormalTok{(labelPicture}\OperatorTok{$}\NormalTok{subjID))->}\StringTok{ }\NormalTok{subjs;}
\KeywordTok{sort}\NormalTok{(}\KeywordTok{unique}\NormalTok{(generalizationLP}\OperatorTok{$}\NormalTok{subjID)) ->totsubjs;}

\NormalTok{subjmissed<-}\StringTok{ }\KeywordTok{setdiff}\NormalTok{(totsubjs, subjs);}
\end{Highlighting}
\end{Shaded}

Great, we have 120 participants in this task, so -0, and we have missed
195 over the total 2880, that is 6.7708333. The subject(s) that missed
completely the task is: .

\hypertarget{how-many-what-type-of-trials-do-we-have-1}{%
\paragraph{How many, what type of trials do we
have?}\label{how-many-what-type-of-trials-do-we-have-1}}

How many datapoints did we lose for no-responses?

\begin{Shaded}
\begin{Highlighting}[]
\KeywordTok{round}\NormalTok{(}\KeywordTok{nrow}\NormalTok{(generalizationLP[(}\KeywordTok{is.na}\NormalTok{(generalizationLP}\OperatorTok{$}\NormalTok{resp)) }\OperatorTok{&}
\StringTok{                              }\OperatorTok{!}\NormalTok{(generalizationLP}\OperatorTok{$}\NormalTok{subjID }\OperatorTok\StringTok{ }\NormalTok{badsubjs),]) }\OperatorTok{/}\KeywordTok{nrow}\NormalTok{(generalizationLP[}\OperatorTok{!}\NormalTok{(generalizationLP}\OperatorTok{$}\NormalTok{subjID }\OperatorTok\StringTok{ }\NormalTok{badsubjs),])}\OperatorTok{*}\DecValTok{100}\NormalTok{,}\DecValTok{2}\NormalTok{)}
\end{Highlighting}
\end{Shaded}

\begin{verbatim}
## [1] 6.77
\end{verbatim}

How many trials were rt \textless{} 100?

\begin{Shaded}
\begin{Highlighting}[]
\KeywordTok{round}\NormalTok{(}\KeywordTok{nrow}\NormalTok{(generalizationLP[generalizationLP}\OperatorTok{$}\NormalTok{rt}\OperatorTok{<}\DecValTok{100} \OperatorTok{&}\StringTok{ }
\StringTok{                              }\OperatorTok{!}\NormalTok{(generalizationLP}\OperatorTok{$}\NormalTok{subjID }\OperatorTok\StringTok{ }\NormalTok{badsubjs),])}\OperatorTok{/}\StringTok{ }\KeywordTok{nrow}\NormalTok{(generalizationLP[}\OperatorTok{!}\NormalTok{(generalizationLP}\OperatorTok{$}\NormalTok{subjID }\OperatorTok\StringTok{ }\NormalTok{badsubjs),])}\OperatorTok{*}\DecValTok{100}\NormalTok{,}\DecValTok{2}\NormalTok{)}
\end{Highlighting}
\end{Shaded}

\begin{verbatim}
## [1] 6.81
\end{verbatim}

Once trimmed, how many trials per participant do we have in this task?

\begin{Shaded}
\begin{Highlighting}[]
\NormalTok{labelPicture }\OperatorTok
\StringTok{  }\KeywordTok{group_by}\NormalTok{(subjID) }\OperatorTok
\StringTok{  }\KeywordTok{count}\NormalTok{() }\OperatorTok
\StringTok{  }\KeywordTok{filter}\NormalTok{(n}\OperatorTok{<=}\DecValTok{18}\NormalTok{)}
\end{Highlighting}
\end{Shaded}

\begin{verbatim}
## # A tibble: 8 x 2
## # Groups:   subjID [8]
##    subjID     n
##     <int> <int>
## 1 1420577    18
## 2 1422475    18
## 3 1422477    17
## 4 1422677    17
## 5 1422680     9
## 6 1422689    17
## 7 1432009     8
## 8 1432075    17
\end{verbatim}

Here we have less datapoints. For sure, 1422680 needs to be added to the
black list because has few correct trials.

\begin{Shaded}
\begin{Highlighting}[]
\KeywordTok{c}\NormalTok{(badsubjs, }\DecValTok{1422680}\NormalTok{, }\DecValTok{1432009}\NormalTok{) ->}\StringTok{ }\NormalTok{badsubjs}
\end{Highlighting}
\end{Shaded}

\hypertarget{check-tails-of-the-rt-distribution-1}{%
\paragraph{Check tails of the rt
distribution}\label{check-tails-of-the-rt-distribution-1}}

\begin{Shaded}
\begin{Highlighting}[]
\KeywordTok{par}\NormalTok{(}\DataTypeTok{mfrow=}\KeywordTok{c}\NormalTok{(}\DecValTok{1}\NormalTok{,}\DecValTok{2}\NormalTok{))}
\KeywordTok{hist}\NormalTok{(generalizationLP[generalizationLP}\OperatorTok{$}\NormalTok{rt}\OperatorTok{<}\DecValTok{600} \OperatorTok{&}\StringTok{ }
\StringTok{                        }\OperatorTok{!}\NormalTok{(generalizationLP}\OperatorTok{$}\NormalTok{subjID }\OperatorTok\StringTok{ }\NormalTok{badsubjs),]}\OperatorTok{$}\NormalTok{rt, }\DataTypeTok{main =} \StringTok{'rt < 600ms'}\NormalTok{, }\DataTypeTok{xlab =} \StringTok{'trials'}\NormalTok{);}
\KeywordTok{hist}\NormalTok{(generalizationLP[generalizationLP}\OperatorTok{$}\NormalTok{rt}\OperatorTok{>}\DecValTok{2000} \OperatorTok{&}\StringTok{ }
\StringTok{                        }\OperatorTok{!}\NormalTok{(generalizationLP}\OperatorTok{$}\NormalTok{subjID }\OperatorTok\StringTok{ }\NormalTok{badsubjs),]}\OperatorTok{$}\NormalTok{rt, }\DataTypeTok{main =} \StringTok{'rt > 2000ms'}\NormalTok{, }\DataTypeTok{xlab =} \StringTok{'trials'}\NormalTok{);}
\end{Highlighting}
\end{Shaded}

\begin{center}\includegraphics{preProcessing_files/figure-latex/hist rt tails-1} \end{center}

\begin{Shaded}
\begin{Highlighting}[]
\KeywordTok{par}\NormalTok{(}\DataTypeTok{mfrow=}\KeywordTok{c}\NormalTok{(}\DecValTok{1}\NormalTok{,}\DecValTok{1}\NormalTok{))}
\end{Highlighting}
\end{Shaded}

\begin{Shaded}
\begin{Highlighting}[]
\KeywordTok{rm}\NormalTok{(n, nrows, subjs, totsubjs);}
\NormalTok{labelPicture}\OperatorTok{$}\NormalTok{acc <-}\StringTok{ }\DecValTok{0}\NormalTok{;}
\NormalTok{labelPicture[labelPicture}\OperatorTok{$}\NormalTok{category}\OperatorTok{==}\DecValTok{1} \OperatorTok{&}\StringTok{ }\NormalTok{labelPicture}\OperatorTok{$}\NormalTok{label}\OperatorTok{==}\StringTok{'dep'}\NormalTok{,]}\OperatorTok{$}\NormalTok{acc <-}\StringTok{ }\DecValTok{1}\NormalTok{;}
\NormalTok{labelPicture[labelPicture}\OperatorTok{$}\NormalTok{category}\OperatorTok{==}\DecValTok{2} \OperatorTok{&}\StringTok{ }\NormalTok{labelPicture}\OperatorTok{$}\NormalTok{label}\OperatorTok{==}\StringTok{'bim'}\NormalTok{,]}\OperatorTok{$}\NormalTok{acc <-}\StringTok{ }\DecValTok{1}\NormalTok{;}
\NormalTok{labelPicture[labelPicture}\OperatorTok{$}\NormalTok{category}\OperatorTok{==}\DecValTok{3} \OperatorTok{&}\StringTok{ }\NormalTok{labelPicture}\OperatorTok{$}\NormalTok{label}\OperatorTok{==}\StringTok{'tob'}\NormalTok{,]}\OperatorTok{$}\NormalTok{acc <-}\StringTok{ }\DecValTok{1}\NormalTok{;}
\end{Highlighting}
\end{Shaded}

\hypertarget{barplot-accuracy-by-categorylearningfrequency}{%
\paragraph{Barplot accuracy by
category+learning+frequency}\label{barplot-accuracy-by-categorylearningfrequency}}

Calculate the proportion of correct in each condition

\begin{Shaded}
\begin{Highlighting}[]
\KeywordTok{rm}\NormalTok{(subjmissed)}
\NormalTok{ss_prop<-}\KeywordTok{aggregate}\NormalTok{(acc }\OperatorTok{~}\StringTok{ }\NormalTok{frequency}\OperatorTok{+}\NormalTok{category}\OperatorTok{+}\NormalTok{subjID}\OperatorTok{+}\NormalTok{learning, }
                   \DataTypeTok{data =}\NormalTok{ labelPicture[labelPicture}\OperatorTok{$}\NormalTok{rt }\OperatorTok{>}\StringTok{ }\DecValTok{100} \OperatorTok{&}
\StringTok{                                         }\NormalTok{labelPicture}\OperatorTok{$}\NormalTok{rt }\OperatorTok{<=}\StringTok{ }\DecValTok{2500} \OperatorTok{&}
\StringTok{                                         }\OperatorTok{!}\NormalTok{(labelPicture}\OperatorTok{$}\NormalTok{subjID }\OperatorTok\StringTok{ }\NormalTok{badsubjs),], }\DataTypeTok{FUN =}\NormalTok{ mean)}
\end{Highlighting}
\end{Shaded}

Plot aggregated over subjs. To see accuracy distributed over categories.

\begin{Shaded}
\begin{Highlighting}[]
\NormalTok{ms <-}\StringTok{ }\NormalTok{ss_prop }\OperatorTok
\StringTok{  }\KeywordTok{group_by}\NormalTok{(category, frequency, learning) }\OperatorTok
\StringTok{  }\KeywordTok{summarise}\NormalTok{(}
    \DataTypeTok{n=}\KeywordTok{n}\NormalTok{(),}
    \DataTypeTok{mean=}\KeywordTok{mean}\NormalTok{(acc),}
    \DataTypeTok{sd=}\KeywordTok{sd}\NormalTok{(acc)}
\NormalTok{  ) }\OperatorTok
\StringTok{  }\KeywordTok{mutate}\NormalTok{( }\DataTypeTok{se=}\NormalTok{sd}\OperatorTok{/}\KeywordTok{sqrt}\NormalTok{(n))  }\OperatorTok\StringTok{ }
\StringTok{  }\KeywordTok{mutate}\NormalTok{( }\DataTypeTok{ci=}\NormalTok{se }\OperatorTok{*}\StringTok{ }\KeywordTok{qt}\NormalTok{((}\DecValTok{1}\FloatTok{-0.05}\NormalTok{)}\OperatorTok{/}\DecValTok{2} \OperatorTok{+}\StringTok{ }\FloatTok{.5}\NormalTok{, n}\DecValTok{-1}\NormalTok{))}

\NormalTok{ms}\OperatorTok{$}\NormalTok{frequency <-}\StringTok{ }\KeywordTok{as.factor}\NormalTok{(ms}\OperatorTok{$}\NormalTok{frequency)}
\NormalTok{plyr}\OperatorTok{::}\KeywordTok{revalue}\NormalTok{(ms}\OperatorTok{$}\NormalTok{frequency, }\KeywordTok{c}\NormalTok{(}\StringTok{"25"}\NormalTok{=}\StringTok{"low"}\NormalTok{))->}\StringTok{ }\NormalTok{ms}\OperatorTok{$}\NormalTok{frequency;}
\NormalTok{plyr}\OperatorTok{::}\KeywordTok{revalue}\NormalTok{(ms}\OperatorTok{$}\NormalTok{frequency, }\KeywordTok{c}\NormalTok{(}\StringTok{"75"}\NormalTok{=}\StringTok{"high"}\NormalTok{))->}\StringTok{ }\NormalTok{ms}\OperatorTok{$}\NormalTok{frequency;}

\KeywordTok{ggplot}\NormalTok{(}\KeywordTok{aes}\NormalTok{(}\DataTypeTok{x =}\NormalTok{ category, }\DataTypeTok{y =}\NormalTok{ mean, }\DataTypeTok{fill =}\NormalTok{ frequency), }\DataTypeTok{data =}\NormalTok{ ms) }\OperatorTok{+}
\StringTok{  }\KeywordTok{facet_grid}\NormalTok{( . }\OperatorTok{~}\StringTok{ }\NormalTok{learning) }\OperatorTok{+}\StringTok{ }
\StringTok{  }\KeywordTok{geom_bar}\NormalTok{(}\DataTypeTok{stat =} \StringTok{"identity"}\NormalTok{, }\DataTypeTok{color=}\StringTok{'white'}\NormalTok{, }\DataTypeTok{position=}\KeywordTok{position_dodge}\NormalTok{(), }\DataTypeTok{size=}\FloatTok{1.2}\NormalTok{) }\OperatorTok{+}
\StringTok{  }\KeywordTok{geom_errorbar}\NormalTok{(}\KeywordTok{aes}\NormalTok{(}\DataTypeTok{ymin=}\NormalTok{mean}\OperatorTok{-}\NormalTok{se, }\DataTypeTok{ymax=}\NormalTok{mean}\OperatorTok{+}\NormalTok{se), }\DataTypeTok{width=}\NormalTok{.}\DecValTok{15}\NormalTok{, }\DataTypeTok{size=}\DecValTok{1}\NormalTok{,}\DataTypeTok{position=}\KeywordTok{position_dodge}\NormalTok{(.}\DecValTok{9}\NormalTok{)) }\OperatorTok{+}
\StringTok{  }\KeywordTok{ylab}\NormalTok{(}\StringTok{"Accuracy "}\NormalTok{) }\OperatorTok{+}
\StringTok{  }\KeywordTok{xlab}\NormalTok{(}\StringTok{"category"}\NormalTok{) }\OperatorTok{+}
\StringTok{  }\KeywordTok{ggtitle}\NormalTok{(}\StringTok{'labelPictures'}\NormalTok{) }\OperatorTok{+}
\StringTok{  }\KeywordTok{coord_cartesian}\NormalTok{(}\DataTypeTok{ylim =} \KeywordTok{c}\NormalTok{(}\DecValTok{0}\NormalTok{, }\DecValTok{1}\NormalTok{))}\OperatorTok{+}
\StringTok{  }\NormalTok{ggpubr}\OperatorTok{::}\KeywordTok{theme_pubclean}\NormalTok{() }\OperatorTok{+}\StringTok{ }
\StringTok{  }\KeywordTok{theme}\NormalTok{(}\DataTypeTok{legend.position=}\StringTok{"bottom"}\NormalTok{, }\DataTypeTok{legend.title =} \KeywordTok{element_blank}\NormalTok{()) }\OperatorTok{+}
\StringTok{  }\KeywordTok{theme}\NormalTok{(}\DataTypeTok{text =} \KeywordTok{element_text}\NormalTok{(}\DataTypeTok{size=}\DecValTok{10}\NormalTok{)) }\OperatorTok{+}
\StringTok{  }\KeywordTok{geom_hline}\NormalTok{(}\DataTypeTok{yintercept =} \FloatTok{.33}\NormalTok{, }\DataTypeTok{col=}\StringTok{'red'}\NormalTok{, }\DataTypeTok{lwd=}\DecValTok{1}\NormalTok{);}
\end{Highlighting}
\end{Shaded}

\includegraphics{preProcessing_files/figure-latex/unnamed-chunk-59-1.pdf}

\hypertarget{violin-plot-accuracy-by-categorylearningfrequency}{%
\paragraph{Violin plot accuracy by
category+learning+frequency}\label{violin-plot-accuracy-by-categorylearningfrequency}}

\begin{Shaded}
\begin{Highlighting}[]
\NormalTok{ms <-}\StringTok{ }\KeywordTok{aggregate}\NormalTok{(acc }\OperatorTok{~}\StringTok{ }\NormalTok{subjID}\OperatorTok{+}\NormalTok{frequency}\OperatorTok{+}\NormalTok{learning}\OperatorTok{+}\NormalTok{category, }
                \DataTypeTok{data =}\NormalTok{ labelPicture[labelPicture}\OperatorTok{$}\NormalTok{rt }\OperatorTok{>}\StringTok{ }\DecValTok{100} \OperatorTok{&}\StringTok{ }
\StringTok{                                      }\NormalTok{labelPicture}\OperatorTok{$}\NormalTok{rt }\OperatorTok{<=}\DecValTok{2500} \OperatorTok{&}
\StringTok{                                         }\OperatorTok{!}\NormalTok{(labelPicture}\OperatorTok{$}\NormalTok{subjID }\OperatorTok\StringTok{ }\NormalTok{badsubjs),], mean)}

\NormalTok{ms}\OperatorTok{$}\NormalTok{frequency <-}\StringTok{ }\KeywordTok{as.factor}\NormalTok{(ms}\OperatorTok{$}\NormalTok{frequency)}
\NormalTok{plyr}\OperatorTok{::}\KeywordTok{revalue}\NormalTok{(ms}\OperatorTok{$}\NormalTok{frequency, }\KeywordTok{c}\NormalTok{(}\StringTok{"25"}\NormalTok{=}\StringTok{"low"}\NormalTok{))->}\StringTok{ }\NormalTok{ms}\OperatorTok{$}\NormalTok{frequency;}
\NormalTok{plyr}\OperatorTok{::}\KeywordTok{revalue}\NormalTok{(ms}\OperatorTok{$}\NormalTok{frequency, }\KeywordTok{c}\NormalTok{(}\StringTok{"75"}\NormalTok{=}\StringTok{"high"}\NormalTok{))->}\StringTok{ }\NormalTok{ms}\OperatorTok{$}\NormalTok{frequency;}

\KeywordTok{ggviolin}\NormalTok{(ms, }\DataTypeTok{x =} \StringTok{"frequency"}\NormalTok{, }\DataTypeTok{y =} \StringTok{"acc"}\NormalTok{, }\DataTypeTok{fill =} \StringTok{"frequency"}\NormalTok{,}
         \DataTypeTok{palette =} \KeywordTok{c}\NormalTok{(}\StringTok{"#00AFBB"}\NormalTok{, }\StringTok{"#E7B800"}\NormalTok{),}
         \DataTypeTok{add =} \StringTok{"boxplot"}\NormalTok{, }
         \DataTypeTok{add.params =} \KeywordTok{list}\NormalTok{(}\DataTypeTok{fill =} \StringTok{"white"}\NormalTok{),}
         \DataTypeTok{trim=}\OtherTok{TRUE}\NormalTok{) }\OperatorTok{+}
\StringTok{         }\KeywordTok{ggtitle}\NormalTok{(}\StringTok{'labelPictures'}\NormalTok{) }\OperatorTok{+}
\StringTok{        }\KeywordTok{facet_grid}\NormalTok{( learning }\OperatorTok{~}\StringTok{ }\NormalTok{category) }\OperatorTok{+}
\StringTok{        }\KeywordTok{theme_pubclean}\NormalTok{()}\OperatorTok{+}
\StringTok{  }\KeywordTok{geom_hline}\NormalTok{(}\DataTypeTok{yintercept =} \FloatTok{.33}\NormalTok{, }\DataTypeTok{col=}\StringTok{'red'}\NormalTok{, }\DataTypeTok{lwd=}\DecValTok{1}\NormalTok{);}
\end{Highlighting}
\end{Shaded}

\begin{center}\includegraphics{preProcessing_files/figure-latex/unnamed-chunk-60-1} \end{center}

\begin{Shaded}
\begin{Highlighting}[]
\CommentTok{#rm(ms, ss_prop)}
\end{Highlighting}
\end{Shaded}

\hypertarget{violinplot-accuracy-by-learningfrequency}{%
\paragraph{Violinplot accuracy by
learning+frequency}\label{violinplot-accuracy-by-learningfrequency}}

\begin{Shaded}
\begin{Highlighting}[]
\NormalTok{ms <-}\StringTok{ }\KeywordTok{aggregate}\NormalTok{(acc }\OperatorTok{~}\StringTok{ }\NormalTok{subjID}\OperatorTok{+}\NormalTok{frequency}\OperatorTok{+}\NormalTok{learning, }
                \DataTypeTok{data =}\NormalTok{ labelPicture[labelPicture}\OperatorTok{$}\NormalTok{rt }\OperatorTok{>}\StringTok{ }\DecValTok{100} \OperatorTok{&}
\StringTok{                                      }\NormalTok{labelPicture}\OperatorTok{$}\NormalTok{rt }\OperatorTok{<=}\StringTok{ }\DecValTok{2500} \OperatorTok{&}
\StringTok{                                         }\OperatorTok{!}\NormalTok{(labelPicture}\OperatorTok{$}\NormalTok{subjID }\OperatorTok\StringTok{ }\NormalTok{badsubjs),], mean)}

\NormalTok{ms}\OperatorTok{$}\NormalTok{frequency <-}\StringTok{ }\KeywordTok{as.factor}\NormalTok{(ms}\OperatorTok{$}\NormalTok{frequency)}
\NormalTok{plyr}\OperatorTok{::}\KeywordTok{revalue}\NormalTok{(ms}\OperatorTok{$}\NormalTok{frequency, }\KeywordTok{c}\NormalTok{(}\StringTok{"25"}\NormalTok{=}\StringTok{"low"}\NormalTok{))->}\StringTok{ }\NormalTok{ms}\OperatorTok{$}\NormalTok{frequency;}
\NormalTok{plyr}\OperatorTok{::}\KeywordTok{revalue}\NormalTok{(ms}\OperatorTok{$}\NormalTok{frequency, }\KeywordTok{c}\NormalTok{(}\StringTok{"75"}\NormalTok{=}\StringTok{"high"}\NormalTok{))->}\StringTok{ }\NormalTok{ms}\OperatorTok{$}\NormalTok{frequency;}

\KeywordTok{ggviolin}\NormalTok{(ms, }\DataTypeTok{x =} \StringTok{"frequency"}\NormalTok{, }\DataTypeTok{y =} \StringTok{"acc"}\NormalTok{, }\DataTypeTok{fill =} \StringTok{"frequency"}\NormalTok{,}
         \DataTypeTok{palette =} \KeywordTok{c}\NormalTok{(}\StringTok{"#00AFBB"}\NormalTok{, }\StringTok{"#E7B800"}\NormalTok{),}
         \DataTypeTok{add =} \StringTok{"boxplot"}\NormalTok{, }
         \DataTypeTok{add.params =} \KeywordTok{list}\NormalTok{(}\DataTypeTok{fill =} \StringTok{"white"}\NormalTok{),}
         \DataTypeTok{trim=}\OtherTok{TRUE}\NormalTok{) }\OperatorTok{+}
\StringTok{         }\KeywordTok{ggtitle}\NormalTok{(}\StringTok{'labelPictures'}\NormalTok{) }\OperatorTok{+}
\StringTok{        }\KeywordTok{facet_grid}\NormalTok{( . }\OperatorTok{~}\StringTok{ }\NormalTok{learning) }\OperatorTok{+}
\StringTok{        }\KeywordTok{theme_pubclean}\NormalTok{()}\OperatorTok{+}
\StringTok{  }\KeywordTok{geom_hline}\NormalTok{(}\DataTypeTok{yintercept =} \FloatTok{.33}\NormalTok{, }\DataTypeTok{col=}\StringTok{'red'}\NormalTok{, }\DataTypeTok{lwd=}\DecValTok{1}\NormalTok{);}
\end{Highlighting}
\end{Shaded}

\begin{center}\includegraphics{preProcessing_files/figure-latex/unnamed-chunk-61-1} \end{center}

\begin{Shaded}
\begin{Highlighting}[]
\CommentTok{#rm(ms, ss_prop)}
\end{Highlighting}
\end{Shaded}

\begin{Shaded}
\begin{Highlighting}[]
\NormalTok{ms }\OperatorTok
\StringTok{  }\KeywordTok{group_by}\NormalTok{(learning, frequency) }\OperatorTok
\StringTok{  }\KeywordTok{summarise}\NormalTok{(}\KeywordTok{mean}\NormalTok{(acc))}
\end{Highlighting}
\end{Shaded}

\begin{verbatim}
## # A tibble: 4 x 3
## # Groups:   learning [2]
##   learning frequency `mean(acc)`
##   <fct>    <fct>           <dbl>
## 1 FL       low             0.439
## 2 FL       high            0.750
## 3 LF       low             0.439
## 4 LF       high            0.644
\end{verbatim}

\begin{Shaded}
\begin{Highlighting}[]
\KeywordTok{par}\NormalTok{(}\DataTypeTok{mfrow=}\KeywordTok{c}\NormalTok{(}\DecValTok{2}\NormalTok{,}\DecValTok{2}\NormalTok{))}
\KeywordTok{hist}\NormalTok{(ms[ms}\OperatorTok{$}\NormalTok{frequency}\OperatorTok{==}\StringTok{'low'} \OperatorTok{&}\StringTok{ }\NormalTok{ms}\OperatorTok{$}\NormalTok{learning}\OperatorTok{==}\StringTok{'FL'}\NormalTok{,]}\OperatorTok{$}\NormalTok{acc, }\DataTypeTok{xlab =} \StringTok{'acc'}\NormalTok{, }\DataTypeTok{main =} \StringTok{'low freq - FL '}\NormalTok{)}
\KeywordTok{hist}\NormalTok{(ms[ms}\OperatorTok{$}\NormalTok{frequency}\OperatorTok{==}\StringTok{'low'} \OperatorTok{&}\StringTok{ }\NormalTok{ms}\OperatorTok{$}\NormalTok{learning}\OperatorTok{==}\StringTok{'LF'}\NormalTok{,]}\OperatorTok{$}\NormalTok{acc, }\DataTypeTok{xlab =} \StringTok{'acc'}\NormalTok{, }\DataTypeTok{main =} \StringTok{'low freq - LF '}\NormalTok{)}
\KeywordTok{hist}\NormalTok{(ms[ms}\OperatorTok{$}\NormalTok{frequency}\OperatorTok{==}\StringTok{'high'} \OperatorTok{&}\StringTok{ }\NormalTok{ms}\OperatorTok{$}\NormalTok{learning}\OperatorTok{==}\StringTok{'FL'}\NormalTok{,]}\OperatorTok{$}\NormalTok{acc, }\DataTypeTok{xlab =} \StringTok{'acc'}\NormalTok{, }\DataTypeTok{main =} \StringTok{'high freq - FL '}\NormalTok{)}
\KeywordTok{hist}\NormalTok{(ms[ms}\OperatorTok{$}\NormalTok{frequency}\OperatorTok{==}\StringTok{'high'} \OperatorTok{&}\StringTok{ }\NormalTok{ms}\OperatorTok{$}\NormalTok{learning}\OperatorTok{==}\StringTok{'LF'}\NormalTok{,]}\OperatorTok{$}\NormalTok{acc, }\DataTypeTok{xlab =} \StringTok{'acc'}\NormalTok{, }\DataTypeTok{main =} \StringTok{'high freq - LF '}\NormalTok{)}
\end{Highlighting}
\end{Shaded}

\includegraphics{preProcessing_files/figure-latex/hist distribution of responses-1.pdf}

\begin{Shaded}
\begin{Highlighting}[]
\KeywordTok{par}\NormalTok{(}\DataTypeTok{mfrow=}\KeywordTok{c}\NormalTok{(}\DecValTok{1}\NormalTok{,}\DecValTok{1}\NormalTok{))}
\end{Highlighting}
\end{Shaded}

\hypertarget{barplot-accuracy-by-frequency-learning-1}{%
\paragraph{Barplot accuracy by frequency +
learning}\label{barplot-accuracy-by-frequency-learning-1}}

\begin{Shaded}
\begin{Highlighting}[]
\CommentTok{#barPlot aggregated over categories:}

\NormalTok{ms <-}\StringTok{ }\KeywordTok{aggregate}\NormalTok{(acc }\OperatorTok{~}\StringTok{ }\NormalTok{subjID}\OperatorTok{+}\NormalTok{frequency}\OperatorTok{+}\NormalTok{learning, }
                \DataTypeTok{data=}\NormalTok{labelPicture[labelPicture}\OperatorTok{$}\NormalTok{rt }\OperatorTok{>}\StringTok{ }\DecValTok{100} \OperatorTok{&}
\StringTok{                                    }\NormalTok{labelPicture}\OperatorTok{$}\NormalTok{rt }\OperatorTok{<=}\StringTok{ }\DecValTok{2500} \OperatorTok{&}
\StringTok{                                  }\OperatorTok{!}\NormalTok{(labelPicture}\OperatorTok{$}\NormalTok{subjID }\OperatorTok\StringTok{ }\NormalTok{badsubjs)  ,], }\DataTypeTok{FUN=}\NormalTok{ mean)}

\NormalTok{df<-}\StringTok{ }\NormalTok{ms }\OperatorTok
\StringTok{  }\KeywordTok{group_by}\NormalTok{(frequency, learning)}\OperatorTok
\StringTok{  }\KeywordTok{summarise}\NormalTok{(}
    \DataTypeTok{mean =} \KeywordTok{mean}\NormalTok{(acc),}
    \DataTypeTok{sd =} \KeywordTok{sd}\NormalTok{(acc),}
    \DataTypeTok{n =} \KeywordTok{n}\NormalTok{()) }\OperatorTok
\StringTok{  }\KeywordTok{mutate}\NormalTok{( }\DataTypeTok{se=}\NormalTok{sd}\OperatorTok{/}\KeywordTok{sqrt}\NormalTok{(n))  }\OperatorTok\StringTok{ }
\StringTok{  }\KeywordTok{mutate}\NormalTok{( }\DataTypeTok{ci=}\NormalTok{se }\OperatorTok{*}\StringTok{ }\KeywordTok{qt}\NormalTok{((}\DecValTok{1}\FloatTok{-0.05}\NormalTok{)}\OperatorTok{/}\DecValTok{2} \OperatorTok{+}\StringTok{ }\FloatTok{.5}\NormalTok{, n}\DecValTok{-1}\NormalTok{))}

\NormalTok{df}\OperatorTok{$}\NormalTok{frequency <-}\StringTok{ }\KeywordTok{as.factor}\NormalTok{(df}\OperatorTok{$}\NormalTok{frequency)}
\NormalTok{plyr}\OperatorTok{::}\KeywordTok{revalue}\NormalTok{(df}\OperatorTok{$}\NormalTok{frequency, }\KeywordTok{c}\NormalTok{(}\StringTok{"25"}\NormalTok{=}\StringTok{"low"}\NormalTok{))->}\StringTok{ }\NormalTok{df}\OperatorTok{$}\NormalTok{frequency;}
\NormalTok{plyr}\OperatorTok{::}\KeywordTok{revalue}\NormalTok{(df}\OperatorTok{$}\NormalTok{frequency, }\KeywordTok{c}\NormalTok{(}\StringTok{"75"}\NormalTok{=}\StringTok{"high"}\NormalTok{))->}\StringTok{ }\NormalTok{df}\OperatorTok{$}\NormalTok{frequency;}


\NormalTok{lp<-}\KeywordTok{ggplot}\NormalTok{(}\KeywordTok{aes}\NormalTok{(}\DataTypeTok{x =}\NormalTok{ frequency, }\DataTypeTok{y =}\NormalTok{ mean, }\DataTypeTok{fill =}\NormalTok{ frequency), }\DataTypeTok{data =}\NormalTok{ df) }\OperatorTok{+}
\StringTok{  }\KeywordTok{facet_grid}\NormalTok{( . }\OperatorTok{~}\StringTok{ }\NormalTok{learning) }\OperatorTok{+}
\StringTok{  }\KeywordTok{geom_bar}\NormalTok{(}\DataTypeTok{stat =} \StringTok{"identity"}\NormalTok{, }\DataTypeTok{color=}\StringTok{'white'}\NormalTok{, }\DataTypeTok{position=}\KeywordTok{position_dodge}\NormalTok{(), }\DataTypeTok{size=}\FloatTok{1.2}\NormalTok{) }\OperatorTok{+}
\StringTok{  }\KeywordTok{geom_errorbar}\NormalTok{(}\KeywordTok{aes}\NormalTok{(}\DataTypeTok{ymin=}\NormalTok{mean}\OperatorTok{-}\NormalTok{se, }\DataTypeTok{ymax=}\NormalTok{mean}\OperatorTok{+}\NormalTok{se), }\DataTypeTok{width=}\NormalTok{.}\DecValTok{15}\NormalTok{, }\DataTypeTok{size=}\DecValTok{1}\NormalTok{,}\DataTypeTok{position=}\KeywordTok{position_dodge}\NormalTok{(.}\DecValTok{9}\NormalTok{)) }\OperatorTok{+}
\StringTok{  }\KeywordTok{ylab}\NormalTok{(}\StringTok{"Accuracy "}\NormalTok{) }\OperatorTok{+}
\StringTok{  }\KeywordTok{xlab}\NormalTok{(}\StringTok{"frequency"}\NormalTok{) }\OperatorTok{+}
\StringTok{  }\KeywordTok{ggtitle}\NormalTok{(}\StringTok{'labelPictures'}\NormalTok{) }\OperatorTok{+}
\StringTok{  }\KeywordTok{coord_cartesian}\NormalTok{(}\DataTypeTok{ylim =} \KeywordTok{c}\NormalTok{(}\DecValTok{0}\NormalTok{, }\DecValTok{1}\NormalTok{))}\OperatorTok{+}
\StringTok{  }\NormalTok{ggpubr}\OperatorTok{::}\KeywordTok{theme_pubclean}\NormalTok{() }\OperatorTok{+}\StringTok{ }
\StringTok{  }\KeywordTok{theme}\NormalTok{(}\DataTypeTok{legend.position=}\StringTok{"bottom"}\NormalTok{, }\DataTypeTok{legend.title =} \KeywordTok{element_blank}\NormalTok{()) }\OperatorTok{+}
\StringTok{  }\KeywordTok{theme}\NormalTok{(}\DataTypeTok{text =} \KeywordTok{element_text}\NormalTok{(}\DataTypeTok{size=}\DecValTok{10}\NormalTok{)) }\OperatorTok{+}
\StringTok{  }\KeywordTok{geom_hline}\NormalTok{(}\DataTypeTok{yintercept =} \FloatTok{.33}\NormalTok{, }\DataTypeTok{col=}\StringTok{'red'}\NormalTok{, }\DataTypeTok{lwd=}\DecValTok{1}\NormalTok{);}
\end{Highlighting}
\end{Shaded}

\hypertarget{comparison-by-frequency-by-learning-by-tasks}{%
\subsubsection{Comparison by frequency by learning by
tasks}\label{comparison-by-frequency-by-learning-by-tasks}}

Quick summary of what we have so far:

\begin{Shaded}
\begin{Highlighting}[]
\KeywordTok{grid.arrange}\NormalTok{(pl,lp)}
\end{Highlighting}
\end{Shaded}

\begin{center}\includegraphics{preProcessing_files/figure-latex/barplots with both tasks-1} \end{center}

What's going on in the low frequency condition? One way to see whether
they simply learned another association is to check that wrong choices
are distributed equally (50\%) to the other two categories. If they are,
then they didn't learn anything, but if they are not distributed
equally, they have learned another association.

Label picture:

\begin{Shaded}
\begin{Highlighting}[]
\CommentTok{#select only inaccurate trials}
\NormalTok{temp <-}\StringTok{ }\NormalTok{labelPicture[labelPicture}\OperatorTok{$}\NormalTok{acc}\OperatorTok{==}\DecValTok{0}\NormalTok{,] }

\KeywordTok{round}\NormalTok{(}\KeywordTok{nrow}\NormalTok{(temp)}\OperatorTok{/}\KeywordTok{nrow}\NormalTok{(labelPicture)}\OperatorTok{*}\DecValTok{100}\NormalTok{,}\DecValTok{2}\NormalTok{)}
\end{Highlighting}
\end{Shaded}

\begin{verbatim}
## [1] 46.07
\end{verbatim}

How many of those are low frequency trials?

\begin{Shaded}
\begin{Highlighting}[]
\KeywordTok{round}\NormalTok{(}\KeywordTok{nrow}\NormalTok{(temp[temp}\OperatorTok{$}\NormalTok{frequency}\OperatorTok{==}\DecValTok{25}\NormalTok{,])}\OperatorTok{/}\KeywordTok{nrow}\NormalTok{(labelPicture)}\OperatorTok{*}\DecValTok{100}\NormalTok{,}\DecValTok{2}\NormalTok{)}
\end{Highlighting}
\end{Shaded}

\begin{verbatim}
## [1] 29.5
\end{verbatim}

How many of those are low frequency trials and how are they distributed
across learnings?

\begin{Shaded}
\begin{Highlighting}[]
\KeywordTok{round}\NormalTok{(}\KeywordTok{nrow}\NormalTok{(temp[temp}\OperatorTok{$}\NormalTok{frequency}\OperatorTok{==}\DecValTok{25} \OperatorTok{&}\StringTok{ }\NormalTok{temp}\OperatorTok{$}\NormalTok{learning}\OperatorTok{==}\StringTok{"FL"}\NormalTok{,])}\OperatorTok{/}\KeywordTok{nrow}\NormalTok{(labelPicture)}\OperatorTok{*}\DecValTok{100}\NormalTok{,}\DecValTok{2}\NormalTok{)}
\end{Highlighting}
\end{Shaded}

\begin{verbatim}
## [1] 15.79
\end{verbatim}

\begin{Shaded}
\begin{Highlighting}[]
\KeywordTok{round}\NormalTok{(}\KeywordTok{nrow}\NormalTok{(temp[temp}\OperatorTok{$}\NormalTok{frequency}\OperatorTok{==}\DecValTok{25} \OperatorTok{&}\StringTok{ }\NormalTok{temp}\OperatorTok{$}\NormalTok{learning}\OperatorTok{==}\StringTok{"LF"}\NormalTok{,])}\OperatorTok{/}\KeywordTok{nrow}\NormalTok{(labelPicture)}\OperatorTok{*}\DecValTok{100}\NormalTok{,}\DecValTok{2}\NormalTok{)}
\end{Highlighting}
\end{Shaded}

\begin{verbatim}
## [1] 13.71
\end{verbatim}

FL people make more errors in the low freq condition

How many of those are high frequency trials and how are they distributed
across learnings?

\begin{Shaded}
\begin{Highlighting}[]
\KeywordTok{round}\NormalTok{(}\KeywordTok{nrow}\NormalTok{(temp[temp}\OperatorTok{$}\NormalTok{frequency}\OperatorTok{==}\DecValTok{75} \OperatorTok{&}\StringTok{ }\NormalTok{temp}\OperatorTok{$}\NormalTok{learning}\OperatorTok{==}\StringTok{"FL"}\NormalTok{,])}\OperatorTok{/}\KeywordTok{nrow}\NormalTok{(labelPicture)}\OperatorTok{*}\DecValTok{100}\NormalTok{,}\DecValTok{2}\NormalTok{)}
\end{Highlighting}
\end{Shaded}

\begin{verbatim}
## [1] 7.19
\end{verbatim}

\begin{Shaded}
\begin{Highlighting}[]
\KeywordTok{round}\NormalTok{(}\KeywordTok{nrow}\NormalTok{(temp[temp}\OperatorTok{$}\NormalTok{frequency}\OperatorTok{==}\DecValTok{75} \OperatorTok{&}\StringTok{ }\NormalTok{temp}\OperatorTok{$}\NormalTok{learning}\OperatorTok{==}\StringTok{"LF"}\NormalTok{,])}\OperatorTok{/}\KeywordTok{nrow}\NormalTok{(labelPicture)}\OperatorTok{*}\DecValTok{100}\NormalTok{,}\DecValTok{2}\NormalTok{)}
\end{Highlighting}
\end{Shaded}

\begin{verbatim}
## [1] 9.39
\end{verbatim}

While they are pretty much the same in the high frequency

Label picture task:

correct choice is listed in ``label'', that is, label presented.
Participant's choice is listed in ``category'', that is, the fribble's
category.

\begin{Shaded}
\begin{Highlighting}[]
\NormalTok{temp }\OperatorTok
\StringTok{  }\KeywordTok{filter}\NormalTok{(frequency}\OperatorTok{==}\StringTok{"25"}\NormalTok{) }\OperatorTok
\StringTok{  }\KeywordTok{group_by}\NormalTok{(learning, label, category) }\OperatorTok
\StringTok{  }\KeywordTok{count}\NormalTok{()}
\end{Highlighting}
\end{Shaded}

\begin{verbatim}
## # A tibble: 12 x 4
## # Groups:   learning, label, category [12]
##    learning label category     n
##    <fct>    <fct>    <int> <int>
##  1 FL       bim          1    36
##  2 FL       bim          3    92
##  3 FL       dep          2   107
##  4 FL       dep          3    43
##  5 FL       tob          1   107
##  6 FL       tob          2    39
##  7 LF       bim          1    22
##  8 LF       bim          3   100
##  9 LF       dep          2    74
## 10 LF       dep          3    44
## 11 LF       tob          1    78
## 12 LF       tob          2    50
\end{verbatim}

Nope, they definitely learned another association. The association they
have learned is based on the high saliency feature, rather than on the
low saliency one. Let's see if that is the case also for the other task:

Picture label task:

\begin{Shaded}
\begin{Highlighting}[]
\CommentTok{#select only inaccurate trials}
\NormalTok{temp <-}\StringTok{ }\NormalTok{pictureLabel[pictureLabel}\OperatorTok{$}\NormalTok{acc}\OperatorTok{==}\DecValTok{0}\NormalTok{,] }

\KeywordTok{round}\NormalTok{(}\KeywordTok{nrow}\NormalTok{(temp)}\OperatorTok{/}\KeywordTok{nrow}\NormalTok{(pictureLabel)}\OperatorTok{*}\DecValTok{100}\NormalTok{,}\DecValTok{2}\NormalTok{)}
\end{Highlighting}
\end{Shaded}

\begin{verbatim}
## [1] 46.26
\end{verbatim}

How many of those are low frequency trials?

\begin{Shaded}
\begin{Highlighting}[]
\KeywordTok{round}\NormalTok{(}\KeywordTok{nrow}\NormalTok{(temp[temp}\OperatorTok{$}\NormalTok{frequency}\OperatorTok{==}\DecValTok{25}\NormalTok{,])}\OperatorTok{/}\KeywordTok{nrow}\NormalTok{(pictureLabel)}\OperatorTok{*}\DecValTok{100}\NormalTok{,}\DecValTok{2}\NormalTok{)}
\end{Highlighting}
\end{Shaded}

\begin{verbatim}
## [1] 29.93
\end{verbatim}

How many of those are low frequency trials and how are they distributed
across learnings?

\begin{Shaded}
\begin{Highlighting}[]
\KeywordTok{round}\NormalTok{(}\KeywordTok{nrow}\NormalTok{(temp[temp}\OperatorTok{$}\NormalTok{frequency}\OperatorTok{==}\DecValTok{25} \OperatorTok{&}\StringTok{ }\NormalTok{temp}\OperatorTok{$}\NormalTok{learning}\OperatorTok{==}\StringTok{"FL"}\NormalTok{,])}\OperatorTok{/}\KeywordTok{nrow}\NormalTok{(pictureLabel)}\OperatorTok{*}\DecValTok{100}\NormalTok{,}\DecValTok{2}\NormalTok{)}
\end{Highlighting}
\end{Shaded}

\begin{verbatim}
## [1] 15.44
\end{verbatim}

\begin{Shaded}
\begin{Highlighting}[]
\KeywordTok{round}\NormalTok{(}\KeywordTok{nrow}\NormalTok{(temp[temp}\OperatorTok{$}\NormalTok{frequency}\OperatorTok{==}\DecValTok{25} \OperatorTok{&}\StringTok{ }\NormalTok{temp}\OperatorTok{$}\NormalTok{learning}\OperatorTok{==}\StringTok{"LF"}\NormalTok{,])}\OperatorTok{/}\KeywordTok{nrow}\NormalTok{(pictureLabel)}\OperatorTok{*}\DecValTok{100}\NormalTok{,}\DecValTok{2}\NormalTok{)}
\end{Highlighting}
\end{Shaded}

\begin{verbatim}
## [1] 14.48
\end{verbatim}

How many of those are high frequency trials and how are they distributed
across learnings?

\begin{Shaded}
\begin{Highlighting}[]
\KeywordTok{round}\NormalTok{(}\KeywordTok{nrow}\NormalTok{(temp[temp}\OperatorTok{$}\NormalTok{frequency}\OperatorTok{==}\DecValTok{75} \OperatorTok{&}\StringTok{ }\NormalTok{temp}\OperatorTok{$}\NormalTok{learning}\OperatorTok{==}\StringTok{"FL"}\NormalTok{,])}\OperatorTok{/}\KeywordTok{nrow}\NormalTok{(pictureLabel)}\OperatorTok{*}\DecValTok{100}\NormalTok{,}\DecValTok{2}\NormalTok{)}
\end{Highlighting}
\end{Shaded}

\begin{verbatim}
## [1] 7.93
\end{verbatim}

\begin{Shaded}
\begin{Highlighting}[]
\KeywordTok{round}\NormalTok{(}\KeywordTok{nrow}\NormalTok{(temp[temp}\OperatorTok{$}\NormalTok{frequency}\OperatorTok{==}\DecValTok{75} \OperatorTok{&}\StringTok{ }\NormalTok{temp}\OperatorTok{$}\NormalTok{learning}\OperatorTok{==}\StringTok{"LF"}\NormalTok{,])}\OperatorTok{/}\KeywordTok{nrow}\NormalTok{(pictureLabel)}\OperatorTok{*}\DecValTok{100}\NormalTok{,}\DecValTok{2}\NormalTok{)}
\end{Highlighting}
\end{Shaded}

\begin{verbatim}
## [1] 8.41
\end{verbatim}

Picture label task:

correct choice is listed in ``category'', that is, the category of the
fribble presented. Participant's choice is listed in ``resp'' column,
that is, the label chosen.

\begin{Shaded}
\begin{Highlighting}[]
\NormalTok{temp }\OperatorTok
\StringTok{  }\KeywordTok{filter}\NormalTok{(frequency}\OperatorTok{==}\StringTok{"25"}\NormalTok{) }\OperatorTok
\StringTok{  }\KeywordTok{group_by}\NormalTok{(learning, category, resp) }\OperatorTok
\StringTok{  }\KeywordTok{count}\NormalTok{()}
\end{Highlighting}
\end{Shaded}

\begin{verbatim}
## # A tibble: 12 x 4
## # Groups:   learning, category, resp [12]
##    learning category resp      n
##    <fct>       <int> <fct> <int>
##  1 FL              1 bim      44
##  2 FL              1 tob     110
##  3 FL              2 dep      78
##  4 FL              2 tob      60
##  5 FL              3 bim      83
##  6 FL              3 dep      42
##  7 LF              1 bim      53
##  8 LF              1 tob      93
##  9 LF              2 dep      61
## 10 LF              2 tob      55
## 11 LF              3 bim      91
## 12 LF              3 dep      38
\end{verbatim}

In both tasks participants were driven by the high salient feature in
making errors, they simply learned only one association between the
label and the high salient feature, and made decisions based on this.

\hypertarget{speed-accuracy-trade-off-by-tasks}{%
\subsection{Speed-accuracy trade-off by
tasks}\label{speed-accuracy-trade-off-by-tasks}}

Inspection of the speed-accuracy trade-off:

Label Picture

\begin{Shaded}
\begin{Highlighting}[]
\KeywordTok{aggregate}\NormalTok{(acc }\OperatorTok{~}\StringTok{ }\NormalTok{subjID}\OperatorTok{+}\NormalTok{learning, labelPicture[labelPicture}\OperatorTok{$}\NormalTok{rt }\OperatorTok{>}\StringTok{ }\DecValTok{100} \OperatorTok{&}
\StringTok{                                         }\OperatorTok{!}\NormalTok{(labelPicture}\OperatorTok{$}\NormalTok{subjID }\OperatorTok\StringTok{ }\NormalTok{badsubjs) ,], mean)->}\StringTok{ }\NormalTok{speedacc}

\KeywordTok{aggregate}\NormalTok{(rt }\OperatorTok{~}\StringTok{ }\NormalTok{subjID}\OperatorTok{+}\NormalTok{learning, labelPicture[labelPicture}\OperatorTok{$}\NormalTok{rt }\OperatorTok{>}\StringTok{ }\DecValTok{100} \OperatorTok{&}
\StringTok{                                         }\OperatorTok{!}\NormalTok{(labelPicture}\OperatorTok{$}\NormalTok{subjID }\OperatorTok\StringTok{ }\NormalTok{badsubjs),], mean)->}\StringTok{ }\NormalTok{speedacc2}
\KeywordTok{merge}\NormalTok{(speedacc, speedacc2, }\DataTypeTok{by =}  \KeywordTok{c}\NormalTok{(}\StringTok{"subjID"}\NormalTok{, }\StringTok{"learning"}\NormalTok{))->}\StringTok{ }\NormalTok{speedacc}

\KeywordTok{ggplot}\NormalTok{(}\KeywordTok{aes}\NormalTok{(}\DataTypeTok{x=}\NormalTok{rt, }\DataTypeTok{y=}\NormalTok{acc), }
           \DataTypeTok{data =}\NormalTok{ speedacc) }\OperatorTok{+}\StringTok{ }
\StringTok{  }\KeywordTok{facet_grid}\NormalTok{( . }\OperatorTok{~}\StringTok{ }\NormalTok{learning) }\OperatorTok{+}\StringTok{ }
\StringTok{  }\KeywordTok{geom_point}\NormalTok{( }\DataTypeTok{shape =} \DecValTok{21}\NormalTok{, }\DataTypeTok{fill =} \StringTok{"white"}\NormalTok{, }\DataTypeTok{size =} \DecValTok{3}\NormalTok{, }\DataTypeTok{stroke =} \FloatTok{1.5}\NormalTok{) }\OperatorTok{+}
\StringTok{  }\CommentTok{#geom_smooth(method = "lm", formula = y ~ poly(x,2), se = TRUE, color = "#0892d0", fill = "lightgray") +}
\StringTok{  }\KeywordTok{geom_hline}\NormalTok{(}\DataTypeTok{yintercept =} \FloatTok{0.33}\NormalTok{, }\DataTypeTok{lty =} \StringTok{"dashed"}\NormalTok{, }\DataTypeTok{color =} \StringTok{'red'}\NormalTok{) }\OperatorTok{+}
\StringTok{  }\KeywordTok{coord_cartesian}\NormalTok{(}\DataTypeTok{ylim =} \KeywordTok{c}\NormalTok{(}\DecValTok{0}\NormalTok{, }\DecValTok{1}\NormalTok{))}\OperatorTok{+}
\StringTok{  }\NormalTok{ggthemes}\OperatorTok{::}\KeywordTok{theme_hc}\NormalTok{()}\OperatorTok{+}
\StringTok{  }\KeywordTok{xlab}\NormalTok{(}\StringTok{"Average RT on subjs"}\NormalTok{) }\OperatorTok{+}
\StringTok{  }\KeywordTok{ylab}\NormalTok{(}\StringTok{"Proportion Correct"}\NormalTok{) }\OperatorTok{+}
\StringTok{  }\KeywordTok{ggtitle}\NormalTok{(}\StringTok{"speed-acc tradeoff - labelPicture"}\NormalTok{)}
\end{Highlighting}
\end{Shaded}

\includegraphics{preProcessing_files/figure-latex/unnamed-chunk-74-1.pdf}

\begin{Shaded}
\begin{Highlighting}[]
\KeywordTok{aggregate}\NormalTok{(acc }\OperatorTok{~}\StringTok{ }\NormalTok{subjID}\OperatorTok{+}\NormalTok{learning}\OperatorTok{+}\NormalTok{frequency, labelPicture[labelPicture}\OperatorTok{$}\NormalTok{rt }\OperatorTok{>}\StringTok{ }\DecValTok{100} \OperatorTok{&}
\StringTok{                                                          }\NormalTok{labelPicture}\OperatorTok{$}\NormalTok{rt }\OperatorTok{<=}\StringTok{ }\DecValTok{2500} \OperatorTok{&}
\StringTok{                                         }\OperatorTok{!}\NormalTok{(labelPicture}\OperatorTok{$}\NormalTok{subjID }\OperatorTok\StringTok{ }\NormalTok{badsubjs) ,], mean)->}\StringTok{ }\NormalTok{speedacc}

\KeywordTok{aggregate}\NormalTok{(rt }\OperatorTok{~}\StringTok{ }\NormalTok{subjID}\OperatorTok{+}\NormalTok{learning}\OperatorTok{+}\NormalTok{frequency, labelPicture[labelPicture}\OperatorTok{$}\NormalTok{rt }\OperatorTok{>}\StringTok{ }\DecValTok{100} \OperatorTok{&}
\StringTok{                                                         }\NormalTok{labelPicture}\OperatorTok{$}\NormalTok{rt }\OperatorTok{<=}\StringTok{ }\DecValTok{2500} \OperatorTok{&}
\StringTok{                                         }\OperatorTok{!}\NormalTok{(labelPicture}\OperatorTok{$}\NormalTok{subjID }\OperatorTok\StringTok{ }\NormalTok{badsubjs),], mean)->}\StringTok{ }\NormalTok{speedacc2}
\KeywordTok{merge}\NormalTok{(speedacc, speedacc2, }\DataTypeTok{by =}  \KeywordTok{c}\NormalTok{(}\StringTok{"subjID"}\NormalTok{, }\StringTok{"learning"}\NormalTok{, }\StringTok{"frequency"}\NormalTok{))->}\StringTok{ }\NormalTok{speedacc}
\KeywordTok{rm}\NormalTok{(speedacc2)}
\NormalTok{dplyr}\OperatorTok{::}\KeywordTok{recode}\NormalTok{(speedacc}\OperatorTok{$}\NormalTok{frequency, }\StringTok{"25"}\NormalTok{=}\StringTok{"low"}\NormalTok{, }\StringTok{"75"}\NormalTok{=}\StringTok{"high"}\NormalTok{)->}\StringTok{ }\NormalTok{speedacc}\OperatorTok{$}\NormalTok{frequency;}

\KeywordTok{ggplot}\NormalTok{(}\KeywordTok{aes}\NormalTok{(}\DataTypeTok{x=}\NormalTok{rt, }\DataTypeTok{y=}\NormalTok{acc), }
           \DataTypeTok{data =}\NormalTok{ speedacc) }\OperatorTok{+}\StringTok{ }
\StringTok{  }\KeywordTok{facet_grid}\NormalTok{( learning }\OperatorTok{~}\StringTok{ }\NormalTok{frequency) }\OperatorTok{+}\StringTok{ }
\StringTok{  }\KeywordTok{geom_point}\NormalTok{( }\DataTypeTok{shape =} \DecValTok{21}\NormalTok{, }\DataTypeTok{fill =} \StringTok{"white"}\NormalTok{, }\DataTypeTok{size =} \DecValTok{3}\NormalTok{, }\DataTypeTok{stroke =} \FloatTok{1.5}\NormalTok{) }\OperatorTok{+}
\StringTok{  }\CommentTok{#geom_smooth(method = "lm", formula = y ~ poly(x,2), se = TRUE, color = "#0892d0", fill = "lightgray") +}
\StringTok{  }\KeywordTok{geom_hline}\NormalTok{(}\DataTypeTok{yintercept =} \FloatTok{0.33}\NormalTok{, }\DataTypeTok{lty =} \StringTok{"dashed"}\NormalTok{, }\DataTypeTok{color =} \StringTok{'red'}\NormalTok{) }\OperatorTok{+}
\StringTok{  }\KeywordTok{coord_cartesian}\NormalTok{(}\DataTypeTok{ylim =} \KeywordTok{c}\NormalTok{(}\DecValTok{0}\NormalTok{, }\DecValTok{1}\NormalTok{))}\OperatorTok{+}
\StringTok{  }\NormalTok{ggthemes}\OperatorTok{::}\KeywordTok{theme_base}\NormalTok{()}\OperatorTok{+}
\StringTok{  }\KeywordTok{xlab}\NormalTok{(}\StringTok{"Average RT on subjs"}\NormalTok{) }\OperatorTok{+}
\StringTok{  }\KeywordTok{ylab}\NormalTok{(}\StringTok{"Proportion Correct"}\NormalTok{) }\OperatorTok{+}
\StringTok{  }\KeywordTok{ggtitle}\NormalTok{(}\StringTok{"speed-acc tradeoff - labelPicture"}\NormalTok{)}
\end{Highlighting}
\end{Shaded}

\begin{center}\includegraphics{preProcessing_files/figure-latex/unnamed-chunk-75-1} \end{center}

\begin{Shaded}
\begin{Highlighting}[]
\NormalTok{speedacc }\OperatorTok
\StringTok{  }\KeywordTok{group_by}\NormalTok{(frequency, learning) }\OperatorTok
\StringTok{  }\KeywordTok{summarise}\NormalTok{(}\KeywordTok{mean}\NormalTok{(rt), }\KeywordTok{median}\NormalTok{(rt))}
\end{Highlighting}
\end{Shaded}

\begin{verbatim}
## # A tibble: 4 x 4
## # Groups:   frequency [2]
##   frequency learning `mean(rt)` `median(rt)`
##   <chr>     <fct>         <dbl>        <dbl>
## 1 high      FL            1390.        1373.
## 2 high      LF            1441.        1435.
## 3 low       FL            1489.        1504.
## 4 low       LF            1539.        1537.
\end{verbatim}

PictureLabel

\begin{Shaded}
\begin{Highlighting}[]
\KeywordTok{aggregate}\NormalTok{(acc }\OperatorTok{~}\StringTok{ }\NormalTok{subjID}\OperatorTok{+}\NormalTok{learning, pictureLabel[pictureLabel}\OperatorTok{$}\NormalTok{rt }\OperatorTok{>}\StringTok{ }\DecValTok{100}  \OperatorTok{&}
\StringTok{                                         }\OperatorTok{!}\NormalTok{(pictureLabel}\OperatorTok{$}\NormalTok{subjID }\OperatorTok\StringTok{ }\NormalTok{badsubjs),], mean)->}\StringTok{ }\NormalTok{speedacc}

\KeywordTok{aggregate}\NormalTok{(rt }\OperatorTok{~}\StringTok{ }\NormalTok{subjID}\OperatorTok{+}\NormalTok{learning, pictureLabel[pictureLabel}\OperatorTok{$}\NormalTok{rt }\OperatorTok{>}\StringTok{ }\DecValTok{100}  \OperatorTok{&}
\StringTok{                                         }\OperatorTok{!}\NormalTok{(pictureLabel}\OperatorTok{$}\NormalTok{subjID }\OperatorTok\StringTok{ }\NormalTok{badsubjs),], mean)->}\StringTok{ }\NormalTok{speedacc2}
\KeywordTok{merge}\NormalTok{(speedacc, speedacc2, }\DataTypeTok{by =}  \KeywordTok{c}\NormalTok{(}\StringTok{"subjID"}\NormalTok{, }\StringTok{"learning"}\NormalTok{))->}\StringTok{ }\NormalTok{speedacc}

\KeywordTok{ggplot}\NormalTok{(}\KeywordTok{aes}\NormalTok{(}\DataTypeTok{x=}\NormalTok{rt, }\DataTypeTok{y=}\NormalTok{acc), }
           \DataTypeTok{data =}\NormalTok{ speedacc) }\OperatorTok{+}\StringTok{ }
\StringTok{  }\KeywordTok{facet_grid}\NormalTok{( . }\OperatorTok{~}\StringTok{ }\NormalTok{learning) }\OperatorTok{+}\StringTok{ }
\StringTok{  }\KeywordTok{geom_point}\NormalTok{( }\DataTypeTok{shape =} \DecValTok{21}\NormalTok{, }\DataTypeTok{fill =} \StringTok{"white"}\NormalTok{, }\DataTypeTok{size =} \DecValTok{3}\NormalTok{, }\DataTypeTok{stroke =} \FloatTok{1.5}\NormalTok{) }\OperatorTok{+}
\StringTok{  }\CommentTok{#geom_smooth(method = "lm", formula = y ~ poly(x,2), se = TRUE, color = "#0892d0", fill = "lightgray") +}
\StringTok{  }\KeywordTok{geom_hline}\NormalTok{(}\DataTypeTok{yintercept =} \FloatTok{0.33}\NormalTok{, }\DataTypeTok{lty =} \StringTok{"dashed"}\NormalTok{, }\DataTypeTok{color =} \StringTok{'red'}\NormalTok{) }\OperatorTok{+}
\StringTok{  }\KeywordTok{coord_cartesian}\NormalTok{(}\DataTypeTok{ylim =} \KeywordTok{c}\NormalTok{(}\DecValTok{0}\NormalTok{, }\DecValTok{1}\NormalTok{))}\OperatorTok{+}
\StringTok{  }\NormalTok{ggthemes}\OperatorTok{::}\KeywordTok{theme_hc}\NormalTok{()}\OperatorTok{+}
\StringTok{  }\KeywordTok{xlab}\NormalTok{(}\StringTok{"Average RT on subjs"}\NormalTok{) }\OperatorTok{+}
\StringTok{  }\KeywordTok{ylab}\NormalTok{(}\StringTok{"Proportion Correct"}\NormalTok{) }\OperatorTok{+}
\StringTok{  }\KeywordTok{ggtitle}\NormalTok{(}\StringTok{"speed-acc tradeoff - pictureLabel"}\NormalTok{)}
\end{Highlighting}
\end{Shaded}

\includegraphics{preProcessing_files/figure-latex/unnamed-chunk-77-1.pdf}

\begin{Shaded}
\begin{Highlighting}[]
\KeywordTok{aggregate}\NormalTok{(acc }\OperatorTok{~}\StringTok{ }\NormalTok{subjID}\OperatorTok{+}\NormalTok{learning}\OperatorTok{+}\NormalTok{frequency, pictureLabel[pictureLabel}\OperatorTok{$}\NormalTok{rt }\OperatorTok{>}\StringTok{ }\DecValTok{100}  \OperatorTok{&}
\StringTok{                                         }\OperatorTok{!}\NormalTok{(pictureLabel}\OperatorTok{$}\NormalTok{subjID }\OperatorTok\StringTok{ }\NormalTok{badsubjs),], mean)->}\StringTok{ }\NormalTok{speedacc}
\KeywordTok{aggregate}\NormalTok{(rt }\OperatorTok{~}\StringTok{ }\NormalTok{subjID}\OperatorTok{+}\NormalTok{learning}\OperatorTok{+}\NormalTok{frequency, pictureLabel[pictureLabel}\OperatorTok{$}\NormalTok{rt }\OperatorTok{>}\StringTok{ }\DecValTok{100}  \OperatorTok{&}
\StringTok{                                         }\OperatorTok{!}\NormalTok{(pictureLabel}\OperatorTok{$}\NormalTok{subjID }\OperatorTok\StringTok{ }\NormalTok{badsubjs),], mean)->}\StringTok{ }\NormalTok{speedacc2}
\KeywordTok{merge}\NormalTok{(speedacc, speedacc2, }\DataTypeTok{by =}  \KeywordTok{c}\NormalTok{(}\StringTok{"subjID"}\NormalTok{, }\StringTok{"learning"}\NormalTok{, }\StringTok{"frequency"}\NormalTok{))->}\StringTok{ }\NormalTok{speedacc}
\KeywordTok{rm}\NormalTok{(speedacc2)}
\NormalTok{dplyr}\OperatorTok{::}\KeywordTok{recode}\NormalTok{(speedacc}\OperatorTok{$}\NormalTok{frequency, }\StringTok{"25"}\NormalTok{=}\StringTok{"low"}\NormalTok{, }\StringTok{"75"}\NormalTok{=}\StringTok{"high"}\NormalTok{)->}\StringTok{ }\NormalTok{speedacc}\OperatorTok{$}\NormalTok{frequency;}

\KeywordTok{ggplot}\NormalTok{(}\KeywordTok{aes}\NormalTok{(}\DataTypeTok{x=}\NormalTok{rt, }\DataTypeTok{y=}\NormalTok{acc), }
           \DataTypeTok{data =}\NormalTok{ speedacc) }\OperatorTok{+}\StringTok{ }
\StringTok{  }\KeywordTok{facet_grid}\NormalTok{( learning }\OperatorTok{~}\StringTok{ }\NormalTok{frequency) }\OperatorTok{+}\StringTok{ }
\StringTok{  }\KeywordTok{geom_point}\NormalTok{( }\DataTypeTok{shape =} \DecValTok{21}\NormalTok{, }\DataTypeTok{fill =} \StringTok{"white"}\NormalTok{, }\DataTypeTok{size =} \DecValTok{3}\NormalTok{, }\DataTypeTok{stroke =} \FloatTok{1.5}\NormalTok{) }\OperatorTok{+}
\StringTok{  }\CommentTok{#geom_smooth(method = "lm", formula = y ~ poly(x,2), se = TRUE, color = "#0892d0", fill = "lightgray") +}
\StringTok{  }\KeywordTok{geom_hline}\NormalTok{(}\DataTypeTok{yintercept =} \FloatTok{0.33}\NormalTok{, }\DataTypeTok{lty =} \StringTok{"dashed"}\NormalTok{, }\DataTypeTok{color =} \StringTok{'red'}\NormalTok{) }\OperatorTok{+}
\StringTok{  }\KeywordTok{coord_cartesian}\NormalTok{(}\DataTypeTok{ylim =} \KeywordTok{c}\NormalTok{(}\DecValTok{0}\NormalTok{, }\DecValTok{1}\NormalTok{))}\OperatorTok{+}
\StringTok{  }\NormalTok{ggthemes}\OperatorTok{::}\KeywordTok{theme_base}\NormalTok{()}\OperatorTok{+}
\StringTok{  }\KeywordTok{xlab}\NormalTok{(}\StringTok{"Average RT on subjs"}\NormalTok{) }\OperatorTok{+}
\StringTok{  }\KeywordTok{ylab}\NormalTok{(}\StringTok{"Proportion Correct"}\NormalTok{) }\OperatorTok{+}
\StringTok{  }\KeywordTok{ggtitle}\NormalTok{(}\StringTok{"speed-acc tradeoff - pictureLabel"}\NormalTok{)}
\end{Highlighting}
\end{Shaded}

\begin{center}\includegraphics{preProcessing_files/figure-latex/unnamed-chunk-78-1} \end{center}

\begin{Shaded}
\begin{Highlighting}[]
\NormalTok{speedacc }\OperatorTok
\StringTok{  }\KeywordTok{group_by}\NormalTok{(frequency, learning) }\OperatorTok
\StringTok{  }\KeywordTok{summarise}\NormalTok{(}\KeywordTok{mean}\NormalTok{(rt), }\KeywordTok{median}\NormalTok{(rt))}
\end{Highlighting}
\end{Shaded}

\begin{verbatim}
## # A tibble: 4 x 4
## # Groups:   frequency [2]
##   frequency learning `mean(rt)` `median(rt)`
##   <chr>     <fct>         <dbl>        <dbl>
## 1 high      FL             911.         892.
## 2 high      LF            1016.         966.
## 3 low       FL             948.         918.
## 4 low       LF            1028.        1029.
\end{verbatim}

\hypertarget{final-comparisons}{%
\subsection{Final Comparisons}\label{final-comparisons}}

Barplot labelPicture

\begin{Shaded}
\begin{Highlighting}[]
\NormalTok{ms <-}\StringTok{ }\KeywordTok{aggregate}\NormalTok{(acc }\OperatorTok{~}\StringTok{ }\NormalTok{subjID}\OperatorTok{+}\NormalTok{frequency}\OperatorTok{+}\NormalTok{learning, }
                \DataTypeTok{data=}\NormalTok{labelPicture[labelPicture}\OperatorTok{$}\NormalTok{rt }\OperatorTok{>}\StringTok{ }\DecValTok{100}  \OperatorTok{&}
\StringTok{                                         }\OperatorTok{!}\NormalTok{(labelPicture}\OperatorTok{$}\NormalTok{subjID }\OperatorTok\StringTok{ }\NormalTok{badsubjs),], }\DataTypeTok{FUN=}\NormalTok{ mean)}

\NormalTok{df<-}\StringTok{ }\NormalTok{ms }\OperatorTok
\StringTok{  }\KeywordTok{group_by}\NormalTok{(frequency, learning)}\OperatorTok
\StringTok{  }\KeywordTok{summarise}\NormalTok{(}
    \DataTypeTok{mean =} \KeywordTok{mean}\NormalTok{(acc),}
    \DataTypeTok{sd =} \KeywordTok{sd}\NormalTok{(acc),}
    \DataTypeTok{n =} \KeywordTok{n}\NormalTok{()) }\OperatorTok
\StringTok{  }\KeywordTok{mutate}\NormalTok{( }\DataTypeTok{se=}\NormalTok{sd}\OperatorTok{/}\KeywordTok{sqrt}\NormalTok{(n))  }\OperatorTok\StringTok{ }
\StringTok{  }\KeywordTok{mutate}\NormalTok{( }\DataTypeTok{ci=}\NormalTok{se }\OperatorTok{*}\StringTok{ }\KeywordTok{qt}\NormalTok{((}\DecValTok{1}\FloatTok{-0.05}\NormalTok{)}\OperatorTok{/}\DecValTok{2} \OperatorTok{+}\StringTok{ }\FloatTok{.5}\NormalTok{, n}\DecValTok{-1}\NormalTok{))}

\NormalTok{df}\OperatorTok{$}\NormalTok{frequency <-}\StringTok{ }\KeywordTok{as.factor}\NormalTok{(df}\OperatorTok{$}\NormalTok{frequency)}
\NormalTok{plyr}\OperatorTok{::}\KeywordTok{revalue}\NormalTok{(df}\OperatorTok{$}\NormalTok{frequency, }\KeywordTok{c}\NormalTok{(}\StringTok{"25"}\NormalTok{=}\StringTok{"low"}\NormalTok{))->}\StringTok{ }\NormalTok{df}\OperatorTok{$}\NormalTok{frequency;}
\NormalTok{plyr}\OperatorTok{::}\KeywordTok{revalue}\NormalTok{(df}\OperatorTok{$}\NormalTok{frequency, }\KeywordTok{c}\NormalTok{(}\StringTok{"75"}\NormalTok{=}\StringTok{"high"}\NormalTok{))->}\StringTok{ }\NormalTok{df}\OperatorTok{$}\NormalTok{frequency;}


\NormalTok{lp<-}\KeywordTok{ggplot}\NormalTok{(}\KeywordTok{aes}\NormalTok{(}\DataTypeTok{x =}\NormalTok{ frequency, }\DataTypeTok{y =}\NormalTok{ mean, }\DataTypeTok{fill =}\NormalTok{ frequency), }\DataTypeTok{data =}\NormalTok{ df) }\OperatorTok{+}
\StringTok{  }\KeywordTok{facet_grid}\NormalTok{( . }\OperatorTok{~}\StringTok{ }\NormalTok{learning) }\OperatorTok{+}
\StringTok{  }\KeywordTok{geom_bar}\NormalTok{(}\DataTypeTok{stat =} \StringTok{"identity"}\NormalTok{, }\DataTypeTok{color=}\StringTok{'white'}\NormalTok{, }\DataTypeTok{position=}\KeywordTok{position_dodge}\NormalTok{(), }\DataTypeTok{size=}\FloatTok{1.2}\NormalTok{) }\OperatorTok{+}
\StringTok{  }\KeywordTok{geom_errorbar}\NormalTok{(}\KeywordTok{aes}\NormalTok{(}\DataTypeTok{ymin=}\NormalTok{mean}\OperatorTok{-}\NormalTok{se, }\DataTypeTok{ymax=}\NormalTok{mean}\OperatorTok{+}\NormalTok{se), }\DataTypeTok{width=}\NormalTok{.}\DecValTok{15}\NormalTok{, }\DataTypeTok{size=}\DecValTok{1}\NormalTok{,}\DataTypeTok{position=}\KeywordTok{position_dodge}\NormalTok{(.}\DecValTok{9}\NormalTok{)) }\OperatorTok{+}
\StringTok{  }\KeywordTok{ylab}\NormalTok{(}\StringTok{"Accuracy "}\NormalTok{) }\OperatorTok{+}
\StringTok{  }\KeywordTok{xlab}\NormalTok{(}\StringTok{"frequency"}\NormalTok{) }\OperatorTok{+}
\StringTok{  }\KeywordTok{ggtitle}\NormalTok{(}\StringTok{"labelPictures"}\NormalTok{) }\OperatorTok{+}
\StringTok{  }\KeywordTok{coord_cartesian}\NormalTok{(}\DataTypeTok{ylim =} \KeywordTok{c}\NormalTok{(}\DecValTok{0}\NormalTok{, }\DecValTok{1}\NormalTok{))}\OperatorTok{+}
\StringTok{  }\NormalTok{ggpubr}\OperatorTok{::}\KeywordTok{theme_pubclean}\NormalTok{() }\OperatorTok{+}\StringTok{ }
\StringTok{  }\KeywordTok{theme}\NormalTok{(}\DataTypeTok{legend.position =} \StringTok{"none"}\NormalTok{)  }\OperatorTok{+}
\StringTok{  }\KeywordTok{theme}\NormalTok{(}\DataTypeTok{text =} \KeywordTok{element_text}\NormalTok{(}\DataTypeTok{size=}\DecValTok{10}\NormalTok{)) }\OperatorTok{+}
\StringTok{  }\KeywordTok{geom_hline}\NormalTok{(}\DataTypeTok{yintercept =} \FloatTok{.33}\NormalTok{, }\DataTypeTok{col=}\StringTok{'red'}\NormalTok{, }\DataTypeTok{lwd=}\DecValTok{1}\NormalTok{);}
\end{Highlighting}
\end{Shaded}

\begin{Shaded}
\begin{Highlighting}[]
\KeywordTok{grid.arrange}\NormalTok{(lp, pl, }\DataTypeTok{ncol=}\DecValTok{2}\NormalTok{)}
\end{Highlighting}
\end{Shaded}

\includegraphics{preProcessing_files/figure-latex/unnamed-chunk-81-1.pdf}

\begin{Shaded}
\begin{Highlighting}[]
\NormalTok{ms <-}\StringTok{ }\KeywordTok{aggregate}\NormalTok{(acc }\OperatorTok{~}\StringTok{ }\NormalTok{subjID}\OperatorTok{+}\NormalTok{frequency}\OperatorTok{+}\NormalTok{learning, }
                \DataTypeTok{data=}\NormalTok{labelPicture[labelPicture}\OperatorTok{$}\NormalTok{rt }\OperatorTok{>}\StringTok{ }\DecValTok{100}  \OperatorTok{&}
\StringTok{                                    }\NormalTok{labelPicture}\OperatorTok{$}\NormalTok{rt }\OperatorTok{<=}\DecValTok{2500} \OperatorTok{&}
\StringTok{                                         }\OperatorTok{!}\NormalTok{(labelPicture}\OperatorTok{$}\NormalTok{subjID }\OperatorTok\StringTok{ }\NormalTok{badsubjs),], }\DataTypeTok{FUN=}\NormalTok{ mean)}

\NormalTok{ms}\OperatorTok{$}\NormalTok{frequency <-}\StringTok{ }\KeywordTok{as.factor}\NormalTok{(ms}\OperatorTok{$}\NormalTok{frequency)}
\NormalTok{plyr}\OperatorTok{::}\KeywordTok{revalue}\NormalTok{(ms}\OperatorTok{$}\NormalTok{frequency, }\KeywordTok{c}\NormalTok{(}\StringTok{"25"}\NormalTok{=}\StringTok{"low"}\NormalTok{))->}\StringTok{ }\NormalTok{ms}\OperatorTok{$}\NormalTok{frequency;}
\NormalTok{plyr}\OperatorTok{::}\KeywordTok{revalue}\NormalTok{(ms}\OperatorTok{$}\NormalTok{frequency, }\KeywordTok{c}\NormalTok{(}\StringTok{"75"}\NormalTok{=}\StringTok{"high"}\NormalTok{))->}\StringTok{ }\NormalTok{ms}\OperatorTok{$}\NormalTok{frequency;}

\NormalTok{lp_violin<-}\StringTok{ }\KeywordTok{ggviolin}\NormalTok{(ms, }\DataTypeTok{x =} \StringTok{"frequency"}\NormalTok{, }\DataTypeTok{y =} \StringTok{"acc"}\NormalTok{, }\DataTypeTok{fill =} \StringTok{"frequency"}\NormalTok{,}
         \DataTypeTok{palette =} \KeywordTok{c}\NormalTok{(}\StringTok{"#00AFBB"}\NormalTok{, }\StringTok{"#E7B800"}\NormalTok{),}
         \DataTypeTok{add =} \StringTok{"boxplot"}\NormalTok{, }
         \DataTypeTok{add.params =} \KeywordTok{list}\NormalTok{(}\DataTypeTok{fill =} \StringTok{"white"}\NormalTok{),}
         \DataTypeTok{trim=}\OtherTok{TRUE}\NormalTok{) }\OperatorTok{+}
\StringTok{         }\KeywordTok{ggtitle}\NormalTok{(}\StringTok{'labelPictures'}\NormalTok{) }\OperatorTok{+}
\StringTok{        }\KeywordTok{facet_grid}\NormalTok{( . }\OperatorTok{~}\StringTok{ }\NormalTok{learning) }\OperatorTok{+}
\StringTok{        }\KeywordTok{theme_pubclean}\NormalTok{()}\OperatorTok{+}
\StringTok{  }\KeywordTok{theme}\NormalTok{(}\DataTypeTok{legend.position =} \StringTok{"none"}\NormalTok{) }\OperatorTok{+}
\StringTok{  }\KeywordTok{geom_hline}\NormalTok{(}\DataTypeTok{yintercept =} \FloatTok{.33}\NormalTok{, }\DataTypeTok{col=}\StringTok{'red'}\NormalTok{, }\DataTypeTok{lwd=}\DecValTok{1}\NormalTok{);}


\NormalTok{ms <-}\StringTok{ }\KeywordTok{aggregate}\NormalTok{(acc }\OperatorTok{~}\StringTok{ }\NormalTok{subjID}\OperatorTok{+}\NormalTok{frequency}\OperatorTok{+}\NormalTok{learning, }
                \DataTypeTok{data=}\NormalTok{pictureLabel[pictureLabel}\OperatorTok{$}\NormalTok{rt }\OperatorTok{>}\StringTok{ }\DecValTok{100}  \OperatorTok{&}
\StringTok{                                         }\OperatorTok{!}\NormalTok{(pictureLabel}\OperatorTok{$}\NormalTok{subjID }\OperatorTok\StringTok{ }\NormalTok{badsubjs),], }\DataTypeTok{FUN=}\NormalTok{ mean)}

\NormalTok{ms}\OperatorTok{$}\NormalTok{frequency <-}\StringTok{ }\KeywordTok{as.factor}\NormalTok{(ms}\OperatorTok{$}\NormalTok{frequency)}
\NormalTok{plyr}\OperatorTok{::}\KeywordTok{revalue}\NormalTok{(ms}\OperatorTok{$}\NormalTok{frequency, }\KeywordTok{c}\NormalTok{(}\StringTok{"25"}\NormalTok{=}\StringTok{"low"}\NormalTok{))->}\StringTok{ }\NormalTok{ms}\OperatorTok{$}\NormalTok{frequency;}
\NormalTok{plyr}\OperatorTok{::}\KeywordTok{revalue}\NormalTok{(ms}\OperatorTok{$}\NormalTok{frequency, }\KeywordTok{c}\NormalTok{(}\StringTok{"75"}\NormalTok{=}\StringTok{"high"}\NormalTok{))->}\StringTok{ }\NormalTok{ms}\OperatorTok{$}\NormalTok{frequency;}

\NormalTok{pl_violin<-}\StringTok{ }\KeywordTok{ggviolin}\NormalTok{(ms, }\DataTypeTok{x =} \StringTok{"frequency"}\NormalTok{, }\DataTypeTok{y =} \StringTok{"acc"}\NormalTok{, }\DataTypeTok{fill =} \StringTok{"frequency"}\NormalTok{,}
         \DataTypeTok{palette =} \KeywordTok{c}\NormalTok{(}\StringTok{"#00AFBB"}\NormalTok{, }\StringTok{"#E7B800"}\NormalTok{),}
         \DataTypeTok{add =} \StringTok{"boxplot"}\NormalTok{, }
         \DataTypeTok{add.params =} \KeywordTok{list}\NormalTok{(}\DataTypeTok{fill =} \StringTok{"white"}\NormalTok{),}
         \DataTypeTok{trim=}\OtherTok{TRUE}\NormalTok{) }\OperatorTok{+}
\StringTok{         }\KeywordTok{ggtitle}\NormalTok{(}\StringTok{'pictureLabel'}\NormalTok{) }\OperatorTok{+}
\StringTok{        }\KeywordTok{facet_grid}\NormalTok{( . }\OperatorTok{~}\StringTok{ }\NormalTok{learning) }\OperatorTok{+}
\StringTok{        }\KeywordTok{theme_pubclean}\NormalTok{()}\OperatorTok{+}
\StringTok{  }\KeywordTok{theme}\NormalTok{(}\DataTypeTok{legend.position =} \StringTok{"none"}\NormalTok{) }\OperatorTok{+}
\StringTok{  }\KeywordTok{geom_hline}\NormalTok{(}\DataTypeTok{yintercept =} \FloatTok{.33}\NormalTok{, }\DataTypeTok{col=}\StringTok{'red'}\NormalTok{, }\DataTypeTok{lwd=}\DecValTok{1}\NormalTok{);}

\KeywordTok{grid.arrange}\NormalTok{(lp_violin, pl_violin, }\DataTypeTok{ncol=}\DecValTok{2}\NormalTok{)}
\end{Highlighting}
\end{Shaded}

\begin{center}\includegraphics{preProcessing_files/figure-latex/unnamed-chunk-82-1} \end{center}

\begin{Shaded}
\begin{Highlighting}[]
\CommentTok{#rm(ms, ss_prop)}
\end{Highlighting}
\end{Shaded}

Barplots + violinPlots with data from both tasks:

\begin{Shaded}
\begin{Highlighting}[]
\KeywordTok{rm}\NormalTok{(ms, df, ss_prop)}
\NormalTok{genTask <-}\StringTok{ }\KeywordTok{rbind}\NormalTok{(labelPicture, pictureLabel)}
\end{Highlighting}
\end{Shaded}

\begin{Shaded}
\begin{Highlighting}[]
\NormalTok{ms <-}\StringTok{ }\KeywordTok{aggregate}\NormalTok{(acc }\OperatorTok{~}\StringTok{ }\NormalTok{subjID}\OperatorTok{+}\NormalTok{frequency}\OperatorTok{+}\NormalTok{learning, }\DataTypeTok{data =}\NormalTok{ genTask[genTask}\OperatorTok{$}\NormalTok{rt}\OperatorTok{>}\DecValTok{100} \OperatorTok{&}
\StringTok{                                                                  }\NormalTok{genTask}\OperatorTok{$}\NormalTok{Experiment.Version}\OperatorTok{==}\NormalTok{ver2 }\OperatorTok{&}
\StringTok{                                         }\OperatorTok{!}\NormalTok{(genTask}\OperatorTok{$}\NormalTok{subjID }\OperatorTok\StringTok{ }\NormalTok{badsubjs),], mean)}

\NormalTok{ms}\OperatorTok{$}\NormalTok{frequency <-}\StringTok{ }\KeywordTok{as.factor}\NormalTok{(ms}\OperatorTok{$}\NormalTok{frequency)}
\NormalTok{plyr}\OperatorTok{::}\KeywordTok{revalue}\NormalTok{(ms}\OperatorTok{$}\NormalTok{frequency, }\KeywordTok{c}\NormalTok{(}\StringTok{"25"}\NormalTok{=}\StringTok{"low"}\NormalTok{))->}\StringTok{ }\NormalTok{ms}\OperatorTok{$}\NormalTok{frequency;}
\NormalTok{plyr}\OperatorTok{::}\KeywordTok{revalue}\NormalTok{(ms}\OperatorTok{$}\NormalTok{frequency, }\KeywordTok{c}\NormalTok{(}\StringTok{"75"}\NormalTok{=}\StringTok{"high"}\NormalTok{))->}\StringTok{ }\NormalTok{ms}\OperatorTok{$}\NormalTok{frequency;}

\KeywordTok{ggviolin}\NormalTok{(ms, }\DataTypeTok{x =} \StringTok{"frequency"}\NormalTok{, }\DataTypeTok{y =} \StringTok{"acc"}\NormalTok{, }\DataTypeTok{fill =} \StringTok{"frequency"}\NormalTok{,}
         \DataTypeTok{palette =} \KeywordTok{c}\NormalTok{(}\StringTok{"#00AFBB"}\NormalTok{, }\StringTok{"#E7B800"}\NormalTok{),}
         \DataTypeTok{add =} \StringTok{"boxplot"}\NormalTok{, }
         \DataTypeTok{add.params =} \KeywordTok{list}\NormalTok{(}\DataTypeTok{fill =} \StringTok{"white"}\NormalTok{),}
         \DataTypeTok{trim=}\OtherTok{TRUE}\NormalTok{) }\OperatorTok{+}
\StringTok{        }\KeywordTok{ggtitle}\NormalTok{(}\StringTok{'labelPictures + pictureLabels'}\NormalTok{) }\OperatorTok{+}\StringTok{ }
\StringTok{        }\KeywordTok{facet_grid}\NormalTok{( . }\OperatorTok{~}\StringTok{ }\NormalTok{learning) }\OperatorTok{+}
\StringTok{        }\KeywordTok{theme_pubclean}\NormalTok{()}\OperatorTok{+}
\StringTok{  }\KeywordTok{geom_hline}\NormalTok{(}\DataTypeTok{yintercept =} \FloatTok{.33}\NormalTok{, }\DataTypeTok{col=}\StringTok{'red'}\NormalTok{, }\DataTypeTok{lwd=}\DecValTok{1}\NormalTok{);}
\end{Highlighting}
\end{Shaded}

\includegraphics{preProcessing_files/figure-latex/unnamed-chunk-84-1.pdf}

\begin{Shaded}
\begin{Highlighting}[]
\NormalTok{ms <-}\StringTok{ }\KeywordTok{aggregate}\NormalTok{(acc }\OperatorTok{~}\StringTok{ }\NormalTok{subjID}\OperatorTok{+}\NormalTok{frequency}\OperatorTok{+}\NormalTok{learning, }\DataTypeTok{data=}\NormalTok{genTask[genTask}\OperatorTok{$}\NormalTok{rt}\OperatorTok{>}\DecValTok{100} \OperatorTok{&}
\StringTok{                                                                }\NormalTok{genTask}\OperatorTok{$}\NormalTok{Experiment.Version}\OperatorTok{==}\NormalTok{ver2 }\OperatorTok{&}
\StringTok{                                         }\OperatorTok{!}\NormalTok{(genTask}\OperatorTok{$}\NormalTok{subjID }\OperatorTok\StringTok{ }\NormalTok{badsubjs),], }\DataTypeTok{FUN=}\NormalTok{ mean)}

\NormalTok{df<-}\StringTok{ }\NormalTok{ms }\OperatorTok
\StringTok{  }\KeywordTok{group_by}\NormalTok{(frequency, learning)}\OperatorTok
\StringTok{  }\KeywordTok{summarise}\NormalTok{(}
    \DataTypeTok{mean =} \KeywordTok{mean}\NormalTok{(acc),}
    \DataTypeTok{sd =} \KeywordTok{sd}\NormalTok{(acc),}
    \DataTypeTok{n =} \KeywordTok{n}\NormalTok{()) }\OperatorTok
\StringTok{  }\KeywordTok{mutate}\NormalTok{( }\DataTypeTok{se=}\NormalTok{sd}\OperatorTok{/}\KeywordTok{sqrt}\NormalTok{(n))  }\OperatorTok\StringTok{ }
\StringTok{  }\KeywordTok{mutate}\NormalTok{( }\DataTypeTok{ci=}\NormalTok{se }\OperatorTok{*}\StringTok{ }\KeywordTok{qt}\NormalTok{((}\DecValTok{1}\FloatTok{-0.05}\NormalTok{)}\OperatorTok{/}\DecValTok{2} \OperatorTok{+}\StringTok{ }\FloatTok{.5}\NormalTok{, n}\DecValTok{-1}\NormalTok{))}

\NormalTok{df}\OperatorTok{$}\NormalTok{frequency <-}\StringTok{ }\KeywordTok{as.factor}\NormalTok{(df}\OperatorTok{$}\NormalTok{frequency)}
\NormalTok{plyr}\OperatorTok{::}\KeywordTok{revalue}\NormalTok{(df}\OperatorTok{$}\NormalTok{frequency, }\KeywordTok{c}\NormalTok{(}\StringTok{"25"}\NormalTok{=}\StringTok{"low"}\NormalTok{))->}\StringTok{ }\NormalTok{df}\OperatorTok{$}\NormalTok{frequency;}
\NormalTok{plyr}\OperatorTok{::}\KeywordTok{revalue}\NormalTok{(df}\OperatorTok{$}\NormalTok{frequency, }\KeywordTok{c}\NormalTok{(}\StringTok{"75"}\NormalTok{=}\StringTok{"high"}\NormalTok{))->}\StringTok{ }\NormalTok{df}\OperatorTok{$}\NormalTok{frequency;}


\KeywordTok{ggplot}\NormalTok{(}\KeywordTok{aes}\NormalTok{(}\DataTypeTok{x =}\NormalTok{ frequency, }\DataTypeTok{y =}\NormalTok{ mean, }\DataTypeTok{fill =}\NormalTok{ frequency), }\DataTypeTok{data =}\NormalTok{ df) }\OperatorTok{+}
\StringTok{  }\KeywordTok{facet_grid}\NormalTok{( . }\OperatorTok{~}\StringTok{ }\NormalTok{learning) }\OperatorTok{+}
\StringTok{  }\KeywordTok{geom_bar}\NormalTok{(}\DataTypeTok{stat =} \StringTok{"identity"}\NormalTok{, }\DataTypeTok{color=}\StringTok{'white'}\NormalTok{, }\DataTypeTok{position=}\KeywordTok{position_dodge}\NormalTok{(), }\DataTypeTok{size=}\FloatTok{1.2}\NormalTok{) }\OperatorTok{+}
\StringTok{  }\KeywordTok{geom_errorbar}\NormalTok{(}\KeywordTok{aes}\NormalTok{(}\DataTypeTok{ymin=}\NormalTok{mean}\OperatorTok{-}\NormalTok{se, }\DataTypeTok{ymax=}\NormalTok{mean}\OperatorTok{+}\NormalTok{se), }\DataTypeTok{width=}\NormalTok{.}\DecValTok{15}\NormalTok{, }\DataTypeTok{size=}\DecValTok{1}\NormalTok{,}\DataTypeTok{position=}\KeywordTok{position_dodge}\NormalTok{(.}\DecValTok{9}\NormalTok{)) }\OperatorTok{+}
\StringTok{  }\KeywordTok{ylab}\NormalTok{(}\StringTok{"Accuracy "}\NormalTok{) }\OperatorTok{+}
\StringTok{  }\KeywordTok{xlab}\NormalTok{(}\StringTok{"frequency"}\NormalTok{) }\OperatorTok{+}
\StringTok{  }\KeywordTok{ggtitle}\NormalTok{(}\StringTok{"labelPicture"}\NormalTok{) }\OperatorTok{+}
\StringTok{  }\KeywordTok{ggtitle}\NormalTok{(}\StringTok{'picturelabels + labelpictures'}\NormalTok{) }\OperatorTok{+}
\StringTok{  }\KeywordTok{coord_cartesian}\NormalTok{(}\DataTypeTok{ylim =} \KeywordTok{c}\NormalTok{(}\DecValTok{0}\NormalTok{, }\DecValTok{1}\NormalTok{))}\OperatorTok{+}
\StringTok{  }\NormalTok{ggpubr}\OperatorTok{::}\KeywordTok{theme_pubclean}\NormalTok{() }\OperatorTok{+}\StringTok{ }
\StringTok{  }\KeywordTok{theme}\NormalTok{(}\DataTypeTok{legend.position=}\StringTok{"bottom"}\NormalTok{, }\DataTypeTok{legend.title =} \KeywordTok{element_blank}\NormalTok{()) }\OperatorTok{+}
\StringTok{  }\KeywordTok{theme}\NormalTok{(}\DataTypeTok{text =} \KeywordTok{element_text}\NormalTok{(}\DataTypeTok{size=}\DecValTok{10}\NormalTok{)) }\OperatorTok{+}
\StringTok{  }\KeywordTok{geom_hline}\NormalTok{(}\DataTypeTok{yintercept =} \FloatTok{.33}\NormalTok{, }\DataTypeTok{col=}\StringTok{'red'}\NormalTok{, }\DataTypeTok{lwd=}\DecValTok{1}\NormalTok{);}
\end{Highlighting}
\end{Shaded}

\includegraphics{preProcessing_files/figure-latex/unnamed-chunk-85-1.pdf}

\hypertarget{task-3-contingency-judgement}{%
\section{Task 3: Contingency
judgement}\label{task-3-contingency-judgement}}

\begin{Shaded}
\begin{Highlighting}[]
\KeywordTok{length}\NormalTok{(}\KeywordTok{unique}\NormalTok{(contingencyJudgement}\OperatorTok{$}\NormalTok{subjID))}
\end{Highlighting}
\end{Shaded}

\begin{verbatim}
## [1] 120
\end{verbatim}

\begin{Shaded}
\begin{Highlighting}[]
\NormalTok{fl<-}\StringTok{ }\KeywordTok{length}\NormalTok{(}\KeywordTok{unique}\NormalTok{(contingencyJudgement[contingencyJudgement}\OperatorTok{$}\NormalTok{learning}\OperatorTok{==}\StringTok{'FL'} \OperatorTok{&}\StringTok{ }\NormalTok{contingencyJudgement}\OperatorTok{$}\NormalTok{Experiment.Version}\OperatorTok{==}\NormalTok{ver2,]}\OperatorTok{$}\NormalTok{subjID))}

\NormalTok{lf<-}\StringTok{ }\KeywordTok{length}\NormalTok{(}\KeywordTok{unique}\NormalTok{(contingencyJudgement[contingencyJudgement}\OperatorTok{$}\NormalTok{learning}\OperatorTok{==}\StringTok{'LF'} \OperatorTok{&}\StringTok{ }\NormalTok{contingencyJudgement}\OperatorTok{$}\NormalTok{Experiment.Version}\OperatorTok{==}\NormalTok{ver2,]}\OperatorTok{$}\NormalTok{subjID))}

\NormalTok{fl}
\end{Highlighting}
\end{Shaded}

\begin{verbatim}
## [1] 41
\end{verbatim}

\begin{Shaded}
\begin{Highlighting}[]
\NormalTok{lf}
\end{Highlighting}
\end{Shaded}

\begin{verbatim}
## [1] 39
\end{verbatim}

We have 41 for feature-label learning, and 39 for label-feature
learning.

\begin{Shaded}
\begin{Highlighting}[]
\KeywordTok{rm}\NormalTok{(fl,lf)}
\NormalTok{conjudge <-}\StringTok{ }\NormalTok{contingencyJudgement[}\OperatorTok{!}\NormalTok{(}\KeywordTok{is.na}\NormalTok{(contingencyJudgement}\OperatorTok{$}\NormalTok{resp)),]}
\NormalTok{n<-}\StringTok{ }\KeywordTok{length}\NormalTok{(}\KeywordTok{unique}\NormalTok{(conjudge}\OperatorTok{$}\NormalTok{subjID))}
\NormalTok{nrows <-}\StringTok{ }\NormalTok{(}\KeywordTok{nrow}\NormalTok{(contingencyJudgement)) }\OperatorTok{-}\StringTok{ }\NormalTok{(}\KeywordTok{nrow}\NormalTok{(conjudge))}

\KeywordTok{sort}\NormalTok{(}\KeywordTok{unique}\NormalTok{(conjudge}\OperatorTok{$}\NormalTok{subjID))->}\StringTok{ }\NormalTok{subjs;}
\KeywordTok{sort}\NormalTok{(}\KeywordTok{unique}\NormalTok{(contingencyJudgement}\OperatorTok{$}\NormalTok{subjID)) ->totsubjs;}

\NormalTok{subjmissed<-}\StringTok{ }\KeywordTok{setdiff}\NormalTok{(totsubjs, subjs);}

\NormalTok{badsubjs <-}\StringTok{ }\KeywordTok{c}\NormalTok{(badsubjs, subjmissed)}
\NormalTok{badsubjs <-}\StringTok{ }\KeywordTok{unique}\NormalTok{(badsubjs)}
\end{Highlighting}
\end{Shaded}

We have 111 participants in this task, so -9, and we have missed 559
over the total 2880, that is 19.4097222. The subject(s) that missed
completely the task is/are: 1414932, 1420171, 1420199, 1422475, 1431960,
1431997, 1459020, 1459057, 1459078.

\begin{Shaded}
\begin{Highlighting}[]
\KeywordTok{par}\NormalTok{(}\DataTypeTok{mfrow=}\KeywordTok{c}\NormalTok{(}\DecValTok{1}\NormalTok{,}\DecValTok{2}\NormalTok{))}
\KeywordTok{hist}\NormalTok{(conjudge[conjudge}\OperatorTok{$}\NormalTok{rt}\OperatorTok{<}\DecValTok{1500} \OperatorTok{&}\StringTok{ }\OperatorTok{!}\NormalTok{(conjudge}\OperatorTok{$}\NormalTok{subjID }\OperatorTok\StringTok{ }\NormalTok{badsubjs),]}\OperatorTok{$}\NormalTok{rt, }\DataTypeTok{main =} \StringTok{'rt < 1500ms'}\NormalTok{, }\DataTypeTok{xlab =} \StringTok{'trials'}\NormalTok{);}
\KeywordTok{hist}\NormalTok{(conjudge[conjudge}\OperatorTok{$}\NormalTok{rt}\OperatorTok{>}\DecValTok{3000} \OperatorTok{&}\StringTok{ }\OperatorTok{!}\NormalTok{(conjudge}\OperatorTok{$}\NormalTok{subjID }\OperatorTok\StringTok{ }\NormalTok{badsubjs),]}\OperatorTok{$}\NormalTok{rt, }\DataTypeTok{main =} \StringTok{'rt > 3000ms'}\NormalTok{, }\DataTypeTok{xlab =} \StringTok{'trials'}\NormalTok{);}
\end{Highlighting}
\end{Shaded}

\includegraphics{preProcessing_files/figure-latex/unnamed-chunk-88-1.pdf}

\begin{Shaded}
\begin{Highlighting}[]
\KeywordTok{par}\NormalTok{(}\DataTypeTok{mfrow=}\KeywordTok{c}\NormalTok{(}\DecValTok{1}\NormalTok{,}\DecValTok{1}\NormalTok{))}
\end{Highlighting}
\end{Shaded}

Resp is coded as factor, need to correct this:

\begin{Shaded}
\begin{Highlighting}[]
\KeywordTok{as.numeric}\NormalTok{(}\KeywordTok{levels}\NormalTok{(conjudge}\OperatorTok{$}\NormalTok{resp))[conjudge}\OperatorTok{$}\NormalTok{resp]->}\StringTok{ }\NormalTok{conjudge}\OperatorTok{$}\NormalTok{resp}
\end{Highlighting}
\end{Shaded}

\begin{Shaded}
\begin{Highlighting}[]
\KeywordTok{hist}\NormalTok{(conjudge[}\OperatorTok{!}\NormalTok{(conjudge}\OperatorTok{$}\NormalTok{subjID }\OperatorTok\StringTok{ }\NormalTok{badsubjs),]}\OperatorTok{$}\NormalTok{resp, }\DataTypeTok{main =} \StringTok{'resp distribution'}\NormalTok{, }\DataTypeTok{xlab =} \StringTok{'choices'}\NormalTok{)}
\end{Highlighting}
\end{Shaded}

\includegraphics{preProcessing_files/figure-latex/histogram-1.pdf}

Ok, here we don't have right or wrong answers, but we are more
interested in take a look how the participants rated the fribble label
association:

\begin{Shaded}
\begin{Highlighting}[]
\KeywordTok{aggregate}\NormalTok{(resp }\OperatorTok{~}\StringTok{ }\NormalTok{category, }\DataTypeTok{data =}\NormalTok{ conjudge[}\OperatorTok{!}\NormalTok{(conjudge}\OperatorTok{$}\NormalTok{subjID }\OperatorTok\StringTok{ }\NormalTok{badsubjs),], }\DataTypeTok{FUN =}\NormalTok{ mean)}
\end{Highlighting}
\end{Shaded}

\begin{verbatim}
##   category       resp
## 1        1 -12.119084
## 2        2  -6.564636
## 3        3 -10.635821
\end{verbatim}

Okay, in this task one fribble was presented along with a label. The
association between the fribble presented and the label could have been
correct, or wrong. In this case then accuracy column does \textbf{not}
refer to the participants' accuracy, but rather to the fribble-label
pair presented. This should be therefore necessarily equal to the chance
level, i.e, around 33\%, of course this number is dependent by the
number of datapoints left without no-responses because we filtered out
those.

\begin{Shaded}
\begin{Highlighting}[]
\NormalTok{conjudge}\OperatorTok{$}\NormalTok{acc <-}\StringTok{ }\DecValTok{0}\NormalTok{;}
\NormalTok{conjudge[conjudge}\OperatorTok{$}\NormalTok{category}\OperatorTok{==}\DecValTok{1} \OperatorTok{&}\StringTok{ }\NormalTok{conjudge}\OperatorTok{$}\NormalTok{label}\OperatorTok{==}\StringTok{'dep'}\NormalTok{,]}\OperatorTok{$}\NormalTok{acc <-}\StringTok{ }\DecValTok{1}\NormalTok{;}
\NormalTok{conjudge[conjudge}\OperatorTok{$}\NormalTok{category}\OperatorTok{==}\DecValTok{2} \OperatorTok{&}\StringTok{ }\NormalTok{conjudge}\OperatorTok{$}\NormalTok{label}\OperatorTok{==}\StringTok{'bim'}\NormalTok{,]}\OperatorTok{$}\NormalTok{acc <-}\StringTok{ }\DecValTok{1}\NormalTok{;}
\NormalTok{conjudge[conjudge}\OperatorTok{$}\NormalTok{category}\OperatorTok{==}\DecValTok{3} \OperatorTok{&}\StringTok{ }\NormalTok{conjudge}\OperatorTok{$}\NormalTok{label}\OperatorTok{==}\StringTok{'tob'}\NormalTok{,]}\OperatorTok{$}\NormalTok{acc <-}\StringTok{ }\DecValTok{1}\NormalTok{;}
\end{Highlighting}
\end{Shaded}

\begin{Shaded}
\begin{Highlighting}[]
\KeywordTok{mean}\NormalTok{(conjudge[}\OperatorTok{!}\NormalTok{(conjudge}\OperatorTok{$}\NormalTok{subjID }\OperatorTok\StringTok{ }\NormalTok{badsubjs),]}\OperatorTok{$}\NormalTok{acc)}
\end{Highlighting}
\end{Shaded}

\begin{verbatim}
## [1] 0.3523524
\end{verbatim}

Quite there, everything good.

\begin{Shaded}
\begin{Highlighting}[]
\NormalTok{respDistr<-}\KeywordTok{summarySEwithin}\NormalTok{(}\DataTypeTok{data =}\NormalTok{ conjudge[}\OperatorTok{!}\NormalTok{(conjudge}\OperatorTok{$}\NormalTok{subjID }\OperatorTok\StringTok{ }\NormalTok{badsubjs),], }\DataTypeTok{measurevar =} \StringTok{"resp"}\NormalTok{, }\DataTypeTok{betweenvars =} \StringTok{"learning"}\NormalTok{, }\DataTypeTok{withinvars =} \KeywordTok{c}\NormalTok{(}\StringTok{"frequency"}\NormalTok{, }\StringTok{"category"}\NormalTok{, }\StringTok{"label"}\NormalTok{), }\DataTypeTok{idvar =} \StringTok{"subjID"}\NormalTok{, }\DataTypeTok{conf.interval =} \FloatTok{.95}\NormalTok{)}
\end{Highlighting}
\end{Shaded}

\begin{verbatim}
## Automatically converting the following non-factors to factors: frequency, category
\end{verbatim}

\begin{verbatim}
## Loading required package: plyr
\end{verbatim}

\begin{verbatim}
## ------------------------------------------------------------------------------
\end{verbatim}

\begin{verbatim}
## You have loaded plyr after dplyr - this is likely to cause problems.
## If you need functions from both plyr and dplyr, please load plyr first, then dplyr:
## library(plyr); library(dplyr)
\end{verbatim}

\begin{verbatim}
## ------------------------------------------------------------------------------
\end{verbatim}

\begin{verbatim}
## 
## Attaching package: 'plyr'
\end{verbatim}

\begin{verbatim}
## The following object is masked from 'package:ggpubr':
## 
##     mutate
\end{verbatim}

\begin{verbatim}
## The following objects are masked from 'package:dplyr':
## 
##     arrange, count, desc, failwith, id, mutate, rename, summarise,
##     summarize
\end{verbatim}

\begin{verbatim}
## The following object is masked from 'package:purrr':
## 
##     compact
\end{verbatim}

\begin{Shaded}
\begin{Highlighting}[]
\NormalTok{respDistr}
\end{Highlighting}
\end{Shaded}

\begin{verbatim}
##    learning frequency category label  N        resp   resp_norm       sd
## 1        FL        25        1   bim 67 -31.3582090 -14.1087315 51.86845
## 2        FL        25        1   dep 45   2.3333333 -28.6281805 56.62847
## 3        FL        25        1   tob 59  -0.6271186  16.7903079 59.40639
## 4        FL        25        2   bim 67  28.2537313   0.3038255 71.42933
## 5        FL        25        2   dep 60 -13.3166667   3.6104924 54.78085
## 6        FL        25        2   tob 53 -31.2075472  -2.6381954 43.77948
## 7        FL        25        3   bim 46 -33.7608696  -0.5991529 56.19839
## 8        FL        25        3   dep 71 -48.0281690 -31.5804323 48.20588
## 9        FL        25        3   tob 66  14.9848485  -8.1380523 69.60998
## 10       FL        75        1   bim 72 -29.7916667 -13.2997975 66.07554
## 11       FL        75        1   dep 47  24.8085106  -6.0097113 59.45004
## 12       FL        75        1   tob 66 -56.7272727 -39.7207891 50.47919
## 13       FL        75        2   bim 70  66.3142857  37.7407473 61.56664
## 14       FL        75        2   dep 65 -60.9230769 -39.6869158 47.20973
## 15       FL        75        2   tob 54 -53.1481481 -25.3652798 36.29847
## 16       FL        75        3   bim 47 -68.7872340 -34.7896631 43.58048
## 17       FL        75        3   dep 71 -49.6338028 -37.7716413 44.79396
## 18       FL        75        3   tob 70  63.1857143  39.3122969 60.83277
## 19       LF        25        1   bim 52 -24.0192308 -13.4298651 46.36911
## 20       LF        25        1   dep 56   5.2857143 -21.8581395 69.28297
## 21       LF        25        1   tob 40 -11.2750000  -6.3878238 60.86102
## 22       LF        25        2   bim 56  10.0714286 -23.3775019 51.38502
## 23       LF        25        2   dep 39  -6.8205128   4.7026189 57.82144
## 24       LF        25        2   tob 57 -42.3684211 -26.3358630 47.08686
## 25       LF        25        3   bim 38  11.1578947  26.3661505 63.27118
## 26       LF        25        3   dep 51 -38.2352941 -32.4228652 53.64881
## 27       LF        25        3   tob 54   5.3518519 -23.9193328 54.61213
## 28       LF        75        1   bim 54 -39.1851852 -28.4874570 36.67076
## 29       LF        75        1   dep 56  69.3035714  34.8370718 56.31310
## 30       LF        75        1   tob 41 -37.6341463 -31.9129596 61.39834
## 31       LF        75        2   bim 57  66.9473684  30.2636915 64.75691
## 32       LF        75        2   dep 37 -47.3783784 -36.2311056 61.97356
## 33       LF        75        2   tob 58 -27.8620690 -12.5594870 59.68045
## 34       LF        75        3   bim 39 -56.4615385 -41.2231689 46.00065
## 35       LF        75        3   dep 57 -14.8421053 -13.1330737 57.65721
## 36       LF        75        3   tob 60  57.7833333  25.8176773 65.50872
##           se        ci
## 1   6.336740 12.651710
## 2   8.441674 17.013075
## 3   7.734053 15.481389
## 4   8.726481 17.422982
## 5   7.072178 14.151395
## 6   6.013574 12.067118
## 7   8.286001 16.688863
## 8   5.720986 11.410147
## 9   8.568396 17.112268
## 10  7.787078 15.526992
## 11  8.671680 17.455186
## 12  6.213559 12.409334
## 13  7.358621 14.680047
## 14  5.855647 11.697999
## 15  4.939596  9.907574
## 16  6.356867 12.795709
## 17  5.316065 10.602558
## 18  7.270906 14.505061
## 19  6.430238 12.909242
## 20  9.258327 18.554101
## 21  9.622972 19.464299
## 22  6.866612 13.760999
## 23  9.258840 18.743542
## 24  6.236806 12.493825
## 25 10.263941 20.796721
## 26  7.512336 15.088970
## 27  7.431769 14.906241
## 28  4.990258 10.009189
## 29  7.525155 15.080748
## 30  9.588809 19.379707
## 31  8.577262 17.182321
## 32 10.188391 20.663014
## 33  7.836426 15.692173
## 34  7.365999 14.911685
## 35  7.636884 15.298517
## 36  8.457139 16.922696
\end{verbatim}

\hypertarget{plot-mean-responses}{%
\subsection{plot mean responses}\label{plot-mean-responses}}

\begin{Shaded}
\begin{Highlighting}[]
\NormalTok{plyr}\OperatorTok{::}\KeywordTok{revalue}\NormalTok{(respDistr}\OperatorTok{$}\NormalTok{frequency, }\KeywordTok{c}\NormalTok{(}\StringTok{"25"}\NormalTok{=}\StringTok{"low"}\NormalTok{))->}\StringTok{ }\NormalTok{respDistr}\OperatorTok{$}\NormalTok{frequency;}
\NormalTok{plyr}\OperatorTok{::}\KeywordTok{revalue}\NormalTok{(respDistr}\OperatorTok{$}\NormalTok{frequency, }\KeywordTok{c}\NormalTok{(}\StringTok{"75"}\NormalTok{=}\StringTok{"high"}\NormalTok{))->}\StringTok{ }\NormalTok{respDistr}\OperatorTok{$}\NormalTok{frequency;}

\NormalTok{lollipop<-}\KeywordTok{ggdotchart}\NormalTok{(respDistr, }\DataTypeTok{x =} \StringTok{"label"}\NormalTok{, }\DataTypeTok{y =} \StringTok{"resp"}\NormalTok{,}
           \DataTypeTok{color =} \StringTok{"category"}\NormalTok{,                                }\CommentTok{# Color by groups}
           \DataTypeTok{palette =} \KeywordTok{c}\NormalTok{(}\StringTok{"#00AFBB"}\NormalTok{, }\StringTok{"#E7B800"}\NormalTok{, }\StringTok{"#FC4E07"}\NormalTok{), }\CommentTok{# Custom color palette}
           \DataTypeTok{add =} \StringTok{"segments"}\NormalTok{,                             }\CommentTok{# Add segments from y = 0 to dots}
           \DataTypeTok{rotate =}\NormalTok{ T,}
           \DataTypeTok{add.params =} \KeywordTok{list}\NormalTok{(}\DataTypeTok{color =} \StringTok{"lightgray"}\NormalTok{, }\DataTypeTok{size =} \DecValTok{2}\NormalTok{), }\CommentTok{# Change segment color and size}
           \DataTypeTok{group =} \StringTok{"category"}\NormalTok{,                                }\CommentTok{# Order by groups}
           \DataTypeTok{dot.size =} \DecValTok{10}\NormalTok{,                                 }\CommentTok{# Large dot size}
           \DataTypeTok{label =} \KeywordTok{round}\NormalTok{(respDistr}\OperatorTok{$}\NormalTok{resp,}\DecValTok{1}\NormalTok{),                        }\CommentTok{# Add mpg values as dot labels}
           \DataTypeTok{font.label =} \KeywordTok{list}\NormalTok{(}\DataTypeTok{color =} \StringTok{"white"}\NormalTok{, }\DataTypeTok{size =} \DecValTok{9}\NormalTok{, }
                             \DataTypeTok{vjust =} \FloatTok{0.5}\NormalTok{),               }\CommentTok{# Adjust label parameters}
           \DataTypeTok{ggtheme =} \KeywordTok{theme_pubr}\NormalTok{()                        }\CommentTok{# ggplot2 theme}
\NormalTok{           )}\OperatorTok{+}\StringTok{ }\KeywordTok{facet_grid}\NormalTok{( frequency }\OperatorTok{~}\StringTok{ }\NormalTok{learning) }\OperatorTok{+}
\StringTok{  }\KeywordTok{geom_hline}\NormalTok{(}\DataTypeTok{yintercept =} \DecValTok{0}\NormalTok{, }\DataTypeTok{linetype =} \DecValTok{2}\NormalTok{, }\DataTypeTok{color =} \StringTok{"lightgray"}\NormalTok{) }

\NormalTok{lollipop}
\end{Highlighting}
\end{Shaded}

\begin{center}\includegraphics{preProcessing_files/figure-latex/unnamed-chunk-93-1} \end{center}

Plot to compare with RW weights:

\begin{Shaded}
\begin{Highlighting}[]
\NormalTok{plyr}\OperatorTok{::}\KeywordTok{revalue}\NormalTok{(}\KeywordTok{as.factor}\NormalTok{(conjudge}\OperatorTok{$}\NormalTok{frequency), }\KeywordTok{c}\NormalTok{(}\StringTok{"25"}\NormalTok{=}\StringTok{"low"}\NormalTok{))->}\StringTok{ }\NormalTok{conjudge}\OperatorTok{$}\NormalTok{frequency;}
\NormalTok{plyr}\OperatorTok{::}\KeywordTok{revalue}\NormalTok{(}\KeywordTok{as.factor}\NormalTok{(conjudge}\OperatorTok{$}\NormalTok{frequency), }\KeywordTok{c}\NormalTok{(}\StringTok{"75"}\NormalTok{=}\StringTok{"high"}\NormalTok{))->}\StringTok{ }\NormalTok{conjudge}\OperatorTok{$}\NormalTok{frequency;}
\end{Highlighting}
\end{Shaded}

For each learning condition, look at their average score for each of the
6 combinations of frequency and type:

High frequency:

\begin{Shaded}
\begin{Highlighting}[]
\NormalTok{highFreqFL<-}\KeywordTok{data.frame}\NormalTok{(}
           \DataTypeTok{learning =} \KeywordTok{rep}\NormalTok{(}\StringTok{"FL"}\NormalTok{,}\DecValTok{9}\NormalTok{),}
           \DataTypeTok{frequency =} \KeywordTok{rep}\NormalTok{(}\StringTok{"high"}\NormalTok{,}\DecValTok{9}\NormalTok{),}
           \DataTypeTok{type =} \KeywordTok{c}\NormalTok{(}\KeywordTok{rep}\NormalTok{(}\StringTok{"match"}\NormalTok{,}\DecValTok{3}\NormalTok{), }
                    \KeywordTok{rep}\NormalTok{(}\StringTok{"mismatch-type1"}\NormalTok{,}\DecValTok{3}\NormalTok{), }
                    \KeywordTok{rep}\NormalTok{(}\StringTok{"mismatch-type2"}\NormalTok{,}\DecValTok{3}\NormalTok{)),}
           \DataTypeTok{label =} \KeywordTok{c}\NormalTok{(}\StringTok{"dep_cat1"}\NormalTok{, }\StringTok{"bim_cat2"}\NormalTok{, }\StringTok{"tob_cat3"}\NormalTok{),}
           \DataTypeTok{fribble =} \KeywordTok{c}\NormalTok{(}\FloatTok{1.1}\NormalTok{,}\FloatTok{2.1}\NormalTok{,}\FloatTok{3.1}\NormalTok{,}
                       \FloatTok{3.1}\NormalTok{,}\FloatTok{1.1}\NormalTok{,}\FloatTok{2.1}\NormalTok{,}
                       \FloatTok{2.1}\NormalTok{,}\FloatTok{3.1}\NormalTok{,}\FloatTok{1.1}\NormalTok{),}
           \DataTypeTok{fribbleCategory =} \KeywordTok{c}\NormalTok{(}\StringTok{"cat1"}\NormalTok{, }\StringTok{"cat2"}\NormalTok{, }\StringTok{"cat3"}\NormalTok{, }\CommentTok{#match}
                        \StringTok{"cat3"}\NormalTok{, }\StringTok{"cat1"}\NormalTok{, }\StringTok{"cat2"}\NormalTok{, }\CommentTok{#mis-type1}
                        \StringTok{"cat2"}\NormalTok{, }\StringTok{"cat3"}\NormalTok{, }\StringTok{"cat1"}\NormalTok{)) }\CommentTok{#mis-type2}

\NormalTok{highFreqLF<-}\KeywordTok{data.frame}\NormalTok{(}
           \DataTypeTok{learning =} \KeywordTok{rep}\NormalTok{(}\StringTok{"LF"}\NormalTok{,}\DecValTok{9}\NormalTok{),}
           \DataTypeTok{frequency =} \KeywordTok{rep}\NormalTok{(}\StringTok{"high"}\NormalTok{,}\DecValTok{9}\NormalTok{),}
           \DataTypeTok{type =} \KeywordTok{c}\NormalTok{(}\KeywordTok{rep}\NormalTok{(}\StringTok{"match"}\NormalTok{,}\DecValTok{3}\NormalTok{), }
                    \KeywordTok{rep}\NormalTok{(}\StringTok{"mismatch-type1"}\NormalTok{,}\DecValTok{3}\NormalTok{), }
                    \KeywordTok{rep}\NormalTok{(}\StringTok{"mismatch-type2"}\NormalTok{,}\DecValTok{3}\NormalTok{)),}
           \DataTypeTok{label =} \KeywordTok{c}\NormalTok{(}\StringTok{"dep_cat1"}\NormalTok{, }\StringTok{"bim_cat2"}\NormalTok{, }\StringTok{"tob_cat3"}\NormalTok{),}
           \DataTypeTok{fribble =} \KeywordTok{c}\NormalTok{(}\FloatTok{1.1}\NormalTok{,}\FloatTok{2.1}\NormalTok{,}\FloatTok{3.1}\NormalTok{,}
                       \FloatTok{3.1}\NormalTok{,}\FloatTok{1.1}\NormalTok{,}\FloatTok{2.1}\NormalTok{,}
                       \FloatTok{2.1}\NormalTok{,}\FloatTok{3.1}\NormalTok{,}\FloatTok{1.1}\NormalTok{),}
           \DataTypeTok{fribbleCategory =} \KeywordTok{c}\NormalTok{(}\StringTok{"cat1"}\NormalTok{, }\StringTok{"cat2"}\NormalTok{, }\StringTok{"cat3"}\NormalTok{, }\CommentTok{#match}
                        \StringTok{"cat3"}\NormalTok{, }\StringTok{"cat1"}\NormalTok{, }\StringTok{"cat2"}\NormalTok{, }\CommentTok{#mis-type1}
                        \StringTok{"cat2"}\NormalTok{, }\StringTok{"cat3"}\NormalTok{, }\StringTok{"cat1"}\NormalTok{)) }\CommentTok{#mis-type2}

\KeywordTok{rbind}\NormalTok{(highFreqFL, highFreqLF)->}\StringTok{ }\NormalTok{highFreq}
\KeywordTok{rm}\NormalTok{(highFreqFL, highFreqLF)}
\end{Highlighting}
\end{Shaded}

Okay, let's fill each row:

\begin{Shaded}
\begin{Highlighting}[]
\NormalTok{resp <-}\StringTok{ }\KeywordTok{c}\NormalTok{(}
\CommentTok{#-----------------------------------------FL LEARNING}
  \CommentTok{# ROW 1                                              #MATCH}
  \KeywordTok{mean}\NormalTok{(conjudge[conjudge}\OperatorTok{$}\NormalTok{learning}\OperatorTok{==}\StringTok{"FL"} \OperatorTok{&}\StringTok{ }
\StringTok{           }\NormalTok{conjudge}\OperatorTok{$}\NormalTok{frequency}\OperatorTok{==}\StringTok{"high"} \OperatorTok{&}\StringTok{ }
\StringTok{           }\NormalTok{conjudge}\OperatorTok{$}\NormalTok{label}\OperatorTok{==}\StringTok{'dep'} \OperatorTok{&}\StringTok{ }
\StringTok{           }\NormalTok{conjudge}\OperatorTok{$}\NormalTok{category}\OperatorTok{==}\DecValTok{1} \OperatorTok{&}
\StringTok{           }\OperatorTok{!}\NormalTok{(conjudge}\OperatorTok{$}\NormalTok{subjID }\OperatorTok\StringTok{ }\NormalTok{badsubjs),]}\OperatorTok{$}\NormalTok{resp),}
  \CommentTok{# ROW 2}
  \KeywordTok{mean}\NormalTok{(conjudge[conjudge}\OperatorTok{$}\NormalTok{learning}\OperatorTok{==}\StringTok{"FL"} \OperatorTok{&}
\StringTok{           }\NormalTok{conjudge}\OperatorTok{$}\NormalTok{frequency}\OperatorTok{==}\StringTok{"high"} \OperatorTok{&}\StringTok{ }
\StringTok{           }\NormalTok{conjudge}\OperatorTok{$}\NormalTok{label}\OperatorTok{==}\StringTok{'bim'} \OperatorTok{&}\StringTok{ }
\StringTok{           }\NormalTok{conjudge}\OperatorTok{$}\NormalTok{category}\OperatorTok{==}\DecValTok{2} \OperatorTok{&}
\StringTok{           }\OperatorTok{!}\NormalTok{(conjudge}\OperatorTok{$}\NormalTok{subjID }\OperatorTok\StringTok{ }\NormalTok{badsubjs),]}\OperatorTok{$}\NormalTok{resp),}
  \CommentTok{# ROW 3}
  \KeywordTok{mean}\NormalTok{(conjudge[conjudge}\OperatorTok{$}\NormalTok{learning}\OperatorTok{==}\StringTok{"FL"} \OperatorTok{&}
\StringTok{           }\NormalTok{conjudge}\OperatorTok{$}\NormalTok{frequency}\OperatorTok{==}\StringTok{"high"} \OperatorTok{&}\StringTok{ }
\StringTok{           }\NormalTok{conjudge}\OperatorTok{$}\NormalTok{label}\OperatorTok{==}\StringTok{'tob'} \OperatorTok{&}\StringTok{ }
\StringTok{           }\NormalTok{conjudge}\OperatorTok{$}\NormalTok{category}\OperatorTok{==}\DecValTok{3} \OperatorTok{&}
\StringTok{           }\OperatorTok{!}\NormalTok{(conjudge}\OperatorTok{$}\NormalTok{subjID }\OperatorTok\StringTok{ }\NormalTok{badsubjs),]}\OperatorTok{$}\NormalTok{resp),}
  \CommentTok{# ROW 4                                              #MISMATCH -TYPE1}
  \KeywordTok{mean}\NormalTok{(conjudge[conjudge}\OperatorTok{$}\NormalTok{learning}\OperatorTok{==}\StringTok{"FL"} \OperatorTok{&}
\StringTok{           }\NormalTok{conjudge}\OperatorTok{$}\NormalTok{frequency}\OperatorTok{==}\StringTok{"high"} \OperatorTok{&}\StringTok{ }
\StringTok{           }\NormalTok{conjudge}\OperatorTok{$}\NormalTok{label}\OperatorTok{==}\StringTok{'dep'} \OperatorTok{&}\StringTok{ }
\StringTok{           }\NormalTok{conjudge}\OperatorTok{$}\NormalTok{category}\OperatorTok{==}\DecValTok{3} \OperatorTok{&}
\StringTok{           }\OperatorTok{!}\NormalTok{(conjudge}\OperatorTok{$}\NormalTok{subjID }\OperatorTok\StringTok{ }\NormalTok{badsubjs),]}\OperatorTok{$}\NormalTok{resp),}
  \CommentTok{# ROW 5}
  \KeywordTok{mean}\NormalTok{(conjudge[conjudge}\OperatorTok{$}\NormalTok{learning}\OperatorTok{==}\StringTok{"FL"} \OperatorTok{&}
\StringTok{           }\NormalTok{conjudge}\OperatorTok{$}\NormalTok{frequency}\OperatorTok{==}\StringTok{"high"} \OperatorTok{&}\StringTok{ }
\StringTok{           }\NormalTok{conjudge}\OperatorTok{$}\NormalTok{label}\OperatorTok{==}\StringTok{'bim'} \OperatorTok{&}\StringTok{ }
\StringTok{           }\NormalTok{conjudge}\OperatorTok{$}\NormalTok{category}\OperatorTok{==}\DecValTok{1} \OperatorTok{&}
\StringTok{           }\OperatorTok{!}\NormalTok{(conjudge}\OperatorTok{$}\NormalTok{subjID }\OperatorTok\StringTok{ }\NormalTok{badsubjs),]}\OperatorTok{$}\NormalTok{resp),}
  \CommentTok{# ROW 6}
  \KeywordTok{mean}\NormalTok{(conjudge[conjudge}\OperatorTok{$}\NormalTok{learning}\OperatorTok{==}\StringTok{"FL"} \OperatorTok{&}
\StringTok{           }\NormalTok{conjudge}\OperatorTok{$}\NormalTok{frequency}\OperatorTok{==}\StringTok{"high"} \OperatorTok{&}\StringTok{ }
\StringTok{           }\NormalTok{conjudge}\OperatorTok{$}\NormalTok{label}\OperatorTok{==}\StringTok{'tob'} \OperatorTok{&}\StringTok{ }
\StringTok{           }\NormalTok{conjudge}\OperatorTok{$}\NormalTok{category}\OperatorTok{==}\DecValTok{2} \OperatorTok{&}
\StringTok{           }\OperatorTok{!}\NormalTok{(conjudge}\OperatorTok{$}\NormalTok{subjID }\OperatorTok\StringTok{ }\NormalTok{badsubjs),]}\OperatorTok{$}\NormalTok{resp),}
  \CommentTok{# ROW 7                                              #MISMATCH -TYPE2}
  \KeywordTok{mean}\NormalTok{(conjudge[conjudge}\OperatorTok{$}\NormalTok{learning}\OperatorTok{==}\StringTok{"FL"} \OperatorTok{&}
\StringTok{           }\NormalTok{conjudge}\OperatorTok{$}\NormalTok{frequency}\OperatorTok{==}\StringTok{"high"} \OperatorTok{&}\StringTok{ }
\StringTok{           }\NormalTok{conjudge}\OperatorTok{$}\NormalTok{label}\OperatorTok{==}\StringTok{'dep'} \OperatorTok{&}\StringTok{ }
\StringTok{           }\NormalTok{conjudge}\OperatorTok{$}\NormalTok{category}\OperatorTok{==}\DecValTok{2} \OperatorTok{&}
\StringTok{           }\OperatorTok{!}\NormalTok{(conjudge}\OperatorTok{$}\NormalTok{subjID }\OperatorTok\StringTok{ }\NormalTok{badsubjs),]}\OperatorTok{$}\NormalTok{resp),}
  \CommentTok{# ROW 8}
  \KeywordTok{mean}\NormalTok{(conjudge[conjudge}\OperatorTok{$}\NormalTok{learning}\OperatorTok{==}\StringTok{"FL"} \OperatorTok{&}
\StringTok{           }\NormalTok{conjudge}\OperatorTok{$}\NormalTok{frequency}\OperatorTok{==}\StringTok{"high"} \OperatorTok{&}\StringTok{ }
\StringTok{           }\NormalTok{conjudge}\OperatorTok{$}\NormalTok{label}\OperatorTok{==}\StringTok{'bim'} \OperatorTok{&}\StringTok{ }
\StringTok{           }\NormalTok{conjudge}\OperatorTok{$}\NormalTok{category}\OperatorTok{==}\DecValTok{3} \OperatorTok{&}
\StringTok{           }\OperatorTok{!}\NormalTok{(conjudge}\OperatorTok{$}\NormalTok{subjID }\OperatorTok\StringTok{ }\NormalTok{badsubjs),]}\OperatorTok{$}\NormalTok{resp),}
  \CommentTok{# ROW 9}
  \KeywordTok{mean}\NormalTok{(conjudge[conjudge}\OperatorTok{$}\NormalTok{learning}\OperatorTok{==}\StringTok{"FL"} \OperatorTok{&}
\StringTok{           }\NormalTok{conjudge}\OperatorTok{$}\NormalTok{frequency}\OperatorTok{==}\StringTok{"high"} \OperatorTok{&}\StringTok{ }
\StringTok{           }\NormalTok{conjudge}\OperatorTok{$}\NormalTok{label}\OperatorTok{==}\StringTok{'tob'} \OperatorTok{&}\StringTok{ }
\StringTok{           }\NormalTok{conjudge}\OperatorTok{$}\NormalTok{category}\OperatorTok{==}\DecValTok{1} \OperatorTok{&}
\StringTok{           }\OperatorTok{!}\NormalTok{(conjudge}\OperatorTok{$}\NormalTok{subjID }\OperatorTok\StringTok{ }\NormalTok{badsubjs),]}\OperatorTok{$}\NormalTok{resp),}
\CommentTok{#----------------------------------------------- LF LEARNING }
\CommentTok{# ROW 1                                              #MATCH}
  \KeywordTok{mean}\NormalTok{(conjudge[conjudge}\OperatorTok{$}\NormalTok{learning}\OperatorTok{==}\StringTok{"LF"} \OperatorTok{&}\StringTok{ }
\StringTok{           }\NormalTok{conjudge}\OperatorTok{$}\NormalTok{frequency}\OperatorTok{==}\StringTok{"high"} \OperatorTok{&}\StringTok{ }
\StringTok{           }\NormalTok{conjudge}\OperatorTok{$}\NormalTok{label}\OperatorTok{==}\StringTok{'dep'} \OperatorTok{&}\StringTok{ }
\StringTok{           }\NormalTok{conjudge}\OperatorTok{$}\NormalTok{category}\OperatorTok{==}\DecValTok{1} \OperatorTok{&}
\StringTok{           }\OperatorTok{!}\NormalTok{(conjudge}\OperatorTok{$}\NormalTok{subjID }\OperatorTok\StringTok{ }\NormalTok{badsubjs),]}\OperatorTok{$}\NormalTok{resp),}
  \CommentTok{# ROW 2}
  \KeywordTok{mean}\NormalTok{(conjudge[conjudge}\OperatorTok{$}\NormalTok{learning}\OperatorTok{==}\StringTok{"LF"} \OperatorTok{&}
\StringTok{           }\NormalTok{conjudge}\OperatorTok{$}\NormalTok{frequency}\OperatorTok{==}\StringTok{"high"} \OperatorTok{&}\StringTok{ }
\StringTok{           }\NormalTok{conjudge}\OperatorTok{$}\NormalTok{label}\OperatorTok{==}\StringTok{'bim'} \OperatorTok{&}\StringTok{ }
\StringTok{           }\NormalTok{conjudge}\OperatorTok{$}\NormalTok{category}\OperatorTok{==}\DecValTok{2} \OperatorTok{&}
\StringTok{           }\OperatorTok{!}\NormalTok{(conjudge}\OperatorTok{$}\NormalTok{subjID }\OperatorTok\StringTok{ }\NormalTok{badsubjs),]}\OperatorTok{$}\NormalTok{resp),}
  \CommentTok{# ROW 3}
  \KeywordTok{mean}\NormalTok{(conjudge[conjudge}\OperatorTok{$}\NormalTok{learning}\OperatorTok{==}\StringTok{"LF"} \OperatorTok{&}
\StringTok{           }\NormalTok{conjudge}\OperatorTok{$}\NormalTok{frequency}\OperatorTok{==}\StringTok{"high"} \OperatorTok{&}\StringTok{ }
\StringTok{           }\NormalTok{conjudge}\OperatorTok{$}\NormalTok{label}\OperatorTok{==}\StringTok{'tob'} \OperatorTok{&}\StringTok{ }
\StringTok{           }\NormalTok{conjudge}\OperatorTok{$}\NormalTok{category}\OperatorTok{==}\DecValTok{3} \OperatorTok{&}
\StringTok{           }\OperatorTok{!}\NormalTok{(conjudge}\OperatorTok{$}\NormalTok{subjID }\OperatorTok\StringTok{ }\NormalTok{badsubjs),]}\OperatorTok{$}\NormalTok{resp),}
  \CommentTok{# ROW 4                                              #MISMATCH -TYPE1}
  \KeywordTok{mean}\NormalTok{(conjudge[conjudge}\OperatorTok{$}\NormalTok{learning}\OperatorTok{==}\StringTok{"LF"} \OperatorTok{&}
\StringTok{           }\NormalTok{conjudge}\OperatorTok{$}\NormalTok{frequency}\OperatorTok{==}\StringTok{"high"} \OperatorTok{&}\StringTok{ }
\StringTok{           }\NormalTok{conjudge}\OperatorTok{$}\NormalTok{label}\OperatorTok{==}\StringTok{'dep'} \OperatorTok{&}\StringTok{ }
\StringTok{           }\NormalTok{conjudge}\OperatorTok{$}\NormalTok{category}\OperatorTok{==}\DecValTok{3} \OperatorTok{&}
\StringTok{           }\OperatorTok{!}\NormalTok{(conjudge}\OperatorTok{$}\NormalTok{subjID }\OperatorTok\StringTok{ }\NormalTok{badsubjs),]}\OperatorTok{$}\NormalTok{resp),}
  \CommentTok{# ROW 5}
  \KeywordTok{mean}\NormalTok{(conjudge[conjudge}\OperatorTok{$}\NormalTok{learning}\OperatorTok{==}\StringTok{"LF"} \OperatorTok{&}
\StringTok{           }\NormalTok{conjudge}\OperatorTok{$}\NormalTok{frequency}\OperatorTok{==}\StringTok{"high"} \OperatorTok{&}\StringTok{ }
\StringTok{           }\NormalTok{conjudge}\OperatorTok{$}\NormalTok{label}\OperatorTok{==}\StringTok{'bim'} \OperatorTok{&}\StringTok{ }
\StringTok{           }\NormalTok{conjudge}\OperatorTok{$}\NormalTok{category}\OperatorTok{==}\DecValTok{1} \OperatorTok{&}
\StringTok{           }\OperatorTok{!}\NormalTok{(conjudge}\OperatorTok{$}\NormalTok{subjID }\OperatorTok\StringTok{ }\NormalTok{badsubjs),]}\OperatorTok{$}\NormalTok{resp),}
  \CommentTok{# ROW 6}
  \KeywordTok{mean}\NormalTok{(conjudge[conjudge}\OperatorTok{$}\NormalTok{learning}\OperatorTok{==}\StringTok{"LF"} \OperatorTok{&}
\StringTok{           }\NormalTok{conjudge}\OperatorTok{$}\NormalTok{frequency}\OperatorTok{==}\StringTok{"high"} \OperatorTok{&}\StringTok{ }
\StringTok{           }\NormalTok{conjudge}\OperatorTok{$}\NormalTok{label}\OperatorTok{==}\StringTok{'tob'} \OperatorTok{&}\StringTok{ }
\StringTok{           }\NormalTok{conjudge}\OperatorTok{$}\NormalTok{category}\OperatorTok{==}\DecValTok{2} \OperatorTok{&}
\StringTok{           }\OperatorTok{!}\NormalTok{(conjudge}\OperatorTok{$}\NormalTok{subjID }\OperatorTok\StringTok{ }\NormalTok{badsubjs),]}\OperatorTok{$}\NormalTok{resp),}
  \CommentTok{# ROW 7                                              #MISMATCH -TYPE2}
  \KeywordTok{mean}\NormalTok{(conjudge[conjudge}\OperatorTok{$}\NormalTok{learning}\OperatorTok{==}\StringTok{"LF"} \OperatorTok{&}
\StringTok{           }\NormalTok{conjudge}\OperatorTok{$}\NormalTok{frequency}\OperatorTok{==}\StringTok{"high"} \OperatorTok{&}\StringTok{ }
\StringTok{           }\NormalTok{conjudge}\OperatorTok{$}\NormalTok{label}\OperatorTok{==}\StringTok{'dep'} \OperatorTok{&}\StringTok{ }
\StringTok{           }\NormalTok{conjudge}\OperatorTok{$}\NormalTok{category}\OperatorTok{==}\DecValTok{2} \OperatorTok{&}
\StringTok{           }\OperatorTok{!}\NormalTok{(conjudge}\OperatorTok{$}\NormalTok{subjID }\OperatorTok\StringTok{ }\NormalTok{badsubjs),]}\OperatorTok{$}\NormalTok{resp),}
  \CommentTok{# ROW 8}
  \KeywordTok{mean}\NormalTok{(conjudge[conjudge}\OperatorTok{$}\NormalTok{learning}\OperatorTok{==}\StringTok{"LF"} \OperatorTok{&}
\StringTok{           }\NormalTok{conjudge}\OperatorTok{$}\NormalTok{frequency}\OperatorTok{==}\StringTok{"high"} \OperatorTok{&}\StringTok{ }
\StringTok{           }\NormalTok{conjudge}\OperatorTok{$}\NormalTok{label}\OperatorTok{==}\StringTok{'bim'} \OperatorTok{&}\StringTok{ }
\StringTok{           }\NormalTok{conjudge}\OperatorTok{$}\NormalTok{category}\OperatorTok{==}\DecValTok{3} \OperatorTok{&}
\StringTok{           }\OperatorTok{!}\NormalTok{(conjudge}\OperatorTok{$}\NormalTok{subjID }\OperatorTok\StringTok{ }\NormalTok{badsubjs),]}\OperatorTok{$}\NormalTok{resp),}
  \CommentTok{# ROW 9}
  \KeywordTok{mean}\NormalTok{(conjudge[conjudge}\OperatorTok{$}\NormalTok{learning}\OperatorTok{==}\StringTok{"LF"} \OperatorTok{&}
\StringTok{           }\NormalTok{conjudge}\OperatorTok{$}\NormalTok{frequency}\OperatorTok{==}\StringTok{"high"} \OperatorTok{&}\StringTok{ }
\StringTok{           }\NormalTok{conjudge}\OperatorTok{$}\NormalTok{label}\OperatorTok{==}\StringTok{'tob'} \OperatorTok{&}\StringTok{ }
\StringTok{           }\NormalTok{conjudge}\OperatorTok{$}\NormalTok{category}\OperatorTok{==}\DecValTok{1} \OperatorTok{&}
\StringTok{           }\OperatorTok{!}\NormalTok{(conjudge}\OperatorTok{$}\NormalTok{subjID }\OperatorTok\StringTok{ }\NormalTok{badsubjs),]}\OperatorTok{$}\NormalTok{resp)}
\NormalTok{)}
\NormalTok{highFreq}\OperatorTok{$}\NormalTok{resp <-}\StringTok{ }\NormalTok{resp}
\end{Highlighting}
\end{Shaded}

\begin{Shaded}
\begin{Highlighting}[]
\NormalTok{highFreq}
\end{Highlighting}
\end{Shaded}

\begin{verbatim}
##    learning frequency           type    label fribble fribbleCategory      resp
## 1        FL      high          match dep_cat1     1.1            cat1  24.80851
## 2        FL      high          match bim_cat2     2.1            cat2  66.31429
## 3        FL      high          match tob_cat3     3.1            cat3  63.18571
## 4        FL      high mismatch-type1 dep_cat1     3.1            cat3 -49.63380
## 5        FL      high mismatch-type1 bim_cat2     1.1            cat1 -29.79167
## 6        FL      high mismatch-type1 tob_cat3     2.1            cat2 -53.14815
## 7        FL      high mismatch-type2 dep_cat1     2.1            cat2 -60.92308
## 8        FL      high mismatch-type2 bim_cat2     3.1            cat3 -68.78723
## 9        FL      high mismatch-type2 tob_cat3     1.1            cat1 -56.72727
## 10       LF      high          match dep_cat1     1.1            cat1  69.30357
## 11       LF      high          match bim_cat2     2.1            cat2  66.94737
## 12       LF      high          match tob_cat3     3.1            cat3  57.78333
## 13       LF      high mismatch-type1 dep_cat1     3.1            cat3 -14.84211
## 14       LF      high mismatch-type1 bim_cat2     1.1            cat1 -39.18519
## 15       LF      high mismatch-type1 tob_cat3     2.1            cat2 -27.86207
## 16       LF      high mismatch-type2 dep_cat1     2.1            cat2 -47.37838
## 17       LF      high mismatch-type2 bim_cat2     3.1            cat3 -56.46154
## 18       LF      high mismatch-type2 tob_cat3     1.1            cat1 -37.63415
\end{verbatim}

Low frequency:

\begin{Shaded}
\begin{Highlighting}[]
\NormalTok{lowFreqFL<-}\KeywordTok{data.frame}\NormalTok{(}
           \DataTypeTok{learning =} \KeywordTok{rep}\NormalTok{(}\StringTok{"FL"}\NormalTok{,}\DecValTok{9}\NormalTok{),}
           \DataTypeTok{frequency =} \KeywordTok{rep}\NormalTok{(}\StringTok{"low"}\NormalTok{,}\DecValTok{9}\NormalTok{),}
           \DataTypeTok{type =} \KeywordTok{c}\NormalTok{(}\KeywordTok{rep}\NormalTok{(}\StringTok{"match"}\NormalTok{,}\DecValTok{3}\NormalTok{), }
                    \KeywordTok{rep}\NormalTok{(}\StringTok{"mismatch-type1"}\NormalTok{,}\DecValTok{3}\NormalTok{), }
                    \KeywordTok{rep}\NormalTok{(}\StringTok{"mismatch-type2"}\NormalTok{,}\DecValTok{3}\NormalTok{)),}
           \DataTypeTok{label =} \KeywordTok{c}\NormalTok{(}\StringTok{"dep_cat1"}\NormalTok{, }\StringTok{"bim_cat2"}\NormalTok{, }\StringTok{"tob_cat3"}\NormalTok{),}
           \DataTypeTok{fribble =} \KeywordTok{c}\NormalTok{(}\FloatTok{1.2}\NormalTok{,}\FloatTok{2.2}\NormalTok{,}\FloatTok{3.2}\NormalTok{,}
                       \FloatTok{2.2}\NormalTok{,}\FloatTok{3.2}\NormalTok{,}\FloatTok{1.2}\NormalTok{,}
                       \FloatTok{3.2}\NormalTok{,}\FloatTok{1.2}\NormalTok{,}\FloatTok{2.2}\NormalTok{),}
           \DataTypeTok{fribbleCategory =} \KeywordTok{c}\NormalTok{(}\StringTok{"cat1"}\NormalTok{, }\StringTok{"cat2"}\NormalTok{, }\StringTok{"cat3"}\NormalTok{, }\CommentTok{#match}
                               \StringTok{"cat2"}\NormalTok{, }\StringTok{"cat3"}\NormalTok{, }\StringTok{"cat1"}\NormalTok{, }\CommentTok{#mis-type1}
                               \StringTok{"cat3"}\NormalTok{, }\StringTok{"cat1"}\NormalTok{, }\StringTok{"cat2"}\NormalTok{)) }\CommentTok{#mis-type2}

\NormalTok{lowFreqLF<-}\KeywordTok{data.frame}\NormalTok{(}
           \DataTypeTok{learning =} \KeywordTok{rep}\NormalTok{(}\StringTok{"LF"}\NormalTok{,}\DecValTok{9}\NormalTok{),}
           \DataTypeTok{frequency =} \KeywordTok{rep}\NormalTok{(}\StringTok{"low"}\NormalTok{,}\DecValTok{9}\NormalTok{),}
           \DataTypeTok{type =} \KeywordTok{c}\NormalTok{(}\KeywordTok{rep}\NormalTok{(}\StringTok{"match"}\NormalTok{,}\DecValTok{3}\NormalTok{), }
                    \KeywordTok{rep}\NormalTok{(}\StringTok{"mismatch-type1"}\NormalTok{,}\DecValTok{3}\NormalTok{), }
                    \KeywordTok{rep}\NormalTok{(}\StringTok{"mismatch-type2"}\NormalTok{,}\DecValTok{3}\NormalTok{)),}
           \DataTypeTok{label =} \KeywordTok{c}\NormalTok{(}\StringTok{"dep_cat1"}\NormalTok{, }\StringTok{"bim_cat2"}\NormalTok{, }\StringTok{"tob_cat3"}\NormalTok{),}
           \DataTypeTok{fribble =} \KeywordTok{c}\NormalTok{(}\FloatTok{1.2}\NormalTok{,}\FloatTok{2.2}\NormalTok{,}\FloatTok{3.2}\NormalTok{,}
                       \FloatTok{2.2}\NormalTok{,}\FloatTok{3.2}\NormalTok{,}\FloatTok{1.2}\NormalTok{,}
                       \FloatTok{3.2}\NormalTok{,}\FloatTok{1.2}\NormalTok{,}\FloatTok{2.2}\NormalTok{),}
           \DataTypeTok{fribbleCategory =} \KeywordTok{c}\NormalTok{(}\StringTok{"cat1"}\NormalTok{, }\StringTok{"cat2"}\NormalTok{, }\StringTok{"cat3"}\NormalTok{, }\CommentTok{#match}
                               \StringTok{"cat2"}\NormalTok{, }\StringTok{"cat3"}\NormalTok{, }\StringTok{"cat1"}\NormalTok{, }\CommentTok{#mis-type1}
                               \StringTok{"cat3"}\NormalTok{, }\StringTok{"cat1"}\NormalTok{, }\StringTok{"cat2"}\NormalTok{)) }\CommentTok{#mis-type2}
\NormalTok{lowFreq<-}\StringTok{ }\KeywordTok{rbind}\NormalTok{(lowFreqFL, lowFreqLF)}
\KeywordTok{rm}\NormalTok{(lowFreqFL, lowFreqLF)}
\end{Highlighting}
\end{Shaded}

\begin{Shaded}
\begin{Highlighting}[]
\NormalTok{resp <-}\StringTok{ }\KeywordTok{c}\NormalTok{(}
\CommentTok{#-----------------------------------------FL LEARNING}
  \CommentTok{# ROW 1                                              #MATCH}
  \KeywordTok{mean}\NormalTok{(conjudge[conjudge}\OperatorTok{$}\NormalTok{learning}\OperatorTok{==}\StringTok{"FL"} \OperatorTok{&}
\StringTok{           }\NormalTok{conjudge}\OperatorTok{$}\NormalTok{frequency}\OperatorTok{==}\StringTok{"low"} \OperatorTok{&}\StringTok{ }
\StringTok{           }\NormalTok{conjudge}\OperatorTok{$}\NormalTok{label}\OperatorTok{==}\StringTok{'dep'} \OperatorTok{&}\StringTok{ }
\StringTok{           }\NormalTok{conjudge}\OperatorTok{$}\NormalTok{category}\OperatorTok{==}\DecValTok{1} \OperatorTok{&}
\StringTok{           }\OperatorTok{!}\NormalTok{(conjudge}\OperatorTok{$}\NormalTok{subjID }\OperatorTok\StringTok{ }\NormalTok{badsubjs),]}\OperatorTok{$}\NormalTok{resp),}
  \CommentTok{# ROW 2}
  \KeywordTok{mean}\NormalTok{(conjudge[conjudge}\OperatorTok{$}\NormalTok{learning}\OperatorTok{==}\StringTok{"FL"} \OperatorTok{&}
\StringTok{           }\NormalTok{conjudge}\OperatorTok{$}\NormalTok{frequency}\OperatorTok{==}\StringTok{"low"} \OperatorTok{&}\StringTok{ }
\StringTok{           }\NormalTok{conjudge}\OperatorTok{$}\NormalTok{label}\OperatorTok{==}\StringTok{'bim'} \OperatorTok{&}\StringTok{ }
\StringTok{           }\NormalTok{conjudge}\OperatorTok{$}\NormalTok{category}\OperatorTok{==}\DecValTok{2} \OperatorTok{&}
\StringTok{           }\OperatorTok{!}\NormalTok{(conjudge}\OperatorTok{$}\NormalTok{subjID }\OperatorTok\StringTok{ }\NormalTok{badsubjs),]}\OperatorTok{$}\NormalTok{resp),}
  \CommentTok{# ROW 3}
  \KeywordTok{mean}\NormalTok{(conjudge[conjudge}\OperatorTok{$}\NormalTok{learning}\OperatorTok{==}\StringTok{"FL"} \OperatorTok{&}
\StringTok{           }\NormalTok{conjudge}\OperatorTok{$}\NormalTok{frequency}\OperatorTok{==}\StringTok{"low"} \OperatorTok{&}\StringTok{ }
\StringTok{           }\NormalTok{conjudge}\OperatorTok{$}\NormalTok{label}\OperatorTok{==}\StringTok{'tob'} \OperatorTok{&}\StringTok{ }
\StringTok{           }\NormalTok{conjudge}\OperatorTok{$}\NormalTok{category}\OperatorTok{==}\DecValTok{3} \OperatorTok{&}
\StringTok{           }\OperatorTok{!}\NormalTok{(conjudge}\OperatorTok{$}\NormalTok{subjID }\OperatorTok\StringTok{ }\NormalTok{badsubjs),]}\OperatorTok{$}\NormalTok{resp),}
  \CommentTok{# ROW 4                                              #MISMATCH -TYPE1}
  \KeywordTok{mean}\NormalTok{(conjudge[conjudge}\OperatorTok{$}\NormalTok{learning}\OperatorTok{==}\StringTok{"FL"} \OperatorTok{&}
\StringTok{           }\NormalTok{conjudge}\OperatorTok{$}\NormalTok{frequency}\OperatorTok{==}\StringTok{"low"} \OperatorTok{&}\StringTok{ }
\StringTok{           }\NormalTok{conjudge}\OperatorTok{$}\NormalTok{label}\OperatorTok{==}\StringTok{'dep'} \OperatorTok{&}\StringTok{ }
\StringTok{           }\NormalTok{conjudge}\OperatorTok{$}\NormalTok{category}\OperatorTok{==}\DecValTok{2} \OperatorTok{&}
\StringTok{           }\OperatorTok{!}\NormalTok{(conjudge}\OperatorTok{$}\NormalTok{subjID }\OperatorTok\StringTok{ }\NormalTok{badsubjs),]}\OperatorTok{$}\NormalTok{resp),}
  \CommentTok{# ROW 5}
  \KeywordTok{mean}\NormalTok{(conjudge[conjudge}\OperatorTok{$}\NormalTok{learning}\OperatorTok{==}\StringTok{"FL"} \OperatorTok{&}
\StringTok{           }\NormalTok{conjudge}\OperatorTok{$}\NormalTok{frequency}\OperatorTok{==}\StringTok{"low"} \OperatorTok{&}\StringTok{ }
\StringTok{           }\NormalTok{conjudge}\OperatorTok{$}\NormalTok{label}\OperatorTok{==}\StringTok{'bim'} \OperatorTok{&}\StringTok{ }
\StringTok{           }\NormalTok{conjudge}\OperatorTok{$}\NormalTok{category}\OperatorTok{==}\DecValTok{3} \OperatorTok{&}
\StringTok{           }\OperatorTok{!}\NormalTok{(conjudge}\OperatorTok{$}\NormalTok{subjID }\OperatorTok\StringTok{ }\NormalTok{badsubjs),]}\OperatorTok{$}\NormalTok{resp),}
  \CommentTok{# ROW 6}
  \KeywordTok{mean}\NormalTok{(conjudge[conjudge}\OperatorTok{$}\NormalTok{learning}\OperatorTok{==}\StringTok{"FL"} \OperatorTok{&}
\StringTok{           }\NormalTok{conjudge}\OperatorTok{$}\NormalTok{frequency}\OperatorTok{==}\StringTok{"low"} \OperatorTok{&}\StringTok{ }
\StringTok{           }\NormalTok{conjudge}\OperatorTok{$}\NormalTok{label}\OperatorTok{==}\StringTok{'tob'} \OperatorTok{&}\StringTok{ }
\StringTok{           }\NormalTok{conjudge}\OperatorTok{$}\NormalTok{category}\OperatorTok{==}\DecValTok{1} \OperatorTok{&}
\StringTok{           }\OperatorTok{!}\NormalTok{(conjudge}\OperatorTok{$}\NormalTok{subjID }\OperatorTok\StringTok{ }\NormalTok{badsubjs),]}\OperatorTok{$}\NormalTok{resp),}
  \CommentTok{# ROW 7                                              #MISMATCH -TYPE2}
  \KeywordTok{mean}\NormalTok{(conjudge[conjudge}\OperatorTok{$}\NormalTok{learning}\OperatorTok{==}\StringTok{"FL"} \OperatorTok{&}
\StringTok{           }\NormalTok{conjudge}\OperatorTok{$}\NormalTok{frequency}\OperatorTok{==}\StringTok{"low"} \OperatorTok{&}\StringTok{ }
\StringTok{           }\NormalTok{conjudge}\OperatorTok{$}\NormalTok{label}\OperatorTok{==}\StringTok{'dep'} \OperatorTok{&}\StringTok{ }
\StringTok{           }\NormalTok{conjudge}\OperatorTok{$}\NormalTok{category}\OperatorTok{==}\DecValTok{3} \OperatorTok{&}
\StringTok{           }\OperatorTok{!}\NormalTok{(conjudge}\OperatorTok{$}\NormalTok{subjID }\OperatorTok\StringTok{ }\NormalTok{badsubjs),]}\OperatorTok{$}\NormalTok{resp),}
  \CommentTok{# ROW 8}
  \KeywordTok{mean}\NormalTok{(conjudge[conjudge}\OperatorTok{$}\NormalTok{learning}\OperatorTok{==}\StringTok{"FL"} \OperatorTok{&}
\StringTok{           }\NormalTok{conjudge}\OperatorTok{$}\NormalTok{frequency}\OperatorTok{==}\StringTok{"low"} \OperatorTok{&}\StringTok{ }
\StringTok{           }\NormalTok{conjudge}\OperatorTok{$}\NormalTok{label}\OperatorTok{==}\StringTok{'bim'} \OperatorTok{&}\StringTok{ }
\StringTok{           }\NormalTok{conjudge}\OperatorTok{$}\NormalTok{category}\OperatorTok{==}\DecValTok{1} \OperatorTok{&}
\StringTok{           }\OperatorTok{!}\NormalTok{(conjudge}\OperatorTok{$}\NormalTok{subjID }\OperatorTok\StringTok{ }\NormalTok{badsubjs),]}\OperatorTok{$}\NormalTok{resp),}
  \CommentTok{# ROW 9}
  \KeywordTok{mean}\NormalTok{(conjudge[conjudge}\OperatorTok{$}\NormalTok{learning}\OperatorTok{==}\StringTok{"FL"} \OperatorTok{&}
\StringTok{           }\NormalTok{conjudge}\OperatorTok{$}\NormalTok{frequency}\OperatorTok{==}\StringTok{"low"} \OperatorTok{&}\StringTok{ }
\StringTok{           }\NormalTok{conjudge}\OperatorTok{$}\NormalTok{label}\OperatorTok{==}\StringTok{'tob'} \OperatorTok{&}\StringTok{ }
\StringTok{           }\NormalTok{conjudge}\OperatorTok{$}\NormalTok{category}\OperatorTok{==}\DecValTok{2} \OperatorTok{&}
\StringTok{           }\OperatorTok{!}\NormalTok{(conjudge}\OperatorTok{$}\NormalTok{subjID }\OperatorTok\StringTok{ }\NormalTok{badsubjs),]}\OperatorTok{$}\NormalTok{resp),}

\CommentTok{#-----------------------------------------LF LEARNING}
  \CommentTok{# ROW 1                                              #MATCH}
  \KeywordTok{mean}\NormalTok{(conjudge[conjudge}\OperatorTok{$}\NormalTok{learning}\OperatorTok{==}\StringTok{"LF"} \OperatorTok{&}
\StringTok{           }\NormalTok{conjudge}\OperatorTok{$}\NormalTok{frequency}\OperatorTok{==}\StringTok{"low"} \OperatorTok{&}\StringTok{ }
\StringTok{           }\NormalTok{conjudge}\OperatorTok{$}\NormalTok{label}\OperatorTok{==}\StringTok{'dep'} \OperatorTok{&}\StringTok{ }
\StringTok{           }\NormalTok{conjudge}\OperatorTok{$}\NormalTok{category}\OperatorTok{==}\DecValTok{1} \OperatorTok{&}
\StringTok{           }\OperatorTok{!}\NormalTok{(conjudge}\OperatorTok{$}\NormalTok{subjID }\OperatorTok\StringTok{ }\NormalTok{badsubjs),]}\OperatorTok{$}\NormalTok{resp),}
  \CommentTok{# ROW 2}
  \KeywordTok{mean}\NormalTok{(conjudge[conjudge}\OperatorTok{$}\NormalTok{learning}\OperatorTok{==}\StringTok{"LF"} \OperatorTok{&}
\StringTok{           }\NormalTok{conjudge}\OperatorTok{$}\NormalTok{frequency}\OperatorTok{==}\StringTok{"low"} \OperatorTok{&}\StringTok{ }
\StringTok{           }\NormalTok{conjudge}\OperatorTok{$}\NormalTok{label}\OperatorTok{==}\StringTok{'bim'} \OperatorTok{&}\StringTok{ }
\StringTok{           }\NormalTok{conjudge}\OperatorTok{$}\NormalTok{category}\OperatorTok{==}\DecValTok{2} \OperatorTok{&}
\StringTok{           }\OperatorTok{!}\NormalTok{(conjudge}\OperatorTok{$}\NormalTok{subjID }\OperatorTok\StringTok{ }\NormalTok{badsubjs),]}\OperatorTok{$}\NormalTok{resp),}
  \CommentTok{# ROW 3}
  \KeywordTok{mean}\NormalTok{(conjudge[conjudge}\OperatorTok{$}\NormalTok{learning}\OperatorTok{==}\StringTok{"LF"} \OperatorTok{&}
\StringTok{           }\NormalTok{conjudge}\OperatorTok{$}\NormalTok{frequency}\OperatorTok{==}\StringTok{"low"} \OperatorTok{&}\StringTok{ }
\StringTok{           }\NormalTok{conjudge}\OperatorTok{$}\NormalTok{label}\OperatorTok{==}\StringTok{'tob'} \OperatorTok{&}\StringTok{ }
\StringTok{           }\NormalTok{conjudge}\OperatorTok{$}\NormalTok{category}\OperatorTok{==}\DecValTok{3} \OperatorTok{&}
\StringTok{           }\OperatorTok{!}\NormalTok{(conjudge}\OperatorTok{$}\NormalTok{subjID }\OperatorTok\StringTok{ }\NormalTok{badsubjs),]}\OperatorTok{$}\NormalTok{resp),}
  \CommentTok{# ROW 4                                              #MISMATCH -TYPE1}
  \KeywordTok{mean}\NormalTok{(conjudge[conjudge}\OperatorTok{$}\NormalTok{learning}\OperatorTok{==}\StringTok{"LF"} \OperatorTok{&}
\StringTok{           }\NormalTok{conjudge}\OperatorTok{$}\NormalTok{frequency}\OperatorTok{==}\StringTok{"low"} \OperatorTok{&}\StringTok{ }
\StringTok{           }\NormalTok{conjudge}\OperatorTok{$}\NormalTok{label}\OperatorTok{==}\StringTok{'dep'} \OperatorTok{&}\StringTok{ }
\StringTok{           }\NormalTok{conjudge}\OperatorTok{$}\NormalTok{category}\OperatorTok{==}\DecValTok{2} \OperatorTok{&}
\StringTok{           }\OperatorTok{!}\NormalTok{(conjudge}\OperatorTok{$}\NormalTok{subjID }\OperatorTok\StringTok{ }\NormalTok{badsubjs),]}\OperatorTok{$}\NormalTok{resp),}
  \CommentTok{# ROW 5}
  \KeywordTok{mean}\NormalTok{(conjudge[conjudge}\OperatorTok{$}\NormalTok{learning}\OperatorTok{==}\StringTok{"LF"} \OperatorTok{&}
\StringTok{           }\NormalTok{conjudge}\OperatorTok{$}\NormalTok{frequency}\OperatorTok{==}\StringTok{"low"} \OperatorTok{&}\StringTok{ }
\StringTok{           }\NormalTok{conjudge}\OperatorTok{$}\NormalTok{label}\OperatorTok{==}\StringTok{'bim'} \OperatorTok{&}\StringTok{ }
\StringTok{           }\NormalTok{conjudge}\OperatorTok{$}\NormalTok{category}\OperatorTok{==}\DecValTok{3} \OperatorTok{&}
\StringTok{           }\OperatorTok{!}\NormalTok{(conjudge}\OperatorTok{$}\NormalTok{subjID }\OperatorTok\StringTok{ }\NormalTok{badsubjs),]}\OperatorTok{$}\NormalTok{resp),}
  \CommentTok{# ROW 6}
  \KeywordTok{mean}\NormalTok{(conjudge[conjudge}\OperatorTok{$}\NormalTok{learning}\OperatorTok{==}\StringTok{"LF"} \OperatorTok{&}
\StringTok{           }\NormalTok{conjudge}\OperatorTok{$}\NormalTok{frequency}\OperatorTok{==}\StringTok{"low"} \OperatorTok{&}\StringTok{ }
\StringTok{           }\NormalTok{conjudge}\OperatorTok{$}\NormalTok{label}\OperatorTok{==}\StringTok{'tob'} \OperatorTok{&}\StringTok{ }
\StringTok{           }\NormalTok{conjudge}\OperatorTok{$}\NormalTok{category}\OperatorTok{==}\DecValTok{1} \OperatorTok{&}
\StringTok{           }\OperatorTok{!}\NormalTok{(conjudge}\OperatorTok{$}\NormalTok{subjID }\OperatorTok\StringTok{ }\NormalTok{badsubjs),]}\OperatorTok{$}\NormalTok{resp),}
  \CommentTok{# ROW 7                                              #MISMATCH -TYPE2}
  \KeywordTok{mean}\NormalTok{(conjudge[conjudge}\OperatorTok{$}\NormalTok{learning}\OperatorTok{==}\StringTok{"LF"} \OperatorTok{&}
\StringTok{           }\NormalTok{conjudge}\OperatorTok{$}\NormalTok{frequency}\OperatorTok{==}\StringTok{"low"} \OperatorTok{&}\StringTok{ }
\StringTok{           }\NormalTok{conjudge}\OperatorTok{$}\NormalTok{label}\OperatorTok{==}\StringTok{'dep'} \OperatorTok{&}\StringTok{ }
\StringTok{           }\NormalTok{conjudge}\OperatorTok{$}\NormalTok{category}\OperatorTok{==}\DecValTok{3} \OperatorTok{&}
\StringTok{           }\OperatorTok{!}\NormalTok{(conjudge}\OperatorTok{$}\NormalTok{subjID }\OperatorTok\StringTok{ }\NormalTok{badsubjs),]}\OperatorTok{$}\NormalTok{resp),}
  \CommentTok{# ROW 8}
  \KeywordTok{mean}\NormalTok{(conjudge[conjudge}\OperatorTok{$}\NormalTok{learning}\OperatorTok{==}\StringTok{"LF"} \OperatorTok{&}
\StringTok{           }\NormalTok{conjudge}\OperatorTok{$}\NormalTok{frequency}\OperatorTok{==}\StringTok{"low"} \OperatorTok{&}\StringTok{ }
\StringTok{           }\NormalTok{conjudge}\OperatorTok{$}\NormalTok{label}\OperatorTok{==}\StringTok{'bim'} \OperatorTok{&}\StringTok{ }
\StringTok{           }\NormalTok{conjudge}\OperatorTok{$}\NormalTok{category}\OperatorTok{==}\DecValTok{1} \OperatorTok{&}
\StringTok{           }\OperatorTok{!}\NormalTok{(conjudge}\OperatorTok{$}\NormalTok{subjID }\OperatorTok\StringTok{ }\NormalTok{badsubjs),]}\OperatorTok{$}\NormalTok{resp),}
  \CommentTok{# ROW 9}
  \KeywordTok{mean}\NormalTok{(conjudge[conjudge}\OperatorTok{$}\NormalTok{learning}\OperatorTok{==}\StringTok{"LF"} \OperatorTok{&}
\StringTok{           }\NormalTok{conjudge}\OperatorTok{$}\NormalTok{frequency}\OperatorTok{==}\StringTok{"low"} \OperatorTok{&}\StringTok{ }
\StringTok{           }\NormalTok{conjudge}\OperatorTok{$}\NormalTok{label}\OperatorTok{==}\StringTok{'tob'} \OperatorTok{&}\StringTok{ }
\StringTok{           }\NormalTok{conjudge}\OperatorTok{$}\NormalTok{category}\OperatorTok{==}\DecValTok{2} \OperatorTok{&}
\StringTok{           }\OperatorTok{!}\NormalTok{(conjudge}\OperatorTok{$}\NormalTok{subjID }\OperatorTok\StringTok{ }\NormalTok{badsubjs),]}\OperatorTok{$}\NormalTok{resp)}
\NormalTok{)}
\NormalTok{lowFreq}\OperatorTok{$}\NormalTok{resp <-}\StringTok{ }\NormalTok{resp}
\end{Highlighting}
\end{Shaded}

\begin{Shaded}
\begin{Highlighting}[]
\KeywordTok{rbind}\NormalTok{(highFreq, lowFreq)->}\StringTok{ }\NormalTok{humansWeights}
\NormalTok{humansWeights}\OperatorTok{$}\NormalTok{learning <-}\StringTok{ }\KeywordTok{as.factor}\NormalTok{(humansWeights}\OperatorTok{$}\NormalTok{learning); humansWeights}\OperatorTok{$}\NormalTok{frequency <-}\StringTok{ }\KeywordTok{as.factor}\NormalTok{(humansWeights}\OperatorTok{$}\NormalTok{frequency); humansWeights}\OperatorTok{$}\NormalTok{type <-}\StringTok{ }\KeywordTok{as.factor}\NormalTok{(humansWeights}\OperatorTok{$}\NormalTok{type); humansWeights}\OperatorTok{$}\NormalTok{label <-}\StringTok{ }\KeywordTok{as.factor}\NormalTok{(humansWeights}\OperatorTok{$}\NormalTok{label); humansWeights}\OperatorTok{$}\NormalTok{fribbleCategory <-}\StringTok{ }\KeywordTok{as.factor}\NormalTok{(humansWeights}\OperatorTok{$}\NormalTok{fribbleCategory)}
\KeywordTok{rm}\NormalTok{(highFreq, lowFreq)}
\KeywordTok{summary}\NormalTok{(humansWeights)}
\end{Highlighting}
\end{Shaded}

\begin{verbatim}
##  learning frequency             type         label       fribble    
##  FL:18    high:18   match         :12   bim_cat2:12   Min.   :1.10  
##  LF:18    low :18   mismatch-type1:12   dep_cat1:12   1st Qu.:1.20  
##                     mismatch-type2:12   tob_cat3:12   Median :2.15  
##                                                       Mean   :2.15  
##                                                       3rd Qu.:3.10  
##                                                       Max.   :3.20  
##  fribbleCategory      resp       
##  cat1:12         Min.   :-68.79  
##  cat2:12         1st Qu.:-39.98  
##  cat3:12         Median :-19.43  
##                  Mean   :-11.04  
##                  3rd Qu.: 10.34  
##                  Max.   : 69.30
\end{verbatim}

\begin{Shaded}
\begin{Highlighting}[]
\NormalTok{dataWeight <-}\StringTok{ }\KeywordTok{aggregate}\NormalTok{(resp }\OperatorTok{~}\StringTok{ }\NormalTok{learning }\OperatorTok{+}\StringTok{ }\NormalTok{frequency }\OperatorTok{+}\StringTok{ }\NormalTok{type, }\DataTypeTok{data =}\NormalTok{ humansWeights,}\DataTypeTok{FUN =}\NormalTok{ mean)}
\NormalTok{dataWeight}
\end{Highlighting}
\end{Shaded}

\begin{verbatim}
##    learning frequency           type       resp
## 1        FL      high          match  51.436170
## 2        LF      high          match  64.678091
## 3        FL       low          match  15.190638
## 4        LF       low          match   6.902998
## 5        FL      high mismatch-type1 -44.191206
## 6        LF      high mismatch-type1 -27.296453
## 7        FL       low mismatch-type1 -15.901552
## 8        LF       low mismatch-type1  -2.312539
## 9        FL      high mismatch-type2 -62.145861
## 10       LF      high mismatch-type2 -47.158021
## 11       FL       low mismatch-type2 -36.864642
## 12       LF       low mismatch-type2 -34.874315
\end{verbatim}

\begin{Shaded}
\begin{Highlighting}[]
\NormalTok{lollipopWeight<-}\KeywordTok{ggdotchart}\NormalTok{(dataWeight, }\DataTypeTok{x =} \StringTok{"type"}\NormalTok{, }\DataTypeTok{y =} \StringTok{"resp"}\NormalTok{,}
           \CommentTok{#color = "learning",                                # Color by groups}
           \DataTypeTok{palette =} \KeywordTok{c}\NormalTok{(}\StringTok{"#00AFBB"}\NormalTok{, }\StringTok{"#E7B800"}\NormalTok{, }\StringTok{"#FC4E07"}\NormalTok{), }\CommentTok{# Custom color palette}
           \DataTypeTok{add =} \StringTok{"segments"}\NormalTok{,                             }\CommentTok{# Add segments from y = 0 to dots}
           \DataTypeTok{rotate =}\NormalTok{ T,}
           \DataTypeTok{add.params =} \KeywordTok{list}\NormalTok{(}\DataTypeTok{color =} \StringTok{"lightgray"}\NormalTok{, }\DataTypeTok{size =} \DecValTok{2}\NormalTok{), }\CommentTok{# Change segment color and size}
           \CommentTok{#group = "learning",                                # Order by groups}
           \DataTypeTok{dot.size =} \DecValTok{10}\NormalTok{,                                 }\CommentTok{# Large dot size}
           \DataTypeTok{label =} \KeywordTok{round}\NormalTok{(dataWeight}\OperatorTok{$}\NormalTok{resp,}\DecValTok{1}\NormalTok{),                        }\CommentTok{# Add mpg values as dot labels}
           \DataTypeTok{font.label =} \KeywordTok{list}\NormalTok{(}\DataTypeTok{color =} \StringTok{"white"}\NormalTok{, }\DataTypeTok{size =} \DecValTok{9}\NormalTok{, }
                             \DataTypeTok{vjust =} \FloatTok{0.5}\NormalTok{),               }\CommentTok{# Adjust label parameters}
           \DataTypeTok{ggtheme =} \KeywordTok{theme_pubr}\NormalTok{()                        }\CommentTok{# ggplot2 theme}
\NormalTok{           )}\OperatorTok{+}\StringTok{ }\KeywordTok{facet_grid}\NormalTok{( learning }\OperatorTok{~}\StringTok{ }\NormalTok{frequency) }\OperatorTok{+}
\StringTok{  }\KeywordTok{geom_hline}\NormalTok{(}\DataTypeTok{yintercept =} \DecValTok{0}\NormalTok{, }\DataTypeTok{linetype =} \DecValTok{2}\NormalTok{, }\DataTypeTok{color =} \StringTok{"lightgray"}\NormalTok{) }

\NormalTok{lollipopWeight}
\end{Highlighting}
\end{Shaded}

\includegraphics{preProcessing_files/figure-latex/unnamed-chunk-100-1.pdf}

Raw means

\begin{Shaded}
\begin{Highlighting}[]
\NormalTok{respShape<-}\StringTok{ }\KeywordTok{summarySEwithin}\NormalTok{(}\DataTypeTok{data =}\NormalTok{ conjudge[}\OperatorTok{!}\NormalTok{(conjudge}\OperatorTok{$}\NormalTok{subjID }\OperatorTok\StringTok{ }\NormalTok{badsubjs) }\OperatorTok{&}\StringTok{ }\NormalTok{conjudge}\OperatorTok{$}\NormalTok{acc}\OperatorTok{==}\DecValTok{1}\NormalTok{,], }\DataTypeTok{measurevar =} \StringTok{"resp"}\NormalTok{, }\DataTypeTok{betweenvars =} \StringTok{"learning"}\NormalTok{, }\DataTypeTok{withinvars =} \KeywordTok{c}\NormalTok{(}\StringTok{"frequency"}\NormalTok{, }\StringTok{"bodyShape"}\NormalTok{), }\DataTypeTok{idvar =} \StringTok{"subjID"}\NormalTok{, }\DataTypeTok{conf.interval =} \FloatTok{.95}\NormalTok{)}
\NormalTok{respShape}
\end{Highlighting}
\end{Shaded}

\begin{verbatim}
##    learning frequency bodyShape  N      resp resp_norm       sd       se
## 1        FL      high    barrel 70 63.185714 67.863381 51.73751 6.183815
## 2        FL      high       jar 47 24.808511 32.653574 58.90257 8.591823
## 3        FL      high     trunk 70 66.314286 56.459166 55.08613 6.584052
## 4        FL       low    barrel 45  2.333333  9.495814 51.64421 7.698665
## 5        FL       low       jar 67 28.253731 19.320010 57.25256 6.994513
## 6        FL       low     trunk 66 14.984848 19.362189 57.67604 7.099430
## 7        LF      high    barrel 60 57.783333 60.207209 58.92314 7.606945
## 8        LF      high       jar 56 69.303571 66.621438 55.93900 7.475163
## 9        LF      high     trunk 57 66.947368 62.365961 56.85924 7.531191
## 10       LF       low    barrel 56  5.285714  5.962281 63.37117 8.468328
## 11       LF       low       jar 56 10.071429  8.621142 50.85262 6.795468
## 12       LF       low     trunk 54  5.351852 10.727451 63.93947 8.701059
##          ci
## 1  12.33637
## 2  17.29444
## 3  13.13482
## 4  15.51564
## 5  13.96500
## 6  14.17854
## 7  15.22146
## 8  14.98056
## 9  15.08679
## 10 16.97091
## 11 13.61842
## 12 17.45211
\end{verbatim}

Let's clean the global environment:

\begin{Shaded}
\begin{Highlighting}[]
\KeywordTok{rm}\NormalTok{(n_bins, rt_range, problematicPeople, frequency, dumbPeople, break_seq, task, temp, timeslice_range, p, ms, n, nrows, subjs, totsubjs ,genTask, fribbleSet, count_range, alltasks, subjmissed, df)}
\end{Highlighting}
\end{Shaded}

\begin{verbatim}
## Warning in rm(n_bins, rt_range, problematicPeople, frequency, dumbPeople, :
## object 'n_bins' not found
\end{verbatim}

\begin{verbatim}
## Warning in rm(n_bins, rt_range, problematicPeople, frequency, dumbPeople, :
## object 'rt_range' not found
\end{verbatim}

\begin{verbatim}
## Warning in rm(n_bins, rt_range, problematicPeople, frequency, dumbPeople, :
## object 'break_seq' not found
\end{verbatim}

\begin{verbatim}
## Warning in rm(n_bins, rt_range, problematicPeople, frequency, dumbPeople, :
## object 'timeslice_range' not found
\end{verbatim}

\begin{verbatim}
## Warning in rm(n_bins, rt_range, problematicPeople, frequency, dumbPeople, :
## object 'count_range' not found
\end{verbatim}

\hypertarget{bayes-factor-calculation-with-glmms}{%
\section{Bayes factor calculation with
GLMMs}\label{bayes-factor-calculation-with-glmms}}

\hypertarget{estimates-of-the-betas-from-the-flo-paper}{%
\subsection{Estimates of the betas from the FLO
paper}\label{estimates-of-the-betas-from-the-flo-paper}}

\begin{Shaded}
\begin{Highlighting}[]
\CommentTok{#means}
\NormalTok{highfreq_mean<-}\StringTok{ }\KeywordTok{mean}\NormalTok{(}\DecValTok{88}\NormalTok{, }\DecValTok{98}\NormalTok{)}
\NormalTok{lowfreq_mean <-}\StringTok{ }\KeywordTok{mean}\NormalTok{(}\DecValTok{38}\NormalTok{, }\DecValTok{78}\NormalTok{)}

\NormalTok{n <-}\StringTok{ }\KeywordTok{c}\NormalTok{(}\DecValTok{32}\NormalTok{) }

\CommentTok{#sd}
\NormalTok{highfreq_sd <-}\StringTok{ }\KeywordTok{c}\NormalTok{(}\DecValTok{5}\OperatorTok{*}\KeywordTok{sqrt}\NormalTok{(n))  }\CommentTok{#Paper has standard errors represented (I guess),}
                     \CommentTok{#I'm going to transform it back to standard deviations}
\NormalTok{lowfreq_sd <-}\StringTok{ }\KeywordTok{c}\NormalTok{(}\DecValTok{5}\OperatorTok{*}\KeywordTok{sqrt}\NormalTok{(n)) }\CommentTok{#also, they look the same to me from the picture}
                           \CommentTok{#but low frequency should lead more variability.}
\end{Highlighting}
\end{Shaded}

Main effect of frequency:

\begin{Shaded}
\begin{Highlighting}[]
\NormalTok{frequency_beta<-}\StringTok{ }\KeywordTok{logodds}\NormalTok{(highfreq_mean) }\OperatorTok{-}\StringTok{ }\KeywordTok{logodds}\NormalTok{(lowfreq_mean)}
\end{Highlighting}
\end{Shaded}

Main effect of learning:

\begin{Shaded}
\begin{Highlighting}[]
\CommentTok{#mean}
\NormalTok{LF_mean <-}\StringTok{ }\KeywordTok{mean}\NormalTok{(}\DecValTok{38}\NormalTok{, }\DecValTok{88}\NormalTok{)}
\NormalTok{FL_mean <-}\StringTok{ }\KeywordTok{mean}\NormalTok{(}\DecValTok{78}\NormalTok{, }\DecValTok{98}\NormalTok{)}

\NormalTok{n <-}\StringTok{ }\KeywordTok{c}\NormalTok{(}\DecValTok{16}\NormalTok{)}

\CommentTok{#sd}
\NormalTok{LF_sd <-}\StringTok{ }\KeywordTok{c}\NormalTok{(}\DecValTok{5}\OperatorTok{*}\KeywordTok{sqrt}\NormalTok{(n)) }\CommentTok{#how can be possible that learnings have the same se?}
\NormalTok{FL_sd <-}\StringTok{ }\KeywordTok{c}\NormalTok{(}\DecValTok{5}\OperatorTok{*}\KeywordTok{sqrt}\NormalTok{(n))}
\end{Highlighting}
\end{Shaded}

\begin{Shaded}
\begin{Highlighting}[]
\NormalTok{learning_beta <-}\StringTok{ }\KeywordTok{logodds}\NormalTok{(FL_mean) }\OperatorTok{-}\StringTok{ }\KeywordTok{logodds}\NormalTok{(LF_mean)}
\CommentTok{#positive > higher in the FL}
\end{Highlighting}
\end{Shaded}

Interaction between freq and learning:

Frequency effect (high-low) is greater in the LF than in FL:

\begin{Shaded}
\begin{Highlighting}[]
\CommentTok{#(logodds(highfreq_FL)-logodds(lowfreq_FL))- (logodds(highfreq_LF)-logodds(lowfreq_LF))}
\NormalTok{freqBylearning_beta <-}\StringTok{ }\NormalTok{(}\KeywordTok{logodds}\NormalTok{(}\DecValTok{98}\NormalTok{)}\OperatorTok{-}\KeywordTok{logodds}\NormalTok{(}\DecValTok{78}\NormalTok{))}\OperatorTok{-}\StringTok{ }\NormalTok{(}\KeywordTok{logodds}\NormalTok{(}\DecValTok{88}\NormalTok{)}\OperatorTok{-}\KeywordTok{logodds}\NormalTok{(}\DecValTok{38}\NormalTok{))}\OperatorTok{*-}\DecValTok{1}
\end{Highlighting}
\end{Shaded}

\hypertarget{glmms-with-all-tests-separately}{%
\subsection{GLMMs with all tests
separately}\label{glmms-with-all-tests-separately}}

Picture label

\begin{Shaded}
\begin{Highlighting}[]
\NormalTok{pictureLabel}\OperatorTok{$}\NormalTok{frequency <-}\StringTok{ }\KeywordTok{as.factor}\NormalTok{(pictureLabel}\OperatorTok{$}\NormalTok{frequency)}
\NormalTok{plyr}\OperatorTok{::}\KeywordTok{revalue}\NormalTok{(pictureLabel}\OperatorTok{$}\NormalTok{frequency, }\KeywordTok{c}\NormalTok{(}\StringTok{"25"}\NormalTok{=}\StringTok{"low"}\NormalTok{))->}\StringTok{ }\NormalTok{pictureLabel}\OperatorTok{$}\NormalTok{frequency;}
\NormalTok{plyr}\OperatorTok{::}\KeywordTok{revalue}\NormalTok{(pictureLabel}\OperatorTok{$}\NormalTok{frequency, }\KeywordTok{c}\NormalTok{(}\StringTok{"75"}\NormalTok{=}\StringTok{"high"}\NormalTok{))->}\StringTok{ }\NormalTok{pictureLabel}\OperatorTok{$}\NormalTok{frequency;}

\NormalTok{pictureLabel}\OperatorTok{$}\NormalTok{learning =}\StringTok{ }\KeywordTok{relevel}\NormalTok{(pictureLabel}\OperatorTok{$}\NormalTok{learning, }\DataTypeTok{ref =} \StringTok{"LF"}\NormalTok{)}
\NormalTok{pictureLabel}\OperatorTok{$}\NormalTok{frequency =}\StringTok{ }\KeywordTok{relevel}\NormalTok{(pictureLabel}\OperatorTok{$}\NormalTok{frequency, }\DataTypeTok{ref =} \StringTok{"low"}\NormalTok{)}
\NormalTok{pictureLabel <-}\StringTok{ }\KeywordTok{lizCenter}\NormalTok{(pictureLabel, }\KeywordTok{list}\NormalTok{(}\StringTok{"learning"}\NormalTok{ , }\StringTok{"frequency"}\NormalTok{, }\StringTok{"task"}\NormalTok{))}
\end{Highlighting}
\end{Shaded}

\begin{Shaded}
\begin{Highlighting}[]
\KeywordTok{summarySEwithin}\NormalTok{(}\DataTypeTok{data =}\NormalTok{ pictureLabel[pictureLabel}\OperatorTok{$}\NormalTok{rt }\OperatorTok{>}\StringTok{ }\DecValTok{100} \OperatorTok{&}\StringTok{ }\OperatorTok{!}\NormalTok{(pictureLabel}\OperatorTok{$}\NormalTok{subjID }\OperatorTok\StringTok{ }\NormalTok{badsubjs),], }\DataTypeTok{measurevar =} \StringTok{"acc"}\NormalTok{, }\DataTypeTok{betweenvars =} \StringTok{"learning"}\NormalTok{, }\DataTypeTok{withinvars =} \StringTok{"frequency"}\NormalTok{, }\DataTypeTok{idvar =} \StringTok{"subjID"}\NormalTok{, }\DataTypeTok{conf.interval =} \FloatTok{.95}\NormalTok{)}
\end{Highlighting}
\end{Shaded}

\begin{verbatim}
##   learning frequency   N       acc  acc_norm        sd         se         ci
## 1       FL      high 582 0.7079038 0.6729505 0.5831509 0.02417238 0.04747590
## 2       FL       low 611 0.4877250 0.4550235 0.6269022 0.02536175 0.04980694
## 3       LF      high 465 0.6688172 0.7084257 0.5947813 0.02758232 0.05420174
## 4       LF       low 494 0.3785425 0.4228856 0.6642473 0.02988590 0.05871944
\end{verbatim}

\begin{Shaded}
\begin{Highlighting}[]
\NormalTok{piclab_model <-}\StringTok{ }\KeywordTok{glmer}\NormalTok{(acc }\OperatorTok{~}\StringTok{  }\NormalTok{frequency}\OperatorTok{*}\NormalTok{learning }\OperatorTok{+}\StringTok{ }\NormalTok{(frequency}\OperatorTok{|}\NormalTok{subjID), }
         \DataTypeTok{data =}\NormalTok{ pictureLabel[pictureLabel}\OperatorTok{$}\NormalTok{rt }\OperatorTok{>}\StringTok{ }\DecValTok{100} \OperatorTok{&}\StringTok{ }\OperatorTok{!}\NormalTok{(pictureLabel}\OperatorTok{$}\NormalTok{subjID }\OperatorTok\StringTok{ }\NormalTok{badsubjs),], }
         \DataTypeTok{family=}\StringTok{"binomial"}\NormalTok{,}
         \DataTypeTok{control=}\KeywordTok{glmerControl}\NormalTok{(}\DataTypeTok{optimizer =} \StringTok{"bobyqa"}\NormalTok{))}

\NormalTok{adjusted.piclab_model =}\StringTok{ }\KeywordTok{adjust_intercept_model}\NormalTok{(piclab_model, }\DataTypeTok{chance =} \KeywordTok{log}\NormalTok{(}\FloatTok{0.33}\OperatorTok{/}\NormalTok{(}\DecValTok{1}\FloatTok{-0.33}\NormalTok{)))}
\KeywordTok{round}\NormalTok{(adjusted.piclab_model,}\DecValTok{5}\NormalTok{)}
\end{Highlighting}
\end{Shaded}

\begin{verbatim}
##                          Estimate Std. Error  z value Pr(>|z|)
## (Intercept)               0.09817    0.21988  0.44649  0.65524
## frequencyhigh             1.64364    0.35383  4.64523  0.00000
## learningFL                0.55500    0.29760  1.86492  0.06219
## frequencyhigh:learningFL -0.18451    0.47891 -0.38527  0.70004
\end{verbatim}

\begin{Shaded}
\begin{Highlighting}[]
\NormalTok{piclab_model.emm <-}\StringTok{ }\KeywordTok{emmeans}\NormalTok{(piclab_model , }\OperatorTok{~}\StringTok{ }\NormalTok{frequency}\OperatorTok{*}\StringTok{ }\NormalTok{learning )}
\KeywordTok{contrast}\NormalTok{(piclab_model.emm, }\StringTok{"consec"}\NormalTok{,  }\DataTypeTok{simple =} \StringTok{"each"}\NormalTok{, }\DataTypeTok{combine =}\NormalTok{ F, }\DataTypeTok{adjust =} \StringTok{"bonferroni"}\NormalTok{)}
\end{Highlighting}
\end{Shaded}

\begin{verbatim}
## $`simple contrasts for frequency`
## learning = LF:
##  contrast   estimate    SE  df z.ratio p.value
##  high - low     1.64 0.354 Inf 4.645   <.0001 
## 
## learning = FL:
##  contrast   estimate    SE  df z.ratio p.value
##  high - low     1.46 0.328 Inf 4.448   <.0001 
## 
## Results are given on the log odds ratio (not the response) scale. 
## 
## $`simple contrasts for learning`
## frequency = low:
##  contrast estimate    SE  df z.ratio p.value
##  FL - LF     0.555 0.298 Inf 1.865   0.0622 
## 
## frequency = high:
##  contrast estimate    SE  df z.ratio p.value
##  FL - LF     0.370 0.387 Inf 0.957   0.3386 
## 
## Results are given on the log odds ratio (not the response) scale.
\end{verbatim}

Label picture

\begin{Shaded}
\begin{Highlighting}[]
\NormalTok{labelPicture}\OperatorTok{$}\NormalTok{frequency <-}\StringTok{ }\KeywordTok{as.factor}\NormalTok{(labelPicture}\OperatorTok{$}\NormalTok{frequency)}
\NormalTok{plyr}\OperatorTok{::}\KeywordTok{revalue}\NormalTok{(labelPicture}\OperatorTok{$}\NormalTok{frequency, }\KeywordTok{c}\NormalTok{(}\StringTok{"25"}\NormalTok{=}\StringTok{"low"}\NormalTok{))->}\StringTok{ }\NormalTok{labelPicture}\OperatorTok{$}\NormalTok{frequency;}
\NormalTok{plyr}\OperatorTok{::}\KeywordTok{revalue}\NormalTok{(labelPicture}\OperatorTok{$}\NormalTok{frequency, }\KeywordTok{c}\NormalTok{(}\StringTok{"75"}\NormalTok{=}\StringTok{"high"}\NormalTok{))->}\StringTok{ }\NormalTok{labelPicture}\OperatorTok{$}\NormalTok{frequency;}

\NormalTok{labelPicture}\OperatorTok{$}\NormalTok{learning =}\StringTok{ }\KeywordTok{relevel}\NormalTok{(labelPicture}\OperatorTok{$}\NormalTok{learning, }\DataTypeTok{ref =} \StringTok{"LF"}\NormalTok{)}
\NormalTok{labelPicture}\OperatorTok{$}\NormalTok{frequency =}\StringTok{ }\KeywordTok{relevel}\NormalTok{(labelPicture}\OperatorTok{$}\NormalTok{frequency, }\DataTypeTok{ref =} \StringTok{"low"}\NormalTok{)}
\NormalTok{labelPicture <-}\StringTok{ }\KeywordTok{lizCenter}\NormalTok{(labelPicture, }\KeywordTok{list}\NormalTok{(}\StringTok{"learning"}\NormalTok{ , }\StringTok{"frequency"}\NormalTok{, }\StringTok{"task"}\NormalTok{))}
\end{Highlighting}
\end{Shaded}

\begin{Shaded}
\begin{Highlighting}[]
\KeywordTok{summarySEwithin}\NormalTok{(}\DataTypeTok{data =}\NormalTok{ labelPicture[labelPicture}\OperatorTok{$}\NormalTok{rt }\OperatorTok{>}\StringTok{ }\DecValTok{100} \OperatorTok{&}\StringTok{ }\NormalTok{labelPicture}\OperatorTok{$}\NormalTok{rt }\OperatorTok{<=}\DecValTok{2500} \OperatorTok{&}\StringTok{ }\OperatorTok{!}\NormalTok{(labelPicture}\OperatorTok{$}\NormalTok{subjID }\OperatorTok\StringTok{ }\NormalTok{badsubjs),], }\DataTypeTok{measurevar =} \StringTok{"acc"}\NormalTok{, }\DataTypeTok{betweenvars =} \StringTok{"learning"}\NormalTok{, }\DataTypeTok{withinvars =} \StringTok{"frequency"}\NormalTok{, }\DataTypeTok{idvar =} \StringTok{"subjID"}\NormalTok{, }\DataTypeTok{conf.interval =} \FloatTok{.95}\NormalTok{)}
\end{Highlighting}
\end{Shaded}

\begin{verbatim}
##   learning frequency   N       acc  acc_norm        sd         se         ci
## 1       FL      high 586 0.7440273 0.7151810 0.5322962 0.02198895 0.04318691
## 2       FL       low 554 0.4458484 0.4244508 0.5849702 0.02485300 0.04881783
## 3       LF      high 484 0.6508264 0.6784148 0.6177757 0.02808071 0.05517544
## 4       LF       low 460 0.4304348 0.4639248 0.6508264 0.03034494 0.05963222
\end{verbatim}

\begin{Shaded}
\begin{Highlighting}[]
\NormalTok{labpic_model <-}\StringTok{ }\KeywordTok{glmer}\NormalTok{(acc }\OperatorTok{~}\StringTok{  }\NormalTok{frequency.ct}\OperatorTok{*}\NormalTok{learning.ct }\OperatorTok{+}\StringTok{ }\NormalTok{(frequency.ct}\OperatorTok{|}\NormalTok{subjID), }
         \DataTypeTok{data =}\NormalTok{ labelPicture[labelPicture}\OperatorTok{$}\NormalTok{rt }\OperatorTok{>}\StringTok{ }\DecValTok{100} \OperatorTok{&}\StringTok{ }\NormalTok{labelPicture}\OperatorTok{$}\NormalTok{rt }\OperatorTok{<=}\DecValTok{2500} \OperatorTok{&}\StringTok{ }\OperatorTok{!}\NormalTok{(labelPicture}\OperatorTok{$}\NormalTok{subjID }\OperatorTok\StringTok{ }\NormalTok{badsubjs),], }
         \DataTypeTok{family=}\StringTok{"binomial"}\NormalTok{,}
         \DataTypeTok{control=}\KeywordTok{glmerControl}\NormalTok{(}\DataTypeTok{optimizer =} \StringTok{"bobyqa"}\NormalTok{))}

\NormalTok{adjusted.labpic_model =}\StringTok{ }\KeywordTok{adjust_intercept_model}\NormalTok{(labpic_model, }\DataTypeTok{chance =} \KeywordTok{log}\NormalTok{(}\FloatTok{0.33}\OperatorTok{/}\NormalTok{(}\DecValTok{1}\FloatTok{-0.33}\NormalTok{)))}
\KeywordTok{round}\NormalTok{(adjusted.labpic_model,}\DecValTok{5}\NormalTok{)}
\end{Highlighting}
\end{Shaded}

\begin{verbatim}
##                          Estimate Std. Error z value Pr(>|z|)
## (Intercept)               1.32254    0.17272 7.65720  0.00000
## frequency.ct              1.94468    0.35298 5.50936  0.00000
## learning.ct               0.51646    0.33501 1.54163  0.12316
## frequency.ct:learning.ct  0.59238    0.68714 0.86208  0.38864
\end{verbatim}

\begin{Shaded}
\begin{Highlighting}[]
\NormalTok{labpic_model.emm <-}\StringTok{ }\KeywordTok{emmeans}\NormalTok{(labpic_model, }\OperatorTok{~}\StringTok{ }\NormalTok{frequency.ct}\OperatorTok{*}\StringTok{ }\NormalTok{learning.ct )}
\KeywordTok{contrast}\NormalTok{(labpic_model.emm, }\StringTok{"consec"}\NormalTok{,  }\DataTypeTok{simple =} \StringTok{"each"}\NormalTok{, }\DataTypeTok{combine =}\NormalTok{ F, }\DataTypeTok{adjust =} \StringTok{"bonferroni"}\NormalTok{)}
\end{Highlighting}
\end{Shaded}

\begin{verbatim}
## $`simple contrasts for frequency.ct`
## learning.ct = -0.525:
##  contrast                               estimate    SE  df z.ratio p.value
##  0.491247672253259 - -0.508752327746741     1.63 0.505 Inf 3.233   0.0012 
## 
## learning.ct =  0.475:
##  contrast                               estimate    SE  df z.ratio p.value
##  0.491247672253259 - -0.508752327746741     2.23 0.480 Inf 4.637   <.0001 
## 
## Results are given on the log odds ratio (not the response) scale. 
## 
## $`simple contrasts for learning.ct`
## frequency.ct = -0.509:
##  contrast                               estimate    SE  df z.ratio p.value
##  0.475232774674115 - -0.524767225325885    0.215 0.468 Inf 0.460   0.6455 
## 
## frequency.ct =  0.491:
##  contrast                               estimate    SE  df z.ratio p.value
##  0.475232774674115 - -0.524767225325885    0.807 0.491 Inf 1.643   0.1003 
## 
## Results are given on the log odds ratio (not the response) scale.
\end{verbatim}

Contingency judgement

\begin{Shaded}
\begin{Highlighting}[]
\NormalTok{plyr}\OperatorTok{::}\KeywordTok{revalue}\NormalTok{(}\KeywordTok{as.factor}\NormalTok{(conjudge}\OperatorTok{$}\NormalTok{frequency), }\KeywordTok{c}\NormalTok{(}\StringTok{"25"}\NormalTok{=}\StringTok{"low"}\NormalTok{))->}\StringTok{ }\NormalTok{conjudge}\OperatorTok{$}\NormalTok{frequency;}
\end{Highlighting}
\end{Shaded}

\begin{verbatim}
## The following `from` values were not present in `x`: 25
\end{verbatim}

\begin{Shaded}
\begin{Highlighting}[]
\NormalTok{plyr}\OperatorTok{::}\KeywordTok{revalue}\NormalTok{(}\KeywordTok{as.factor}\NormalTok{(conjudge}\OperatorTok{$}\NormalTok{frequency), }\KeywordTok{c}\NormalTok{(}\StringTok{"75"}\NormalTok{=}\StringTok{"high"}\NormalTok{))->}\StringTok{ }\NormalTok{conjudge}\OperatorTok{$}\NormalTok{frequency;}
\end{Highlighting}
\end{Shaded}

\begin{verbatim}
## The following `from` values were not present in `x`: 75
\end{verbatim}

\begin{Shaded}
\begin{Highlighting}[]
\NormalTok{conjudge}\OperatorTok{$}\NormalTok{learning =}\StringTok{ }\KeywordTok{relevel}\NormalTok{(conjudge}\OperatorTok{$}\NormalTok{learning, }\DataTypeTok{ref =} \StringTok{"FL"}\NormalTok{)}
\NormalTok{conjudge}\OperatorTok{$}\NormalTok{frequency =}\StringTok{ }\KeywordTok{relevel}\NormalTok{(conjudge}\OperatorTok{$}\NormalTok{frequency, }\DataTypeTok{ref =} \StringTok{"low"}\NormalTok{)}
\NormalTok{conjudge <-}\StringTok{ }\KeywordTok{lizCenter}\NormalTok{(conjudge, }\KeywordTok{list}\NormalTok{(}\StringTok{"learning"}\NormalTok{ , }\StringTok{"frequency"}\NormalTok{))}
\end{Highlighting}
\end{Shaded}

\begin{Shaded}
\begin{Highlighting}[]
\NormalTok{conjudge_model <-}\StringTok{ }\KeywordTok{lmer}\NormalTok{(resp }\OperatorTok{~}\StringTok{  }\NormalTok{learning }\OperatorTok{*}\StringTok{ }\NormalTok{frequency }\OperatorTok{+}\NormalTok{(frequency}\OperatorTok{|}\NormalTok{subjID), }
         \DataTypeTok{data =}\NormalTok{ conjudge[}\OperatorTok{!}\NormalTok{(conjudge}\OperatorTok{$}\NormalTok{subjID }\OperatorTok\StringTok{ }\NormalTok{badsubjs) }\OperatorTok{&}\StringTok{ }\NormalTok{conjudge}\OperatorTok{$}\NormalTok{acc}\OperatorTok{==}\DecValTok{0}\NormalTok{,])}
\end{Highlighting}
\end{Shaded}

\begin{Shaded}
\begin{Highlighting}[]
\NormalTok{car}\OperatorTok{::}\KeywordTok{Anova}\NormalTok{(conjudge_model)}
\end{Highlighting}
\end{Shaded}

\begin{verbatim}
## Analysis of Deviance Table (Type II Wald chisquare tests)
## 
## Response: resp
##                      Chisq Df Pr(>Chisq)    
## learning            1.3493  1  0.2454058    
## frequency          11.7018  1  0.0006244 ***
## learning:frequency  0.7078  1  0.4001847    
## ---
## Signif. codes:  0 '***' 0.001 '**' 0.01 '*' 0.05 '.' 0.1 ' ' 1
\end{verbatim}

\begin{Shaded}
\begin{Highlighting}[]
\NormalTok{conjudge_model.emm <-}\StringTok{ }\KeywordTok{emmeans}\NormalTok{(conjudge_model , }\OperatorTok{~}\StringTok{ }\NormalTok{learning}\OperatorTok{*}\StringTok{ }\NormalTok{frequency )}
\KeywordTok{contrast}\NormalTok{(conjudge_model.emm, }\StringTok{"consec"}\NormalTok{,  }\DataTypeTok{simple =} \StringTok{"each"}\NormalTok{, }\DataTypeTok{combine =}\NormalTok{ F, }\DataTypeTok{adjust =} \StringTok{"bonferroni"}\NormalTok{)}
\end{Highlighting}
\end{Shaded}

\begin{verbatim}
## $`simple contrasts for learning`
## frequency = low:
##  contrast estimate   SE   df t.ratio p.value
##  LF - FL      5.48 11.0 70.9 0.500   0.6187 
## 
## frequency = high:
##  contrast estimate   SE   df t.ratio p.value
##  LF - FL     16.13 11.2 71.8 1.434   0.1559 
## 
## Degrees-of-freedom method: kenward-roger 
## 
## $`simple contrasts for frequency`
## learning = FL:
##  contrast   estimate   SE df t.ratio p.value
##  high - low    -26.3 8.44 70 -3.111  0.0027 
## 
## learning = LF:
##  contrast   estimate   SE df t.ratio p.value
##  high - low    -15.6 9.45 72 -1.652  0.1030 
## 
## Degrees-of-freedom method: kenward-roger
\end{verbatim}

\hypertarget{combine-both-generalization-tasks-in-one-dataset}{%
\subsection{Combine both generalization tasks in one
dataset}\label{combine-both-generalization-tasks-in-one-dataset}}

I'm going to combine both generalization tasks in one single dataset
called genTask

\begin{Shaded}
\begin{Highlighting}[]
\NormalTok{genTask <-}\StringTok{ }\KeywordTok{rbind}\NormalTok{(labelPicture[labelPicture}\OperatorTok{$}\NormalTok{rt }\OperatorTok{>}\StringTok{ }\DecValTok{100} \OperatorTok{&}\StringTok{ }\OperatorTok{!}\NormalTok{(labelPicture}\OperatorTok{$}\NormalTok{subjID }\OperatorTok\StringTok{ }\NormalTok{badsubjs),], }
\NormalTok{                 pictureLabel[pictureLabel}\OperatorTok{$}\NormalTok{rt }\OperatorTok{>}\StringTok{ }\DecValTok{100} \OperatorTok{&}\StringTok{ }\OperatorTok{!}\NormalTok{(pictureLabel}\OperatorTok{$}\NormalTok{subjID }\OperatorTok\StringTok{ }\NormalTok{badsubjs),])}

\NormalTok{genTask}\OperatorTok{$}\NormalTok{frequency <-}\StringTok{ }\KeywordTok{as.factor}\NormalTok{(genTask}\OperatorTok{$}\NormalTok{frequency)}
\NormalTok{plyr}\OperatorTok{::}\KeywordTok{revalue}\NormalTok{(genTask}\OperatorTok{$}\NormalTok{frequency, }\KeywordTok{c}\NormalTok{(}\StringTok{"25"}\NormalTok{=}\StringTok{"low"}\NormalTok{))->}\StringTok{ }\NormalTok{genTask}\OperatorTok{$}\NormalTok{frequency;}
\end{Highlighting}
\end{Shaded}

\begin{verbatim}
## The following `from` values were not present in `x`: 25
\end{verbatim}

\begin{Shaded}
\begin{Highlighting}[]
\NormalTok{plyr}\OperatorTok{::}\KeywordTok{revalue}\NormalTok{(genTask}\OperatorTok{$}\NormalTok{frequency, }\KeywordTok{c}\NormalTok{(}\StringTok{"75"}\NormalTok{=}\StringTok{"high"}\NormalTok{))->}\StringTok{ }\NormalTok{genTask}\OperatorTok{$}\NormalTok{frequency;}
\end{Highlighting}
\end{Shaded}

\begin{verbatim}
## The following `from` values were not present in `x`: 75
\end{verbatim}

Relevel the variables:

\begin{Shaded}
\begin{Highlighting}[]
\NormalTok{genTask}\OperatorTok{$}\NormalTok{learning =}\StringTok{ }\KeywordTok{relevel}\NormalTok{(genTask}\OperatorTok{$}\NormalTok{learning, }\DataTypeTok{ref =} \StringTok{"LF"}\NormalTok{)}
\NormalTok{genTask}\OperatorTok{$}\NormalTok{frequency =}\StringTok{ }\KeywordTok{relevel}\NormalTok{(genTask}\OperatorTok{$}\NormalTok{frequency, }\DataTypeTok{ref =} \StringTok{"low"}\NormalTok{)}
\NormalTok{genTask <-}\StringTok{ }\KeywordTok{lizCenter}\NormalTok{(genTask, }\KeywordTok{list}\NormalTok{(}\StringTok{"learning"}\NormalTok{ , }\StringTok{"frequency"}\NormalTok{, }\StringTok{"task"}\NormalTok{))}
\end{Highlighting}
\end{Shaded}

\hypertarget{the-model}{%
\subsection{The model}\label{the-model}}

\begin{Shaded}
\begin{Highlighting}[]
\NormalTok{genTask_model <-}\StringTok{ }\KeywordTok{glmer}\NormalTok{(acc }\OperatorTok{~}\StringTok{  }\NormalTok{frequency.ct}\OperatorTok{*}\NormalTok{learning.ct }\OperatorTok{+}\StringTok{ }\NormalTok{task.ct }\OperatorTok{+}\StringTok{ }\NormalTok{(frequency.ct}\OperatorTok{|}\NormalTok{subjID) , }
         \DataTypeTok{data =}\NormalTok{ genTask, }
         \DataTypeTok{family=}\StringTok{"binomial"}\NormalTok{,}
         \DataTypeTok{control=}\KeywordTok{glmerControl}\NormalTok{(}\DataTypeTok{optimizer =} \StringTok{"bobyqa"}\NormalTok{))}

\NormalTok{adjusted.genTask_model =}\StringTok{ }\KeywordTok{adjust_intercept_model}\NormalTok{(genTask_model, }\DataTypeTok{chance =} \KeywordTok{log}\NormalTok{(}\FloatTok{0.33}\OperatorTok{/}\NormalTok{(}\DecValTok{1}\FloatTok{-0.33}\NormalTok{)))}
\KeywordTok{round}\NormalTok{(adjusted.genTask_model,}\DecValTok{5}\NormalTok{)}
\end{Highlighting}
\end{Shaded}

\begin{verbatim}
##                          Estimate Std. Error  z value Pr(>|z|)
## (Intercept)               1.22227    0.13587  8.99560  0.00000
## frequency.ct              1.68180    0.26120  6.43863  0.00000
## learning.ct               0.47846    0.27079  1.76693  0.07724
## task.ct                  -0.03490    0.07604 -0.45891  0.64630
## frequency.ct:learning.ct  0.28263    0.52044  0.54305  0.58710
\end{verbatim}

Further inspection:

\begin{Shaded}
\begin{Highlighting}[]
\NormalTok{genTask_model.emm <-}\StringTok{ }\KeywordTok{emmeans}\NormalTok{(genTask_model , }\OperatorTok{~}\StringTok{ }\NormalTok{frequency.ct }\OperatorTok{*}\StringTok{ }\NormalTok{learning.ct )}
\KeywordTok{contrast}\NormalTok{(genTask_model.emm, }\StringTok{"consec"}\NormalTok{,  }\DataTypeTok{simple =} \StringTok{"each"}\NormalTok{, }\DataTypeTok{combine =}\NormalTok{ F, }\DataTypeTok{adjust =} \StringTok{"bonferroni"}\NormalTok{)}
\end{Highlighting}
\end{Shaded}

\begin{verbatim}
## $`simple contrasts for frequency.ct`
## learning.ct = -0.55:
##  contrast                               estimate    SE  df z.ratio p.value
##  0.502322340919647 - -0.497677659080353     1.53 0.382 Inf 3.994   0.0001 
## 
## learning.ct =  0.45:
##  contrast                               estimate    SE  df z.ratio p.value
##  0.502322340919647 - -0.497677659080353     1.81 0.356 Inf 5.087   <.0001 
## 
## Results are averaged over the levels of: task.ct 
## Results are given on the log odds ratio (not the response) scale. 
## 
## $`simple contrasts for learning.ct`
## frequency.ct = -0.498:
##  contrast                             estimate    SE  df z.ratio p.value
##  0.44960520204366 - -0.55039479795634    0.338 0.349 Inf 0.967   0.3333 
## 
## frequency.ct =  0.502:
##  contrast                             estimate    SE  df z.ratio p.value
##  0.44960520204366 - -0.55039479795634    0.620 0.400 Inf 1.549   0.1213 
## 
## Results are averaged over the levels of: task.ct 
## Results are given on the log odds ratio (not the response) scale.
\end{verbatim}

Okay, with both tasks together the take home message is the following:

\begin{itemize}
\item
  Main effect of frequency, with high frequency having higher accuracy
  than low frequency in both learnings.
\item
  Main effect of learning, with FL learning having higher accuracy in
  the high frequency condition.
\item
  No difference between learnings in the low frequency condition.
\item
  No difference between tasks
\end{itemize}

\begin{Shaded}
\begin{Highlighting}[]
\NormalTok{genTask }\OperatorTok
\StringTok{  }\KeywordTok{group_by}\NormalTok{(frequency, learning) }\OperatorTok
\StringTok{  }\KeywordTok{summarise}\NormalTok{(}\DataTypeTok{mean =} \KeywordTok{mean}\NormalTok{(acc))}
\end{Highlighting}
\end{Shaded}

\begin{verbatim}
##        mean
## 1 0.5652578
\end{verbatim}

\begin{Shaded}
\begin{Highlighting}[]
\KeywordTok{summarySEwithin}\NormalTok{(}\DataTypeTok{data =}\NormalTok{ genTask, }\DataTypeTok{measurevar =} \StringTok{"acc"}\NormalTok{, }\DataTypeTok{betweenvars =} \StringTok{"learning"}\NormalTok{, }\DataTypeTok{withinvars =} \StringTok{"frequency"}\NormalTok{, }\DataTypeTok{idvar =} \StringTok{"subjID"}\NormalTok{, }\DataTypeTok{conf.interval =} \FloatTok{.95}\NormalTok{)}
\end{Highlighting}
\end{Shaded}

\begin{verbatim}
##   learning frequency    N       acc  acc_norm        sd         se         ci
## 1       FL      high 1182 0.7225042 0.6930796 0.5647076 0.01642536 0.03222614
## 2       FL       low 1188 0.4638047 0.4380815 0.6156339 0.01786135 0.03504334
## 3       LF      high  961 0.6566077 0.6885886 0.6131101 0.01977774 0.03881260
## 4       LF       low  975 0.4082051 0.4436979 0.6707224 0.02148031 0.04215301
\end{verbatim}

I'm going to create a table with the estimates:

\begin{Shaded}
\begin{Highlighting}[]
\NormalTok{genTask_bf =}\StringTok{ }\KeywordTok{data.frame}\NormalTok{(}
    \DataTypeTok{condition =} \KeywordTok{c}\NormalTok{(}
                   \StringTok{"frequency by learning"}\NormalTok{,}
                   \StringTok{"learning"}\NormalTok{,}
                   \StringTok{"frequency"}\NormalTok{,}
                   \StringTok{"task"}
\NormalTok{                   ),}
               
    \DataTypeTok{meandiff =} \KeywordTok{c}\NormalTok{(}
      \KeywordTok{round}\NormalTok{(}\KeywordTok{summary}\NormalTok{(genTask_model)}\OperatorTok{$}\NormalTok{coefficients[}\StringTok{"frequency.ct:learning.ct"}\NormalTok{, }\StringTok{"Estimate"}\NormalTok{],}\DecValTok{3}\NormalTok{),}
       \KeywordTok{round}\NormalTok{(}\KeywordTok{summary}\NormalTok{(genTask_model)}\OperatorTok{$}\NormalTok{coefficients[}\StringTok{"learning.ct"}\NormalTok{, }\StringTok{"Estimate"}\NormalTok{],}\DecValTok{3}\NormalTok{),}
       \KeywordTok{round}\NormalTok{(}\KeywordTok{summary}\NormalTok{(genTask_model)}\OperatorTok{$}\NormalTok{coefficients[}\StringTok{"frequency.ct"}\NormalTok{, }\StringTok{"Estimate"}\NormalTok{],}\DecValTok{3}\NormalTok{),}
       \KeywordTok{round}\NormalTok{(}\KeywordTok{summary}\NormalTok{(genTask_model)}\OperatorTok{$}\NormalTok{coefficients[}\StringTok{"task.ct"}\NormalTok{, }\StringTok{"Estimate"}\NormalTok{],}\DecValTok{3}\NormalTok{)}
\NormalTok{       ),}
    
    \DataTypeTok{se =} \KeywordTok{c}\NormalTok{(}
      \KeywordTok{round}\NormalTok{(}\KeywordTok{summary}\NormalTok{(genTask_model)}\OperatorTok{$}\NormalTok{coefficients[}\StringTok{"frequency.ct:learning.ct"}\NormalTok{, }\StringTok{"Std. Error"}\NormalTok{],}\DecValTok{3}\NormalTok{),}
       \KeywordTok{round}\NormalTok{(}\KeywordTok{summary}\NormalTok{(genTask_model)}\OperatorTok{$}\NormalTok{coefficients[}\StringTok{"learning.ct"}\NormalTok{, }\StringTok{"Std. Error"}\NormalTok{],}\DecValTok{3}\NormalTok{),}
       \KeywordTok{round}\NormalTok{(}\KeywordTok{summary}\NormalTok{(genTask_model)}\OperatorTok{$}\NormalTok{coefficients[}\StringTok{"frequency.ct"}\NormalTok{, }\StringTok{"Std. Error"}\NormalTok{],}\DecValTok{3}\NormalTok{),}
       \KeywordTok{round}\NormalTok{(}\KeywordTok{summary}\NormalTok{(genTask_model)}\OperatorTok{$}\NormalTok{coefficients[}\StringTok{"task.ct"}\NormalTok{, }\StringTok{"Std. Error"}\NormalTok{],}\DecValTok{3}\NormalTok{)}
\NormalTok{       )}
\NormalTok{)}

\NormalTok{genTask_bf}
\end{Highlighting}
\end{Shaded}

\begin{verbatim}
##               condition meandiff    se
## 1 frequency by learning    0.283 0.520
## 2              learning    0.478 0.271
## 3             frequency    1.682 0.261
## 4                  task   -0.035 0.076
\end{verbatim}

\hypertarget{bf-for-frequency}{%
\subsection{BF for Frequency:}\label{bf-for-frequency}}

\begin{Shaded}
\begin{Highlighting}[]
\KeywordTok{Bf}\NormalTok{(}\DataTypeTok{sd =}\NormalTok{ genTask_bf[genTask_bf}\OperatorTok{$}\NormalTok{condition}\OperatorTok{==}\StringTok{'frequency'}\NormalTok{,]}\OperatorTok{$}\NormalTok{se, }
   \DataTypeTok{obtained =}\NormalTok{ genTask_bf[genTask_bf}\OperatorTok{$}\NormalTok{condition}\OperatorTok{==}\StringTok{'frequency'}\NormalTok{,]}\OperatorTok{$}\NormalTok{meandiff, }
   \DataTypeTok{uniform =} \DecValTok{0}\NormalTok{, }
   \DataTypeTok{sdtheory =}\NormalTok{ highfreq_sd, }
   \DataTypeTok{meanoftheory =}\NormalTok{ frequency_beta, }
   \DataTypeTok{tail =} \DecValTok{1}\NormalTok{)}
\end{Highlighting}
\end{Shaded}

\begin{verbatim}
## $LikelihoodTheory
## [1] 0.028197
## 
## $Likelihoodnull
## [1] 1.465403e-09
## 
## $BayesFactor
## [1] 19241809
\end{verbatim}

\hypertarget{bf-for-learning}{%
\subsection{BF for learning:}\label{bf-for-learning}}

\begin{Shaded}
\begin{Highlighting}[]
\KeywordTok{Bf}\NormalTok{(}\DataTypeTok{sd =}\NormalTok{ genTask_bf[genTask_bf}\OperatorTok{$}\NormalTok{condition}\OperatorTok{==}\StringTok{'learning'}\NormalTok{,]}\OperatorTok{$}\NormalTok{se, }
   \DataTypeTok{obtained =}\NormalTok{ genTask_bf[genTask_bf}\OperatorTok{$}\NormalTok{condition}\OperatorTok{==}\StringTok{'learning'}\NormalTok{,]}\OperatorTok{$}\NormalTok{meandiff, }
   \DataTypeTok{uniform =} \DecValTok{0}\NormalTok{, }
   \DataTypeTok{sdtheory =}\NormalTok{ LF_sd, }
   \DataTypeTok{meanoftheory =}\NormalTok{ learning_beta, }
   \DataTypeTok{tail =} \DecValTok{1}\NormalTok{)}
\end{Highlighting}
\end{Shaded}

\begin{verbatim}
## $LikelihoodTheory
## [1] 0.03823377
## 
## $Likelihoodnull
## [1] 0.3107198
## 
## $BayesFactor
## [1] 0.123049
\end{verbatim}

\hypertarget{bf-for-the-interaction-frequency-by-learning}{%
\subsection{BF for the interaction frequency by
learning}\label{bf-for-the-interaction-frequency-by-learning}}

\begin{Shaded}
\begin{Highlighting}[]
\KeywordTok{Bf}\NormalTok{(}\DataTypeTok{sd =}\NormalTok{ genTask_bf[genTask_bf}\OperatorTok{$}\NormalTok{condition}\OperatorTok{==}\StringTok{'frequency by learning'}\NormalTok{,]}\OperatorTok{$}\NormalTok{se, }
   \DataTypeTok{obtained =}\NormalTok{ genTask_bf[genTask_bf}\OperatorTok{$}\NormalTok{condition}\OperatorTok{==}\StringTok{'frequency by learning'}\NormalTok{,]}\OperatorTok{$}\NormalTok{meandiff, }
   \DataTypeTok{uniform =} \DecValTok{0}\NormalTok{, }
   \DataTypeTok{sdtheory =}\NormalTok{ LF_sd, }\CommentTok{#don't know how to compute sd of the interaction}
   \DataTypeTok{meanoftheory =}\NormalTok{ freqBylearning_beta, }
   \DataTypeTok{tail =} \DecValTok{1}\NormalTok{)}
\end{Highlighting}
\end{Shaded}

\begin{verbatim}
## $LikelihoodTheory
## [1] 0.02852534
## 
## $Likelihoodnull
## [1] 0.6615924
## 
## $BayesFactor
## [1] 0.04311618
\end{verbatim}

\begin{Shaded}
\begin{Highlighting}[]
\KeywordTok{rm}\NormalTok{(speedacc, n, lowfreq_mean, highfreq_mean, lowfreq_sd, highfreq_sd, LF_mean, FL_mean, LF_sd, FL_sd)}
\end{Highlighting}
\end{Shaded}

\hypertarget{summary-of-the-results}{%
\section{Summary of the results}\label{summary-of-the-results}}

We have collected 120 participants. Among these, 63 FL learning and 57
LF learning.

We had four tasks:

\begin{itemize}
\item
  Picture label task
\item
  Label picture task
\item
  Contingency judgement task
\item
  Random dot task (attention check)
\end{itemize}

Participants that scored \textless=.5 accuracy and had \textgreater3
timeouts in the attention check (random dot task) were removed from the
analysis. Participants that skipped completely one of the tasks were
removed. Participants that had very few datapoints, i.e., less than 1/2
also removed. In total for picture label task we had 52 for FL learning,
and 43 for LF learning.

Raw means/sd for the effects.

Label Picture:

\begin{verbatim}
##   learning frequency   N       acc  acc_norm        sd         se         ci
## 1       FL      high 586 0.7440273 0.7151810 0.5322962 0.02198895 0.04318691
## 2       FL       low 554 0.4458484 0.4244508 0.5849702 0.02485300 0.04881783
## 3       LF      high 484 0.6508264 0.6784148 0.6177757 0.02808071 0.05517544
## 4       LF       low 460 0.4304348 0.4639248 0.6508264 0.03034494 0.05963222
\end{verbatim}

Picture Label:

\begin{verbatim}
##   learning frequency   N       acc  acc_norm        sd         se         ci
## 1       FL      high 582 0.7079038 0.6729505 0.5831509 0.02417238 0.04747590
## 2       FL       low 611 0.4877250 0.4550235 0.6269022 0.02536175 0.04980694
## 3       LF      high 465 0.6688172 0.7084257 0.5947813 0.02758232 0.05420174
## 4       LF       low 494 0.3785425 0.4228856 0.6642473 0.02988590 0.05871944
\end{verbatim}

Data Visualization:

\begin{center}\includegraphics{preProcessing_files/figure-latex/unnamed-chunk-132-1} \end{center}

GLMMs models:

Picture label

\begin{verbatim}
##                          Estimate Std. Error  z value Pr(>|z|)
## (Intercept)               0.09817    0.21988  0.44649  0.65524
## frequencyhigh             1.64364    0.35383  4.64523  0.00000
## learningFL                0.55500    0.29760  1.86492  0.06219
## frequencyhigh:learningFL -0.18451    0.47891 -0.38527  0.70004
\end{verbatim}

Label picture

\begin{verbatim}
##                          Estimate Std. Error z value Pr(>|z|)
## (Intercept)               1.32254    0.17272 7.65720  0.00000
## frequency.ct              1.94468    0.35298 5.50936  0.00000
## learning.ct               0.51646    0.33501 1.54163  0.12316
## frequency.ct:learning.ct  0.59238    0.68714 0.86208  0.38864
\end{verbatim}

\textbf{What we have learned from these data:}

\begin{itemize}
\item
  Main effect of frequency, with high frequency having higher accuracy
  than low frequency in both learnings.
\item
  Marginal effect of learning in the pictureLabel task, with FL learning
  having higher accuracy in the low frequency condition.
\item
  No difference between learnings in the high frequency condition,
  although there is a trend for FL being higher than LF in the
  pictureLabel task
\item
  What's important here is that the two tasks seems to behave completely
  differently.
\end{itemize}

This means that the effect of frequency (high vs low) was super robust,
and this is the only thing that we have replicated 100\%. The difference
between learnings unfortunately wasn't there, although we see a trend in
this direction in one task, but not the other. Why is this the case?

\textbf{How do we explain these results:} We don't know for sure,
however, throughout this experiment we have realised several important
details that are not identical to the FLO paper and therefore could have
affected the results:

\begin{itemize}
\item
  Learning: stimuli were pseudo-randomised with exemplars belonging to
  high and low frequency category of one category never displayed
  consequentially.
\item
  The whole FLO experiment was visual, not audio, therefore this might
  cause less ambiguity, i.e., higher accuracy, and perfect balance in
  the test tasks for the trial duration. Also, this would remove the
  confound due to the addition of the sentence, in fact, we speculated
  that the two learnings varies in the contiguity between stimulus and
  label. I.e., FL: {[}fribble{]}+``This was a \ldots. X'' Versus LF:
  ``This is a \ldots.X''+{[}fribble{]} introduces two different types of
  lags between the presentation of the label and the stimulus. We
  speculated that this might cause differences, we don't know how.
\item
  Michael suggested that participants in his original experiment did
  only one of the two tasks, and not both. Exposition to both tasks
  might cause greater noise, especially in the labelPicture task where
  participants see 72 different fribbles. This might cause super
  confusion in the participants. Indeed, I found that from the folder
  Mike has shared with me (later on during this experiment) the number
  of stimuli didn't match with the number of test trials reported in the
  paper.
\end{itemize}

\textbf{What we're going to do next:} We're goint to re-do the
replication! This time for real: by checking for the right amount of
test trials, same fribbles used by Michael and same modality.

\end{document}
